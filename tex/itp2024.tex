\documentclass[a4paper,USenglish,cleveref,autoref]{lipics-v2021}

%% Regular papers for ITP 2024 must:
% be no more than 16 pages in length excluding bibliographic references
% not include an appendix; and
% be in LIPIcs format.
% Submissions will undergo single-blind peer review. 

%This is a template for producing LIPIcs articles. 
%See lipics-v2021-authors-guidelines.pdf for further information.
%for A4 paper format use option "a4paper", for US-letter use option "letterpaper"
%for british hyphenation rules use option "UKenglish", for american hyphenation rules use option "USenglish"
%for section-numbered lemmas etc., use "numberwithinsect"
%for enabling cleveref support, use "cleveref"
%for enabling autoref support, use "autoref"
%for anonymousing the authors (e.g. for double-blind review), add "anonymous"
%for enabling thm-restate support, use "thm-restate"
%for enabling a two-column layout for the author/affilation part (only applicable for > 6 authors), use "authorcolumns"
%for producing a PDF according the PDF/A standard, add "pdfa"

%\pdfoutput=1 %uncomment to ensure pdflatex processing (mandatatory e.g. to submit to arXiv)
%\hideLIPIcs  %uncomment to remove references to LIPIcs series (logo, DOI, ...), e.g. when preparing a pre-final version to be uploaded to arXiv or another public repository

%\graphicspath{{./graphics/}}%helpful if your graphic files are in another directory
\usepackage{microtype}%if unwanted, comment out or use option "draft"
%\pdfoutput=1
\usepackage[utf8]{inputenc}
\usepackage{amsmath}
\usepackage{amssymb}
\usepackage{fdsymbol}
\usepackage{mathpartir}
\usepackage{hyperref}
\usepackage{listings}
\usepackage{subcaption}
% \usepackage{minted}
\usepackage{url}
\usepackage{graphicx}
\usepackage{stmaryrd}
\usepackage{latex/agda}
\usepackage{comment}
\usepackage{todonotes}
\newcommand\pt{\todo[author=PT,inline]}
\newcommand\atha{\todo[author={@Ha},inline]}
\lstdefinelanguage{michelson}{
  basicstyle=\fontsize{8}{9.6}\selectfont,
  morekeywords={parameter,storage,or,unit,mutez,pair,bool,address}, sensitive=false,
  morecomment=[l]{\#},
  morecomment=[\STACK]{/*}{*/},
  morestring=[b]",
}
\lstset{
  language=Caml,
  captionpos=b,
  aboveskip=-\smallskipamount,
  belowskip=-\smallskipamount,
  belowcaptionskip=0pt,
  basicstyle=\fontsize{8}{9.6}\selectfont,
  morekeywords={val}
}

\input{latex/agda-generated}
\usepackage{dsfont}
\usepackage{newunicodechar}
\newunicodechar{λ}{\ensuremath{\mathnormal\lambda}}
\newunicodechar{σ}{\ensuremath{\mathnormal\sigma}}
\newunicodechar{τ}{\ensuremath{\mathnormal\tau}}
\newunicodechar{π}{\ensuremath{\mathnormal\pi}}
\newunicodechar{ℕ}{\ensuremath{\mathbb{N}}}
\newunicodechar{∷}{\ensuremath{::}}
\newunicodechar{≡}{\ensuremath{\equiv}}
\newunicodechar{≅}{\ensuremath{\cong}}
\newunicodechar{∀}{\ensuremath{\forall}}
\newunicodechar{ᴸ}{\ensuremath{^L}}
\newunicodechar{ᴿ}{\ensuremath{^R}}
\newunicodechar{ʳ}{\ensuremath{^r}}
\newunicodechar{ⱽ}{\ensuremath{^V}}
\newunicodechar{⟧}{\ensuremath{\rrbracket}}
\newunicodechar{⟦}{\ensuremath{\llbracket}}
\newunicodechar{⊤}{\ensuremath{\top}}
\newunicodechar{⊥}{\ensuremath{\bot}}
\newunicodechar{₁}{\ensuremath{_1}}
\newunicodechar{₂}{\ensuremath{_2}}
\newunicodechar{₃}{\ensuremath{_3}}
\newunicodechar{₄}{\ensuremath{_4}}
\newunicodechar{₅}{\ensuremath{_5}}
\newunicodechar{₆}{\ensuremath{_6}}
\newunicodechar{₇}{\ensuremath{_7}}
\newunicodechar{₈}{\ensuremath{_8}}
\newunicodechar{₉}{\ensuremath{_9}}
\newunicodechar{∈}{\ensuremath{\in}}
\newunicodechar{₀}{\ensuremath{_0}}
\newunicodechar{′}{\ensuremath{'}}
\newunicodechar{ˢ}{\ensuremath{^S}}
\newunicodechar{ᴬ}{\ensuremath{^A}}
\newunicodechar{∘}{\ensuremath{\circ}}
\newunicodechar{𝟙}{\ensuremath{\mathds{1}}}  
\newunicodechar{𝟘}{\ensuremath{\mathds{O}}}
% \newunicodechar{𝟙}{\ensuremath{\mathbb{I}}}  
% \newunicodechar{𝟘}{\ensuremath{\mathbb{O}}}
\newunicodechar{ᴾ}{\ensuremath{^P}}
\newunicodechar{ᵀ}{\ensuremath{^T}}
\newunicodechar{⊎}{\ensuremath{\uplus}}
\newunicodechar{ι}{\ensuremath{\iota}}
\newunicodechar{⇐}{\ensuremath{\Leftarrow}}
\newunicodechar{⇒}{\ensuremath{\Rightarrow}}
%\newunicodechar{∎}{\ensuremath{\mathnormal\blacksquare}}
\newunicodechar{➙}{\ensuremath{\to^P}}
\newunicodechar{Δ}{\ensuremath{\Delta}}
\newunicodechar{∅}{\ensuremath{\emptyset}}
\newunicodechar{⁺}{\ensuremath{^+}}
\newunicodechar{𝕏}{\ensuremath{\mathbb{X}}}
%\newunicodechar{·}{\ensuremath{\cdot}} %seems to be defined already!
\newunicodechar{∙}{\ensuremath{\bullet}}
\newunicodechar{⁇}{\ensuremath{?}}
\newunicodechar{‼}{\ensuremath{!}}
\newunicodechar{⊕}{\ensuremath{\oplus}}
\newunicodechar{ℤ}{\ensuremath{\mathbb{Z}}}
\newunicodechar{μ}{\ensuremath{\mu}}
\newunicodechar{∃}{\ensuremath{\exists}}
\newunicodechar{⨟}{\ensuremath{\fatsemi}}
\newunicodechar{Σ}{\ensuremath{\Sigma}}
\newunicodechar{ᵣ}{\ensuremath{_r}}
\newunicodechar{ᵢ}{\ensuremath{_i}}
\newunicodechar{ₙ}{\ensuremath{_n}}
\newunicodechar{ₘ}{\ensuremath{_m}}
\newunicodechar{≢}{\ensuremath{\not\equiv}}
\newunicodechar{≟}{\ensuremath{\stackrel{{\tiny?}}{=}}}
\newunicodechar{≤}{\ensuremath{\le}}
\newunicodechar{ᵇ}{\ensuremath{^b}}
\newunicodechar{ˡ}{\ensuremath{^l}}
\newunicodechar{𝓣}{\ensuremath{\mathcal{T}}}
\newunicodechar{𝓔}{\ensuremath{\mathcal{E}}}
\newunicodechar{Γ}{\ensuremath{\Gamma}}
\newunicodechar{γ}{\ensuremath{\gamma}}
\newunicodechar{⊔}{\ensuremath{\sqcup}}
\newunicodechar{α}{\ensuremath{\alpha}}
\newunicodechar{η}{\ensuremath{\eta}}
\newunicodechar{ω}{\ensuremath{\omega}}
\newunicodechar{◁}{\ensuremath{\lhd}}
%PT: seems to be already defined
%\newcommand{\lambdabar}{{\mkern0.75mu\mathchar '26\mkern -9.75mu\lambda}}
\newunicodechar{ƛ}{\ensuremath{\lambdabar}}
\newunicodechar{Λ}{\ensuremath{\mathnormal\Lambda}}
\newunicodechar{ρ}{\ensuremath{\rho}}
\newunicodechar{𝓖}{\ensuremath{\mathcal{G}}}
\newunicodechar{ℓ}{\ensuremath{\ell}}
\newunicodechar{♯}{\ensuremath{\sharp}}
\newunicodechar{⇓}{\ensuremath{\Downarrow}}
\newunicodechar{𝓥}{\ensuremath{\mathcal{V}}}
\newunicodechar{∧}{\ensuremath{\wedge}}
\newunicodechar{ₛ}{\ensuremath{_s}}
\newunicodechar{χ}{\ensuremath{\chi}}
\newunicodechar{⊨}{\ensuremath{\models}}
\newunicodechar{⦂}{\ensuremath{\mathbf{:}}}
\newunicodechar{ς}{\ensuremath{\varsigma}}
\newunicodechar{𝓓}{\ensuremath{\mathcal{D}}}
\newunicodechar{∎}{\ensuremath{\square}}
\newunicodechar{■}{\ensuremath{\blacksquare}}
\newunicodechar{↠}{\ensuremath{\twoheadrightarrow}}
\newunicodechar{↪}{\ensuremath{\hookrightarrow}}
\newunicodechar{;}{\ensuremath{;}}
\newunicodechar{⊢}{\ensuremath{\vdash}}
\newunicodechar{∸}{\ensuremath{\dotminus}}
\newunicodechar{≥}{\ensuremath{\ge}}
\newunicodechar{β}{\ensuremath{\beta}}
\newunicodechar{Φ}{\ensuremath{\Phi}}
\newunicodechar{φ}{\ensuremath{\varphi}}
\newunicodechar{⨾}{\ensuremath{;}}
\makeatletter
\newcommand{\dotminus}{\mathbin{\text{\@dotminus}}}

\newcommand{\@dotminus}{%
  \ooalign{\hidewidth\raise1ex\hbox{.}\hidewidth\cr$\m@th-$\cr}%
}
\makeatother


\newcommand\repo{https://github.com/Tezos-Project-Uni-Freiburg/michelson-dynamic-logic}
\newcommand\listref[1]{Fig.~\ref{#1}}
\newcommand\secref[1]{Sec.~\ref{#1}}
\newcommand\chapref[1]{Sec.~\ref{#1}}
\newcommand\tabref[1]{Table~\ref{#1}}

\newcommand*\ACode[1]{\AgdaFontStyle{#1}}
\newcommand*\AField[1]{\AgdaField{#1}}
\newcommand*\ACon[1]{\AgdaInductiveConstructor{#1}}
\newcommand*\AKw[1]{\AgdaKeyword{#1}}
\newcommand*\AFun[1]{\AgdaFunction{#1}}
\newcommand*\ACom[1]{\AgdaComment{#1}}
\newcommand*\ADT[1]{\AgdaDatatype{#1}}

\newcommand\ABaseType{\AgdaDatatype{BaseType}}
\newcommand\AType{\AgdaDatatype{Type}}
\newcommand\ASet{\AgdaDatatype{Set}}
\newcommand\APassable{\AgdaDatatype{Passable}}
\newcommand\APushable{\AgdaDatatype{Pushable}}
\newcommand\AStorable{\AgdaDatatype{Storable}}
\newcommand\AProgram{\AgdaDatatype{Program}}
\newcommand{\Amutez}{\AgdaInductiveConstructor{`mutez}}
\newcommand{\Aaddr}{\AgdaInductiveConstructor{`addr}}
\newcommand{\Aend}{\AgdaInductiveConstructor{end}}
\newcommand{\AADDnn}{\AgdaInductiveConstructor{ADDnn}}
\newcommand{\AADDm}{\AgdaInductiveConstructor{ADDm}}
\newcommand{\AADD}{\AgdaFunction{ADD}}
\newcommand{\Aoperation}{\AgdaInductiveConstructor{operation}}
\newcommand{\AOperation}{\AgdaDatatype{Operation}}
\newcommand{\Acontract}{\AgdaInductiveConstructor{contract}}
\newcommand{\AtransferTokens}{\AgdaInductiveConstructor{transfer-tokens}}
\newcommand{\AInstructionPlus}{\AgdaFunction{Instruction$^+$}}
\newcommand{\Aty}{\AFun{ty}}
\newcommand{\AP}{\AFun{P}}

\newcommand\ASem[1]{\AgdaOperator{\AgdaFunction{⟦#1⟧}}}


%%% Local Variables:
%%% mode: latex
%%% TeX-master: "itp2024"
%%% End:


\bibliographystyle{plainurl}% the mandatory bibstyle

\title{A Dynamic Logic for Symbolic Execution for the Smart Contract Programming Language Michelson}

\titlerunning{Dynamic Logic for Symbolic Execution for Michelson} %TODO optional, please use if title is longer than one line

\author{Barnabas Arvay}{University of Freiburg, Germany}{barnabasarvay@gmail.com}{https://orcid.org/0009-0002-2720-7100}{}%TODO mandatory, please use full name; only 1 author per \author macro; first two parameters are mandatory, other parameters can be empty. Please provide at least the name of the affiliation and the country. The full address is optional. Use additional curly braces to indicate the correct name splitting when the last name consists of multiple name parts.

\author{Thi Thu Ha Doan}{University of Freiburg,
  Germany}{doanha@informatik.uni-freiburg.de}{https://orcid.org/0000−0001−7524−4497}{Supported
  by the Tezos Foundation, grant COOC}

\author{Peter Thiemann}{University of Freiburg, Germany}{thiemann@informatik.uni-freiburg.de}{https://orcid.org/0000−0002−9000−1239}{}

\authorrunning{B. Arvay, T.\,T.\,H. Doan, P. Thiemann} %TODO mandatory. First: Use abbreviated first/middle names. Second (only in severe cases): Use first author plus 'et al.'

\Copyright{Barnabas Arvay and Thi Thu Ha Doan and Peter Thiemann} %TODO mandatory, please use full first names. LIPIcs license is "CC-BY";  http://creativecommons.org/licenses/by/3.0/

\begin{CCSXML}
<ccs2012>
<concept>
<concept_id>10011007.10010940.10010992.10010998.10011000</concept_id>
<concept_desc>Software and its engineering~Automated static analysis</concept_desc>
<concept_significance>500</concept_significance>
</concept>
</ccs2012>
\end{CCSXML}

\ccsdesc[500]{Software and its engineering~Automated static analysis}

\keywords{Smart Contract, Blockchain, Formal Verification, Symbolic Execution} %TODO mandatory; please add comma-separated list of keywords

%\relatedversion{} %optional, e.g. full version hosted on arXiv, HAL, or other respository/website
%\relatedversiondetails[linktext={opt. text shown instead of the URL}, cite=DBLP:books/mk/GrayR93]{Classification (e.g. Full Version, Extended Version, Previous Version}{URL to related version} %linktext and cite are optional

% \supplement{https://freidok.uni-freiburg.de/data/255176}%optional, e.g. related research data, source code, ... hosted on a repository like zenodo, figshare, GitHub, ...
%\supplementdetails[linktext={opt. text shown instead of the URL}, cite=DBLP:books/mk/GrayR93, subcategory={Description, Subcategory}, swhid={Software Heritage Identifier}]{General Classification (e.g. Software, Dataset, Model, ...)}{URL to related version} %linktext, cite, and subcategory are optional
\supplementdetails[subcategory={Source Code}]{Software}{https://freidok.uni-freiburg.de/data/255176}

%\funding{(Optional) general funding statement \dots}%optional, to capture a funding statement, which applies to all authors. Please enter author specific funding statements as fifth argument of the \author macro.

%\acknowledgements{I want to thank \dots}%optional

\nolinenumbers %uncomment to disable line numbering

%Editor-only macros:: begin (do not touch as author)%%%%%%%%%%%%%%%%%%%%%%%%%%%%%%%%%%
\EventEditors{Jonathan Aldrich and Guido Salvaneschi}
\EventNoEds{2}
\EventLongTitle{38th European Conference on Object-Oriented Programming (ECOOP 2024)}
\EventShortTitle{ECOOP 2024}
\EventAcronym{ECOOP}
\EventYear{2024}
\EventDate{September 16--20, 2024}
\EventLocation{Vienna, Austria}
\EventLogo{}
\SeriesVolume{313}
\ArticleNo{46}
%%%%%%%%%%%%%%%%%%%%%%%%%%%%%%%%%%%%%%%%%%%%%%%%%%%%%%

\begin{document}

\maketitle

%TODO mandatory: add short abstract of the document
\begin{abstract}
Verification of smart contracts is an important topic in the context
of blockchain technology. We study an approach to verification that is
based on symbolic execution. 

As a formal basis for symbolic execution, we design a dynamic logic
for Michelson, the smart contract language of the Tezos blockchain,
and prove its soundness in the proof assistant Agda. Towards the
soundness proof we formalize the concrete semantics as well as its
symbolic counterpart in a unified setting. The logic encompasses
single contract runs as well as inter-contract runs chained in a single transaction. 
\end{abstract}

\section{Introduction}
\label{sec:introduction}

Blockchain technology and smart contracts provide decentralized and
immutable systems for secure transactions and automated agreements.
Smart contracts have been targets of spectacular and costly attacks as
contracts are immutable and their source code is publicly available on
the blockchain. 
Hence, it is vital as well as challenging to ensure the correctness of smart contracts
before their deployment. Formal methods and various verification
techniques have been proposed to address this challenge.

The Tezos blockchain \cite{tezos-whitepaper} and its smart contract
language Michelson have been designed from ground up with verification
in mind. Several frameworks have been developed based on, e.g.,
interactive theorem proving \cite{micho}, refinement typing
\cite{helmholtz}, and automated theorem proving \cite{WHYtool}.  We
are interested in automated verification of Michelson programs, which
rules out interactive approaches.  Symbolic
execution~\cite{DBLP:journals/cacm/King76,DBLP:conf/ifip/Burstall74}
is one of the standard approaches to automatically obtain verification
conditions like weakest preconditions for failures as well as normal
termination from a program. Next, an SMT-solver discharges these
verification conditions. There is a wide range of approaches that
apply symbolic execution combined with SMT-solving to smart contracts,
mostly for the Ethereum blockchain (see~\autoref{sec:related-work}).

While there are many approaches to symbolic execution
\cite{DBLP:conf/osdi/CadarDE08,DBLP:conf/icse/CsallnerTS08,Pasareanu2020},
we choose one based on dynamic logic.
Dynamic logic (DL) \cite{DL} is a modal logic for reasoning about
programs. Its signature features are modalities for program
execution. These modalities enable the expression of assertions about
program behavior as logical formulas. For instance, the formula $[p]
\Psi$ states partial correctness: if program $p$ terminates, then $\Psi$ is true. That
is, a Hoare triple $\{\Phi\}~p~\{ \Psi\}$ can be
encoded by $\Phi \to [p]\Psi$. DL also provides a modality
$\langle p\rangle$ for total correctness, but we do not consider it in this work.

Dynamic logic comes with proof rules for the modality derived from the
structure of $p$. For example, if $p;q$ stands for sequential 
execution of $p$ and $q$, then the proof rule $[p;q]\Psi
\leftrightarrow [p][q]\Psi$ states that execution of $p$ enables
execution of $q$ such that $\Psi$ holds in the end. Similarly, the
rule $[\varepsilon]\Psi \leftrightarrow \Psi$ states that the empty
program $\varepsilon$ does not modify the validity of $\Psi$.

\begin{comment}
  Such formulas are verified by successively reducing the program it
  contains until one is left with a purely first order formula that
  does not contain programs anymore, which can then be verified using
  the calculus of the first order logic that the DL is based on:
  \begin{align*}
    \langle i ; i' ; &\cdots ; i^n \rangle \phi
    \\ \leftrightarrow	\qquad		\langle     i' ; &\cdots ; i^n \rangle \phi'
    \\							 &\quad\vdots
    \\ \leftrightarrow	\qquad	\langle \rangle \phi^n	 &\leftrightarrow \phi^n
    \\							 &\quad\vdots
  \end{align*}

  % \section*{Scope and structure}
  This thesis focuses on these symbolic execution rules of the
  calculus and their soundness proof.  Firstly, we choose a
  representative subset of Michelson and give a reference
  implementation of it.  Then we define the symbolic execution rules
  for that subset and prove their soundness with respect to the
  reference implementation.
\end{comment}



In the past, dynamic
logic has been used successfully for as a basis for symbolic execution
in the context of the verification of Java programs \cite{KeY3}, as it
is particularly well suited to keep track of a changing environment (i.e., mutable
objects on Java's heap). 
We design a DL to model Michelson execution because we want to reason about
transactions that span several contract runs. In Michelson
terminology, these transactions are called \emph{chained contract
  executions}, where an externally started contract run
initiates further internal contract runs.
Our DL design models the relevant parts of the
blockchain run-time system on top of the purely functional execution of
Michelson programs. On the level of the run-time system contracts are
very similar to objects: they are identified by an address and they
come with mutable attributes (state and balance). 


The DL treatment of the functional part of Michelson is quite
intuitive: programs are sequences of Michelson instructions, we model
the execution state of a Michelson program by a formula of the form $\Phi \to [p]\Psi$, and the proof
rules for $[i;p]\Psi$ (where $i$ is a single instruction) define the
semantics of symbolic execution.

Gas is an important aspect of computation on the blockchain. The
initial caller of a contract has to pay for executing the transaction
(consisting of one or more chained contract runs) in terms of gas.
A transaction that
runs out of gas is rolled back by the run-time system of the
blockchain as if it never happened. As Michelson does not
suffer from reentrancy problems (cf.\ Section~\ref{sec:michelson}),
gas does not affect reasoning about the functional correctness of
(chained) contract execution. For that reason,  our DL design does not
account for gas. 

It is the sole goal of this paper to provide a \textbf{machine
  verified specification} of symbolic execution for Michelson, rather than an
efficient or otherwise realistic implementation. For that reason, the
paper does not cover all instructions, but rather a carefully chosen
representative subset. This is in contrast to related work
\cite{micho,helmholtz,WHYtool} that describes \textbf{actual
verification tools}. To be useful for a wide range of programs, such a
tool must support as many Michelson instructions as
possible\footnote{Keeping up with the rapid evolution of the language is
  challenging for those tools. As of this writing, most of them support the instruction
  set available in late 2022.}, it must
be reasonably efficient, and it must deal with loops and
nontermination in an appropriate way. None of these issues are
concerns for our specification.

\subsection*{Contributions}
\label{sec:contributions}

\begin{enumerate}
\item We select a representative subset of Michelson
  instructions so as to provide proof templates for all 
  current and future instructions that work similarly.
% \item We define a core calculus for Michelson that covers
%   a representative subset of instructions.
\item We provide a parameterized semantics definition with instances
  for the concrete semantics as well as for an abstract semantics,
  which implements the dynamic logic for Michelson. 
\item We prove the soundness of this logic first for single programs,
  and then for several programs chained in a transaction.
\end{enumerate}

The Agda implementation of the contributions is
available.\footnote{\url{https://freidok.uni-freiburg.de/data/255176},
development version  \url{\repo}.}


\subsection*{Overview}
\label{sec:overview}

Section~\ref{sec:michelson} gives an overview of Michelson, introduces
its type system and our intrinsically typed representation of the language.
Section~\ref{sec:refImpl} defines the execution model of Michelson,
first for single contracts, and then for the chained execution of
several contracts that call each other.
Section~\ref{sec:DL} introduces dynamic logic and its
symbolic execution rules, again first for single execution, and then
for chained execution.
Section~\ref{sec:semantics-soundness} explains the major components of
the soundness proof of the dynamic logic.
Section~\ref{sec:related-work} discusses related work and conclusions.

The paper contains many excerpts from the live, type checked
definitions and proofs in Agda. In particular, all major definitions
and statements of theorems are shown in Agda notation to ensure
consistency of the paper with the machine-checked proofs.



%%% Local Variables:
%%% mode: latex
%%% TeX-master: "itp2024"
%%% End:

\section{Michelson}
\label{sec:michelson}
\label{sec:Mtype}


Michelson~\cite{Mref,devres} is the native language for smart contracts on the Tezos blockchain.
It is a stack-based, low level, strongly-typed functional programming
language. That is, all computation is driven by transforming an input
stack into an output stack. There are no mutable data structures;
blockchain transactions are handled outside of Michelson.
All contracts are statically typed to avoid type errors during runtime.
Michelson is simply-typed as there is neither universal nor
existential quantification.

Each Michelson instruction converts a given input stack into an output stack
where some of its values have been changed, added or removed.
For example, the \verb=ADD= instruction will accept any stack
whose two top most elements are numeric values,
and return a stack where these two values have been replaced by their sum
and the remaining stack is unchanged:
\begin{align*}
	\text{ADD} :: 15 :: 27 :: \text{\emph{remainingStack}}
	\mapsto	           42 :: \text{\emph{remainingStack}}
\label{equ:ADD}
\end{align*}

\subsection{Types}
\label{sec:michelson-types}

Michelson supports the usual data types like numbers, pairs, and lists as well as
some blockchain-specific types for tokens and contracts. 
Figure~\ref{fig:Type} contains Agda definitions for selected subset of
Michelson types {\AType}. As some base types can be treated alike later on, we
represent them with a separate type {\ABaseType}.

\begin{figure}[tp]
  \begin{subfigure}{0.48\textwidth}
    \noindent
    \TypesType
    \TypesPatterns
    \caption{Syntax}
    \label{fig:Type}
  \end{subfigure}
  \begin{subfigure}{0.48\textwidth}
  \TypesSemantics
  \caption{Semantics}
  \label{fig:Type-Semantics}
\end{subfigure}
\caption{Michelson Types}
\label{Type}
\end{figure}

Most types' names are self explanatory. The base type {\Amutez} stands
for tokens, {\Aaddr} stands for blockchain addresses in  Tezos. We use
some obvious shorthand patterns for base types.
The type {\Aoperation} consists of blockchain operations that can
be emitted during contract execution. This mechanism implements token
transfers from the current contract to other accounts or contracts.
The type {\Acontract~\AP} represents such a contract
which accepts a parameter of type {\Aty} represented by {\APassable~\Aty}.
The type predicate {\TypesPassable} is declared mutually recursive with {\AType}
and characterizes types that can be passed as parameters to contracts.

The semantics of types is defined by a mapping to Agda types. Most
Michelson types have  obvious Agda counterparts, except {\Aaddr},
{\Acontract}, and {\Aoperation}.  Addresses and contracts are both
represented by natural numbers. We only define one alternative of the
{\AOperation} datatype: {$\AtransferTokens~v~m~c$}, which models a
token transfer to contract $c$ while passing the parameter value $v$
and tokens $m$.

\subsection{Programs and Instructions}
\label{sec:michelson-programs}


Michelson programs are intrinsically typed and represented accordingly in
Agda by a datatype {\AProgram} indexed by the types on the input stack
and the types on the output stack. We assume that \TypesStack.
\SyntaxProgram

Instructions are indexed in the same way:
If  instruction \verb/inst/ maps an input stack $Si$ to an output stack $So$
and \verb/prg/ maps that output stack $So$ to the final stack $Se$,
then \verb/inst ; prg/ is a program that maps $Si$ to $Se$.
The empty program {\Aend} does not transform the stack.

\begin{figure}[tp]
  \SyntaxInstruction  
  \caption{Instructions of Core Michelson}
  \label{fig:core-michelson-instructions}
\end{figure}
We discuss a representative subset of Michelson instructions shown in Figure~\ref{fig:core-michelson-instructions}.
The definition of {\AInstructionPlus} implements the pattern that most
instructions only transform the top elements of the input stack and are parametric in the rest. 

The first group of instructions operates on a fixed number of values on
the stack and pushes the result. All arithmetic operations belong to
this group and we just give two examples, {\AADDnn}  and {\AADDm},
which perform addition of natural numbers and tokens, respectively.
The original Michelson language overloads arithmetic operators. As
overloading is not supported by Agda, we supply separate
instructions. We come back to this issue at the end of this section. 

\ACon{CAR}, \ACon{CDR}, and \ACon{PAIR} are the standard operations on
pairs. \ACon{NONE} and \ACon{SOME} are the constructors for the
\ACon{option} datatype, and \ACon{NIL} and \ACon{CONS} construct
lists. The constructors for ``empty'' containers, \ACon{NONE} and
\ACon{NIL} are indexed by the element type, otherwise that type can be
inferred from the context.

The last instruction in this group is \ACon{TRANSFER-TOKENS}. Despite
the name, this instruction does \textbf{not} directly transfer tokens
to another account. It rather constructs a value
{$\AtransferTokens~v~m~c$} of type {\Aoperation} from its arguments.

The instructions in the next group differ in that they push zero or more
values on the output stack. \ACon{DROP} pops the stack, \ACon{DUP} duplicates the top of the
stack, \ACon{SWAP} swaps the top entries, and \ACon{UNPAIR} eliminates
a pair and pushes its contents. \ACon{UNPAIR} is a convenience
instruction as it is equivalent to the instruction sequence
\ACon{DUP}; \ACon{CDR}; \ACon{SWAP}; \ACon{CAR}. 

The next group contains instructions that are blockchain
specific. \ACon{AMOUNT} returns the tokens that were transferred with
the currently running contract invocation and \ACon{BALANCE}  returns
the tokens currently owned by it. The \ACon{CONTRACT} instruction is
indexed by a type $t$ that must be {\APassable}. It takes an address
and check on the blockchain whether this address is associated to a
contract that accepts arguments of type $t$. The result is
communicated as an \ACon{option} types. That is, the {\Acontract} type
carries a verified address.

The \ACon{PUSH} instruction pushes a value of type $t$ on the stack
assuming that value is {\APushable} (another predicate an types). The
value is encoded by a type-indexed datatype \AgdaDatatype{Data} for
pushable values. We elide its obvious definition.

The last group of instructions showcases control structures and an
instruction that operates in a non-uniform way on the stack. The
instruction \ACon{IF-NONE} eliminates a value of option type from the
top of the stack. Its parameters are programs that implement the branches for case \ACon{None}
and \ACon{Some}. The latter takes the value wrapped in the \ACon{Some}
constructor as an argument on top of the stack.

The instruction \ACon{ITER} runs a sub-program on every element of its
argument list. The instruction $\ACon{DIP}~n$ runs a sub-program at
depth $n$ on the input stack, that is, it skips over the first $n$
elements of the stack, runs the sub-program, and reattaches those
elements. The extra machinery in the implicit argument of the
instruction makes sure that there are at least $n$ elements on the
stack. This mechanism is called reflection in the  PLFA textbook
\cite{plfa}.

Earlier, we remarked that Agda does not allow overloading of
constructors in the same datatype. However, we can use reflection to
define a ``smart constructor'' that almost suits the purpose.
\SyntaxOverloading

The definition exploits the fact that the input stack of an
instruction is always known in a Michelson program. The same fact also
enables overloading in Michelson's implementation to work.
The function \AFun{overADD} specifies the resolution of overloading
for the \AFun{ADD} instruction. If the argument types are both
\ACon{nat}, then the result type is \ACon{nat} and the chosen instruction is
\ACon{ADDnn}; and so on.\footnote{Full Michelson has ten different
  overloadings of \AFun{ADD}.} If no  overloading is known for a combination
of arguments, the function return \ACon{nothing}.
The smart constructor \AFun{ADD} takes a proof that the overloading is
defined on a given pair of input types. Then it extracts the selected
instruction from the overloading.

Compared to ``real'' Michelson, the smart constructor requires an
extra argument to work:
\SyntaxOverloadingExample

\subsection{Blockchain interface}
\label{sec:blockchain-interface}

A contract on the Tezos blockchain is indexed by a parameter type $p$
and a store type $s$. The type $p$ must be {\APassable} and the type
$s$ must be {\AStorable}. Moreover, each contract comes with a current
balance of tokens and a store of type $s$. The implementation of the
contract is a program that maps a $\ACon{pair}~p~s$ to a
$\ACon{pair}~(\ACon{list}~\Aoperation)~s$, that is, it consumes the
parameter paired with the current store and produces a list of operations (e.g., to invoke further
contracts) paired with the updated store. The program itself is pure;
any side effects, i.e., store update and contract calls, are managed
by the blockchain run-time.
\ConcreteContract

The $Mode$ argument abstracts over the semantics of types. Its type
has two components, one $\mathcal{M}$ for the semantics and the other $\mathcal{F}$ will be
explained in the context of the abstract semantics in Section~\ref{sec:abstract-semantics}.
\ConcreteMODE
Its instantiation for the concrete semantics installs the 
standard semantics from Section~\ref{sec:michelson-types}.
\ConcreteCMode

With this definition, the contract store of the blockchain is just a partial mapping from addresses
to contracts.
\ConcreteBlockchain

To start executing a contract, we make a blockchain transaction to its
address, i.e., we ask the blockchain runtime to transfer tokens to it along with its parameter.
Once a contract has terminated, the runtime updates the
stored value and processes the list of operations.

On the Tezos blockchain a normal account with deposit $init$ is a
contract with a unit parameter, unit store, and a trivial program that issues no operations.
\ConcreteAccount


\begin{comment}
\subsection{Michelson instructions}\label{sec:instructions}

The way Michelson is represented in this thesis is heavily tailored towards
the soundness proof and making it as easy and concise as possible in Agda.

The first consequence of this paradigm is that rather than defining the instructions
and subsequently associating them with their respective typing rules,
we use an approach of \emph{intrinsically-typed} instructions similar to
intrinsically-typed terms as described in~\cite{plfa}:
Our Agda definition for instructions is indexed by the input and output stack
that the instruction can operate on, and it is not possible to give an instruction
without also specifying its in- and output stacks:
\mint{agda}|data Instruction : Stack → Stack → Set|
Note that \verb/Stack/ in the Agda code is a list of \verb/Type/'s representing
the type of a given stack, corresponding to the meaning of ``stack'' throughout this chapter.

Michelson programs are represented as a concatenation of instructions for matching stacks:
%% listing ruler max width ------------------------------------------------|?X
\begin{figure}[tp]
\begin{minted}{agda}
data Program : Stack → Stack → Set where
  end  : ∀ {S} → Program S S
  _;_  : ∀ {Si So Se}
       → Instruction  Si   So
       → Program      So   Se
       → Program      Si   Se
\end{minted}
\caption{Programs are lists of instructions}
\label{Program}
\end{figure}
If \verb/inst/ maps an input stack \verb/Si/ to an output stack \verb/So/
and \verb/prg/ maps that output stack \verb/So/ to the ``end'' stack \verb/Se/,
then \verb/inst ; prg/ is a program that maps \verb/Si/ to \verb/Se/.
This way Agdas typechecker will enforce that only well typed Michelson programs can be entered.

Another consequence is that instructions are further subdivided into different categories.
The main categories are \emph{functional} and \emph{control flow} instructions.
The former are all instructions that represent a function that takes the top of the
current stack as arguments and maps it to some result, while leaving the remaining stack unchanged.
All instructions from the example program in \listref{simple-example} are from that category.
They can be executed in a single step and will be further subdivided into several categories.

Because they do not change the remaining stack and only require the top of the stack to match
their argument types, their subcategories will be given as function types indexed by their
argument and result types, ignoring the remaining unchanged stack.
When defining \verb/Instruction/,
these will be mapped to instructions that work on any remaining substack.

The biggest subcategory of functions combines all functions
whose output only depends on the input arguments from the top of the stack
and no further knowlegde of the execution environment is necessary for their execution.

This group is further subdivided according to their role during symbolic execution:
Multidimensional functions, as well as the \verb/PUSH/ instruction, require special
treatment when they are symbolically executed, which is explained in \secref{sec:calculus}.
All other onedimensional functions serve an additional purpose:
They will later be reused in the Dynamic Logic to define most of the terms for the logic, which,
unlike all other functional instructions, also gives them a universal symbolic execution scheme.

The function types of these instructions is given in \listref{func-type}.
We employ some Agda pattern synonyms for short lists \verb|[ a ] [ a / b ]|
of one or two elements.
\verb/⟦ ty ⟧/ is the set of values of type \verb/ty/.
Notice, that there are two definitions of \verb/ADD/ for naturals and for mutez tokens.
This is necessary for our intrinsic typing scheme, but since the typing rule for each instruction
of a contract is unambiguous, our implementation does not loose expressiveness by this restriction.
% out implementation of Michelson does not loose expressiveness

%% listing ruler max width ------------------------------------------------|?X
\begin{figure}[tp]
\begin{minted}{agda}
data 1-func : Stack → Type → Set where
  ADDnn  :                1-func  [ base   nat / base   nat ]  (base   nat)
  ADDm   :                1-func  [ base mutez / base mutez ]  (base mutez)
  CAR    :  ∀ {t1 t2}  →  1-func               [ pair t1 t2 ]           t1
  CDR    :  ∀ {t1 t2}  →  1-func               [ pair t1 t2 ]           t2
  PAIR   :  ∀ {t1 t2}  →  1-func                  [ t1 / t2 ]  (pair t1 t2)
  NIL    :  ∀  ty      →  1-func                           []  (list    ty)
  NONE   :  ∀  ty      →  1-func                           []  (option  ty)
  SOME   :  ∀ {ty}     →  1-func                       [ ty ]  (option  ty)
  CONS   :  ∀ {ty}     →  1-func             [ ty / list ty ]  (list    ty)
  TRANSFER-TOKENS : ∀ {ty pt}
            →  1-func  [ ty / base mutez / contract {ty} pt ]          ops

data m-func : Stack → Stack × Type → Set where
  UNPAIR  :  ∀ {t1 t2}  →  m-func  [ pair t1 t2 ]   ([ t1 ] , t2)
  SWAP    :  ∀ {t1 t2}  →  m-func     [ t1 / t2 ]   ([ t2 ] , t1)
  DUP     :  ∀ {ty}     →  m-func          [ ty ]   ([ ty ] , ty)

data func-type : Stack → Stack × Type → Set where
  D1   : ∀ {res  args} → 1-func args res      →  func-type args ([] , res)
  Dm   : ∀ {args ress} → m-func args ress     →  func-type args       ress
  PUSH : ∀ {ty}        → Pushable ty → ⟦ ty ⟧ →  func-type []   ([] ,  ty)
\end{minted}
\caption{Function types}
\label{func-type}
\end{figure}

Besides those, \listref{env-func} defines functions that provide information about the
current execution environment,
like the amount of tokens that was transfered when starting the execution,
or the contract stored at a given address.
Their typing scheme does not differ from \verb/func-type/ functions,
but since their execution requires this information,
they have to be executed differently and will be given their own category.

\begin{figure}[tp]
\begin{minted}{agda}
data env-func : Stack → Type → Set where
  AMOUNT    :             env-func             []               (base mutez)
  BALANCE   :             env-func             []               (base mutez)
  CONTRACT  :  ∀ {ty} P → env-func  [ base addr ] (option (contract {ty} P))
\end{minted}
\caption{Functions for blockchain operations}
\label{env-func}
\end{figure}

Control flow instructions are those that take subprogram as arguments.
Their execution is defined in big step semantics,
and since their subprograms can be arbitrarily complex,
their execution requires additional features, which are explained in \chapref{chap:refImpl}.

Their typing rules are defined in the \verb/Instruction/ datatype in \listref{Instruction},
together with the functional instructions.
The different function types are mapped to their corresponding instructions
that work for any possible stack where the top matches the function argument types.
The Agda pattern \verb|[ x // xs ]| stands for the normal list constructor \verb/x ∷ xs/.
Since \verb/results/ of \verb/func-type/ has the type \verb/Stack × Type/,
it must be mapped to a list before it is concatenated with the substack.
\verb/DROP/ is the only 0-dimensional function and therefore provided with its own constructor.

%% listing ruler max width ------------------------------------------------|?X
\begin{figure}[tp]
\begin{minted}{agda}
data Instruction where
  enf       : ∀ {args result S}
            → env-func args result
            → Instruction  (       args ++ S )       [ result // S ]
  fct       : ∀ {args results S}
            → func-type args results
            → Instruction  (       args ++ S )  ([× results ] ++ S)
  DROP      : ∀ {ty S}
            → Instruction  [         ty // S ]                   S
  ITER      : ∀ {ty S}
            → Program      [         ty // S ]                   S
            → Instruction  [    list ty // S ]                   S
  DIP       : ∀ {S Se} n → {T (n ≤ᵇ length S)}
            → Program              (drop n S)                    Se
            → Instruction                  S        (take n S ++ Se)
  IF-NONE   : ∀ {ty S Se}
            → Program                      S                     Se
            → Program      [         ty // S ]                   Se
            → Instruction  [  option ty // S ]                   Se
\end{minted}
\caption{Intrinsically typed Michelson instructions}
\label{Instruction}
\end{figure}

\listref{simple-example-our} shows the example program from \listref{simple-example}
programmed in our Michelson representation.
\begin{figure}[tp]
\begin{minted}{agda}
example : Program [ pair (base nat) (base nat) ] [ pair (list ops) (base nat) ]
example = fct (Dm UNPAIR) ;
          fct (D1 ADDnn) ;
          fct (D1 (NIL ops)) ;
          fct (D1 PAIR) ; end
\end{minted}
\caption{Example program in our Michelson representation}
\label{simple-example-our}
\end{figure}

\end{comment}
\begin{comment}
The biggest and most importand subcategory of functional instructions is that of
those instructions whose typing rule 
These are all onedimensional functions whose execution only requires knowledge
of the current stack values that serve as the arguments of these functions,
like \verb/ADD/, \verb/PAIR/ or \verb/NIL ty/ from the example above.
However there is one exception: \verb/PUSH/ technically matches these requirenments,
but its symbolic execution will require some special treatment as we will see later.
Therefore it is given its own category.
Together with 

They will be used to represent certain terms of their result type
We will give these subcategories and they motivation in the same order they are defined in the
\model{02-Functions-Interpretations} module before explaining the \verb/Instruction/ datatype.

The biggest subcategory is that of onedimensional functions whose implementation
only requires the knowledge of 

% Functions to retrieve information from the execution environment,
% will require these additional informations besides the current state of the stack.
Control flow instructions are those that take subprogram as arguments.
Their execution is defined in big step semantics,
and since these subprograms can be arbitarily complex, their execution is a bit involved
and will be discussed in \chapref{chap:refImpl}.
% Their typing rules however are straight forward and can be easily implemented.

The first subcategory of functional instructions contains instructions 
to retrieve blockchain related information,
like the amount of tokens that was transfered when starting the current execution,
or the contract stored at a given address.
Their typing rules don't differ from other functional instructions,
but will be defined separately since their execution requires these informations
from the current execution environment.
All other functional instructions operate on the current stack values alone
and are further divided:
Onedimensional functions have a special role since they will later be reused in the
Dynamic Logic to define most of its terms, which will also make their symbolic execution
straight forward.
Multidimensional functions (like \verb/UNPAIR/ in the example above) need to be treated
individually during symbolic execution.
\verb/PUSH/ is technically a onedimensional function,
but its symbolic execution requires a special treatment as we will see later,
so it is given its own cathegory as well.

All functional instructions only require the top of the input stack to match the
required argument types and leave the rest of the stack unchanged.
This makes it convenient to define each functional subcategory by their argument and result types
and extend that typing when constructing instructions from them.

are represented separately since their execution
are given its own datatype \verb/env-func/ in the modell.
All other functional instructions operate only on the current stack values


into functions that only require the current
stack values to calculate its results, and functions
whose function operates solely on the values from the current stack
and those that 


\subsection{Design rational}\label{sec:desigr}
The guiding idea for out code design was reusability of typing rules
for the reference implementation in the dynamic logic
to make the soundness proofs as concise and easy as possible.

While the reference implementation will work on a stack of concrete values that are
provided for and manipulated during execution,
the DL will operate on a stack of variables for such values that may or may not be
accompanied by formulas that assign values to them.

So the Michelson instructions are defined by their typing rules,
and during concrete execution those typing rules will be matched to their corresponing
implementation rules (semantic rules),
while during symbolic execution a new variable will be introduced and a formula added
that describes its value by using its typing rule.

So for the \verb=ADD= instruction, the typing rule would be
\[			\text{nat} :: \text{nat} :: \text{\emph{remStack}}
	\Longrightarrow		      \text{nat} :: \text{\emph{remStack}}
\]
which will be mapped to its natural implementation $+ : \bN \to \bN \to \bN$ for concrete execution
as well as to a formula that describes a variable of type $\bN$ as being $\text{ADD} x y$
for two variables $x$ and $y$ of type $\bN$.

Notice that in Michelson some instructions, like \verb=ADD=, are not monomorphic.
However (due to Michelson being \emph{strongly typed}),
because the input and output stacks for smart contracts \emph{are} monomorphic,
the typing rule for every instruction within a given smart contract is fixed.
For that reason, we chose to give a monomorphic reference implementation
instead of
% which eliminates the teadious work of 
matching each instruction with the
typing rule that would be applied for a given contract,
since this reusability scheme applies to the typing rules anyways
and we don't loose generality by giving several add-instructions.
Any given Michelson contract can be compiled to one in our reference implementation
in linear time of its size due to Michelsons strict typing rules.

For \verb=ADD=, we define it for the two typing rules
for those Michelson types we chose to implement, with is \verb=ADDnn= for naturals
and \verb=ADDm= for adding to \verb=mutez= values.

Also, not all instructions represent unary functions \todo{is \emph{unary} correct here?}
like \verb=ADD= does, but only those can be reused for formulas.
So the implemented Michelson instructions are given in certain subcategories
according to their properties.

Of course we want to express in formulas that a given variable has some constant value,
but constant values should be restricted to variables of primitive types.
Constant values for complex types will be represented
with formulas of their according introduction instructions
like \verb=PAIR= for pairs or \verb=CONS= and \verb=NIL= for lists
and more formulas \draft{that fix the variables used there}.
This make reasoning about complex types easier, if not possible in the first place:
In an earlier model where this restriction was not present,
it turned out near impossible to proof that iterating over a list of 3 elements would terminate
without giving concrete values for those 3 elements.
But formulating such conclusions in general without giving the exact values
should be acchievable with a DL.

\subsection{Typing System}\label{sec:typing}

\subsubsection{Types}\label{subsec:types}

The Agda module \todo{link to github} \verb=01-Types= defines a data type \verb=BaseType=
to represent the primitive data types that have been implemented:
\begin{figure}[tp]
\begin{minted}{agda}
data BaseType : Set where
  unit  : BaseType
  nat   : BaseType
  mutez : BaseType
  addr  : BaseType
\end{minted}
\caption{Basic Types}
\label{BaseType}
\end{figure}

A Michelson type in our model is either a \verb=BaseType= or a complex type,
which can be a higher order type like \verb=pair=, \verb=list= or \verb=option=
or blockchain specific types like an \verb=operation= (abbreviated to \verb=ops=)
or a \verb=contract=:
\begin{figure}[tp]
\begin{minted}{agda}
data Type where
  ops          :               Type
  base         : BaseType    → Type
  pair         : Type → Type → Type
  list         : Type        → Type
  option       : Type        → Type
  contract     : ∀ {ty} → Passable ty → Type
\end{minted}
\caption{Michelson Types}
\label{Type}
\end{figure}

For correct implementation of the typing restrictions the 3 predicates
\verb=Pushable=, \verb=Passable= and \verb=Storable= are defined.
There are a lot more restrictions present in Michelson, but only these are needed
for the types implemented here.

After defining the \verb=DecidableEquality= operator on \verb=Types=,
a \verb=Stack= is defined to be a list of \verb=Types=.
To more easily differentiate between the types and values of a Michelson stack
as well as a Stack of DL variables that represent the program stack in the DL,
we will use these following terms:
\begin{description}
	\item[Stack]
		unless stated explicitly, \textbf{Stack} will always referre
		to a Stack of Michelson types (rather than a stack of values
		as in most documents regarding Michelson)
	\item[Interpretation]
		will mean a stack of values, corresponding to the \verb=Stack= indexed
		data type \verb=Int= in the module \verb=02-Functions-Interpretations=
	\item[Matching]
		lastly will mean a stack of DL variables representing a program stack
		during symbolic execution. It corresponds to the \verb=Match= data type
		defined in \verb=11-abstract-representation-and-weakening=,
		which is also indexed by a \verb=Stack=
\end{description}
The Agda constructors for the latter two work exactly like Agda List constructors
and the standard list operations \verb=take=, \verb=drop= and \emph{concatenation}
are implemented for those the same way as for lists,
together with operators to retrieve the top or bottom of an Interpretation or Matching
if they are indexed by a concatenation of two Stacks.

\subsubsection{Functions}\label{subsec:functions}

The module \verb=02-Functions-Interpretations= first defines the different kind of functions
that will be used:
\begin{itemize}
	\item	one-dimensional functions are the most common and those that will be
		reused for formulas. \\
		they are parameterized by the \verb=Stack= of required input types
		and the \emph{single} resulting output type (see \listref{1-func}).
	\item	there are some multidimensional functions (\listref{m-func}).
		% (\listref{m-func} \ldots making the second parameter a product of
		% \verb=Stack= and \verb=Type= enables ).
		The reference implementation won't make a difference between these
		(that's also why the second parameter of \verb=m-func= must be
		the product of \verb=Stack= and \verb=Type=, although only a \verb=Stack=
		should suffice)
		but for the DL implementation special cases are needed here.
	\item	the functions so far work on the current stack alone,
		but instructions for blockchain operations require informations
		about the current blockchain environment they are executed in.
		They don't differ from \verb=1-func= regarding typing
		but have to be treated differently during execution later on (listref{env-func}).
	\item	\emph{there is a 2-dimensional blockchain operation which was'n implemented
		in this}%  model and will require some extensions or reworking of it}
\end{itemize}

All functions that work on stack elements alone are combined in the \verb=func-type=
which also includes the \verb=PUSH= instruction (\listref{func-type}).
It's a onedimensional function regarding its typing rule, but Michelson allows to push
compounded types that can't necessarily be expressed by a single formula,
so it is separated out from the other function types for special care during symbolic execution.

%% listing ruler max width ------------------------------------------------|?X
\begin{figure}[tp]
\begin{minted}{agda}
data func-type : Stack → Stack × Type → Set where
  D1   : ∀ {args res} → 1-func args res      →  func-type args ([] , res)
  Dm   : ∀ {args res} → m-func args res      →  func-type args       res
  PUSH : ∀ {ty}       → Pushable ty → ⟦ ty ⟧ →  func-type []   ([] ,  ty)
\end{minted}
\caption{functions for blockchain operations}
\label{func-type}
\end{figure}


\subsubsection{Shadow Stack and extended Instructions/Programs}\label{subsec:shadow}

\verb=DIP= and \verb=ITER= are special in that they need a second stack to be executed:
During the execution of the sub-program of \verb=DIP= the top \verb=n= elements must be stored
to be later retrieved when sub-program execution has terminated.
Likewise for \verb=ITER=, which consumes the list at the top of the stack
by executing its sub-program for every list element.
Here the currently remaining list has to be stored during sub-program execution.
Since the sub-programs can also contain such instructions that need to store data away for later,
a second stack is needed, as well as new instructions to operate on it:
To execute \texttt{DIP n prg ; \emph{remainingProg}}, the top \verb=n= elements of the
main stack are transfered to this second stack (we call it the \emph{shadow stack}) and
the instruction is replaced by \verb=prg= followed by the \emph{shadow instruction} \verb=DIP'=
which retrieves them from the shadow stack and puts them back onto the mainstack
before continuing execution with \texttt{\emph{remainingProg}}.
For the instruction \verb=ITER prg= the list on top of the main stack will be placed
onto the shadow stack and the instruction will be replaced by its shadow version \verb=ITER' prg=
which does all the actual work: It checks if the list at the top of the shadow stack is empty.
If so, it will be dropped and execution continues.
If not, the first element in the list will be moved to the top of the main stack and
\verb=prg= will be executed. After that \verb=ITER'= is executed again to check the list
until all elements have been iterated over.

These new shadow instructions must therefore be parameterized by 4 stacks:
main input stack, shadow input stack, main output stack and shadow output stack.
Analogous to the abbreviations in our code we will call the main stack \emph{real stack}.

Shadow programs are programs containing ``real'' and shadow instructions (see \listref{shadow}).

%% listing ruler max width ------------------------------------------------|?X
\begin{figure}[tp]
\begin{minted}[linenos]{agda}
data ShadowInst : Stack → Stack → Stack → Stack → Set where
  ITER'     : ∀ {ty rS sS}
            → Program      [ ty // rS ]                              rS
            → ShadowInst           rS   [ list ty // sS ]            rS  sS
  DIP'      : ∀ top {rS sS}
            → ShadowInst           rS        (top ++ sS )    (top ++ rS) sS

data ShadowProg : Stack → Stack → Stack → Stack → Set where
  end  : ∀ {rS sS} → ShadowProg rS sS rS sS
  _;_  : ∀ {ri rn si ro so}
       → Instruction ri     rn
       → ShadowProg  rn si  ro so
       → ShadowProg  ri si  ro so
  _∙_  : ∀ {ri si rn sn ro so}
       → ShadowInst  ri si  rn sn
       → ShadowProg  rn sn  ro so
       → ShadowProg  ri si  ro so
\end{minted}
\caption{shadow instructions and programs}
\label{shadow}
\end{figure}

Operators to concatenate programs and shadow programs are given in the canonical way:
\begin{figure}[tp]
\begin{minted}{agda}
_;;_ : ∀ {Si So Se} → Program Si So → Program So Se → Program Si Se
_;∙_   : ∀ {ri rn si ro so}
       → Program ri rn → ShadowProg rn si ro so → ShadowProg ri si ro so
\end{minted}
\caption{program concatenations}
\label{concat}
\end{figure}

\end{comment}

%%% Local Variables:
%%% mode: latex
%%% TeX-master: "itp2024"
%%% End:

\section{Michelson Reference Implementation}
\label{sec:refImpl}

Program execution is defined in a small-step manner by a function that maps
the current execution state of a program to a new state resulting from executing
the first instruction:
\ConcreteprogStep

The type $\ADT{CProgState}~ro$ is a record that contains an input
stack type $ri$, a program that maps an $ri$ stack to an $ro$ stack,
an input stack of type $ri$, and the execution
environment. \AFun{prog-step} executes the first instruction that must
map an $ri$ stack to an intermediate stack of type $re$,
say. Consequently, the program in the output \ADT{CProgState} maps an
$re$ stack to an $ro$ stack. As instructions as well as programs are
intrinsically typed, the intermediate stack type $re$ is sure to
match. Likewise, the typing of \AFun{prog-step} ensures type preservation.

% However, we have to add an
% instruction to represent intermediate states that arise while executing certain instructions.
% For that reason we recall their semantics from the Michelson specification
% before we present the formal definition of \ADT{ProgState}.

\subsection{Program Execution}
\label{sec:program-execution}

So far we only concerned ourselves with the type of a Michelson stack.
For program execution, both the types and values of stack elements are relevant.
To this end, we have to lift the interpretation of a single type,
i.e., a function from {\AType} to {\ASet}, to the interpretation of a
list of types. The library predicate \ADT{All} does exactly that: it
``maps'' a {\ASet}-typed function over a list, which yields (the type of) a
heterogeneously typed list.

For example, the value interpretation of a type stack is a value stack where
corresponding elements $t$ and $v$ are related by the type
interpretation, that is, $v : \ASem{t}$. 
\begin{minipage}[t]{0.4\linewidth}
  \FunctionsInt
\end{minipage}
\begin{minipage}[t]{0.4\linewidth}
  \ExamplesInt
\end{minipage}

The definition of a program state (see \autoref{fig:prog-step-example}) abstracts over a $Mode$ which
contains a type interpretation that allows us reuse the same structure
for concrete execution and abstract execution.
A program state contains the program that is currently executed,
the stack, and an environment which provides the
context information to execute blockchain instructions like
\ACon{AMOUNT} and \ACon{BALANCE}.
It is parameterized by the output stack type, which does not change during execution.
When executing more than one contract as we demonstrate in \secref{sec:contract-execution},
this parameterization ensures that the results from completed contract executions are well typed.

\begin{figure}[tp]
  \ConcreteProgState
  \ConcreteprogStepfct
  \caption{Program state and single program step execution (excerpt)}
  \label{fig:prog-step-example}
\end{figure}
The function \AFun{prog-step} executes the first instruction of a
program on the current state.
We explain two exemplary cases shown in
Figure~\ref{fig:prog-step-example}.  To explain the first stanza of
the code we have to make a
confession. As several instructions have very similar semantics, our
internal representation of instructions is a refinement of the
datatype shown in Figure~\ref{fig:core-michelson-instructions}. For
example, all instructions that just apply a function to the top of the
stack are grouped under a constructor \ACon{fct} and \ADT{func-type}
is the type defining these instructions.
\FunctionsInstructionfct

The function \AFun{app-fct} applies such a function to a concrete
stack. Roughly speaking, if the underlying function has type $a_1 \to \dots \to a_n \to
(r_1 \times \dots \times r_m)$ it gets transformed into a function
between heterogeneously typed lists
$[a_1, \dots, a_n] \to [r_1, \dots, r_m]$. We elide the definition and
just remark that the function $[{\times}\_]$ implements the
transformation between $(r_1 \times \dots \times r_m)$ and $[r_1,
\dots, r_m]$. The functions \AFun{H.front} and \AFun{H.rest} (in Fig.~\ref{fig:prog-step-example}) split the
input stack according to the stack types expected by the function \textit{ft}. The
function \AFun{H.++} is concatenation of heterogeneous lists.

The \ACon{DROP} instruction drops the top of the stack.


\subsection{Execution of Control Flow Instructions}\label{sec:control-flow}

We have chosen a small-step semantics because its stepwise progression
matches the stepwise proof rules of the dynamic logic. However, 
the Michelson specification defines the semantics in terms of a big-step
judgment.\footnote{The implementation splits it in four judgments for
  programs $\Downarrow$, instructions $\downarrow$, shadow programs $\Ddownarrow$,
  and shadow instructions $\downarrow'$.}
\BigstepConfiguration
\BigstepJudgment
It relates a configuration (environment and input stack of type $ri$) and a
program to an output stack of type $ro$.
% $p : \text{\emph{StackIn}} \mapsto \text{\emph{StackOut}}$.
The definition of the semantics in the Michelson specification takes
some liberties that require some extra machinery in a small-step execution
model. We discuss these issues with some representative instructions.

The instruction \ACon{IF-NONE}~\textit{p-none}~\textit{p-some} expects
a value of  type \ACon{option} on top of the stack.
If that value is \ACon{nothing} (the encoding of \ACon{NONE}), the
\textit{p-none} branch is executed on the rest of the stack:
\BigstepIfNone

% \[	\inferrule*	[Right=IF-NONE-true]
%  	{\textit{thn} : \text{\emph{StackIn}} \mapsto \text{\emph{StackOut}}}
%  	{\text{\ACon{IF-NONE} }\textit{thn}~\textit{els} : N\!O\!N\!E :: \text{\emph{StackIn}} 
% 		\mapsto \text{\emph{StackOut}}}
% \]

If however the top of the stack is $\ACon{just}~x$ (encoding $\ACon{SOME}~x$),
the \textit{p-some} branch is executed on the stack where  $\ACon{just}~x$
is replaced with $x$:
\BigstepIfSome
% \[	\inferrule*	[Right=IF-NONE-false]
% 	{\text{thn} : x :: \text{\emph{StackIn}} \mapsto \text{\emph{StackOut}}}
% 	{\text{\ACon{IF-NONE} thn els} : S\!\!\;O\!M\!E\; x :: \text{\emph{StackIn}} 
% 		\mapsto \text{\emph{StackOut}}}
% \]

To specify the corresponding small-step rule we introduce a type-respecting concatenation
operator $;\!\bullet$ on programs. 
The program $\ACon{IF-NONE}~$\textit{p-none}$~$\textit{p-some}$~;~$\textit{p-rest} either transitions to
\textit{p-none}~$;\!\bullet$~\textit{p-rest} or to \textit{p-some}~$;\!\bullet$~\textit{p-rest},
depending on the value on top of the stack.

The instruction $\ACon{DIP}~n~p$ executes program $p$ on the stack that results from removing the first $n$ elements
of the current stack and reattaches them afterwards.
\BigstepDIP
% \[	\inferrule*	[Right=DIP]
% 	{\text{prg} : \text{\emph{StackIn}} \mapsto \text{\emph{StackOut}}
% 	\\\\	length(\text{\emph{front}}) == \text{n}}
% 	{\text{\ACon{DIP} n prg} :	\text{\emph{front}} +\!\!\!+\; \text{\emph{StackIn}}
% 		\mapsto		\text{\emph{front}} +\!\!\!+\; \text{\emph{StackOut}} }
% \]

In the small-step version, dropping the first $n$ elements of the
stack is easy, but reattaching them requires extra machinery.
Thus, a mechanism for holding on to the top of the stack while executing the subprogram
and retrieving it afterwards is necessary.

Execution of \ACon{ITER} requires the same features in a slightly different way.
It consumes the list on top of the current stack.
If the list is empty, it is dropped from the stack:
%
\BigstepIterNil
% \[	\inferrule*	[Right=ITER-nil]
% 	{ }
% 	{\text{\ACon{ITER} prg} : N\!I\!L :: \text{\emph{StackIn}}  \mapsto \text{\emph{StackIn}} }
% \]

Otherwise the subprogram is applied to the first list element $v$ and then
the \ACon{ITER} instruction is reissued on the rest of the list $vs$
and the current stack:
\BigstepIterCons
% \[	\inferrule*	[Right=ITER-cons]
% 	{	\text{prg} :      x  :: \text{\emph{StackIn}}  \mapsto \text{\emph{StackOut}}
% 	\\\\	\text{\ACon{ITER} prg} : xs :: \text{\emph{StackOut}} \mapsto \text{\emph{StackEnd}} }
% 	{\text{\ACon{ITER} prg} : (x :: xs) :: \text{\emph{StackIn}}  \mapsto \text{\emph{StackEnd}} }
% \]

The typing for \ACon{ITER} requires that the type of the underlying stack must
be preserved, but the subprogram \textit{p-iter} is
entitled to access and modify the stack beyond the first element $x$.
Let's now consider stepwise execution. If the list on top has the form $v
:: vs$,  we need to stash the tail list $vs$ somewhere while the subprogram
processes the stack with $v$ on top.
After execution of the subprogram,
we have to recover $vs$ and try again with \ACon{ITER}.
% we need a mechanism to retrieve the rest of the list
% and either continue iterating over it or proceed with executing the rest of the program.

As subprograms can be arbitrarily complex, in particular, they
may contain \ACon{DIP} and \ACon{ITER}, we need a nestable solution.
To this end, we add a single new instruction \ACon{MPUSH1} that pushes
a single value on the stack. This instruction is
different from the normal \ACon{PUSH} instruction, which is limited to
\ACon{Pushable} values that have a textual representation.
\ConcreteShadowInst

The new instruction is a \emph{shadow instruction} because it does not
appear in input programs. It is indexed by two stack types like
any other instruction.
A \emph{shadow program} is defined like \ADT{Program}, but its first
instruction can be a normal instruction or a shadow instruction. Shadow programs only appear
at the top-level, never as subprograms nested in instructions. We elide the definition of
\ADT{ShadowProg} as it is analogous to \ADT{Program}. Moreover, we
provide a utility function \AFun{mpush} to generate a sequence of
\ACon{MUSH1} instructions from a list of values.
\ConcreteMpush


The small-step version of \ACon{DIP}~$n$~$dp$ takes the top $n$ elements from the stack
and starts executing the program $dp$ followed by 
the new instruction \ACon{mpush}~\textit{front} where
\textit{front} is the list of the $n$ values that were removed from
the stack.
\ConcreteprogStepDIP

The small-step version of \ACon{ITER}~$ip$ just pops the stack if the
list is empty.
Otherwise, if the top contains $v :: vs$, it pops this value, puts $v$
on top of the stack and executes $ip$ followed by
\ACon{mpush}~[\textit{ vs }] and then \ACon{ITER}~$ip$ and the rest of
the program.
\ConcreteprogStepITER

\begin{figure}[tp]
  \ConcreteExampleITER
  \caption{Simple program using \ACon{ITER}}
  \label{fig:ITER-ADD}
\end{figure}
For illustration, 
\tabref{prog-step:ITER-ADD} gives the stacks and shadow program of
each intermediate state resulting from applying \AFun{prog-step} to
the program in Figure~\ref{fig:ITER-ADD} until program termination
for the given input stack interpretation (omitting \ACon{end} for
readability).
This program adds a list of numbers on top of the stack to the number below.

%% listing ruler max width ------------------------------------------------|?X
\begin{table}[tp]
\begin{verbatim}
                  rSI                                    prg
----------------------------------------------------------------------
[ 18 , 24 ] ∷  0 ∷ []                             ITER (ADD)
         18 ∷  0 ∷ []         ADD ; MPUSH [ 24 ]; ITER (ADD)
              18 ∷ []               MPUSH [ 24 ]; ITER (ADD)
     [ 24 ] ∷ 18 ∷ []                             ITER (ADD)
         24 ∷ 18 ∷ []             ADD ; MPUSH []; ITER (ADD)
              42 ∷ []                   MPUSH []; ITER (ADD)
         [] ∷ 42 ∷ []                             ITER (ADD)
              42 ∷ []                                    end
\end{verbatim}
% \begin{tabular}{@{}rrr@{}}
% \toprule
% \midrule
% \bottomrule
% \end{tabular}
\caption{Program states during execution of Figure~\ref{fig:ITER-ADD}}
\label{prog-step:ITER-ADD}
\end{table}

% \ACon{ITER'} will check the remaining list on top of the shadowstack.
% If it still contains some elements, the program will be expanded again and the next list element
% will be placed on to the main stack. Otherwise the empty list is dropped and execution contiues
% with the next instruction after \verb/ITER/.


\subsection{Relation to Big-Step Semantics}
\label{sec:relation-big-step}


Executing a program requires iterating the \AFun{prog-step}
function. The Michelson implementation drives this iteration
by a step counter that is counted
down at each instruction.
\ConcreteprogStepStar

We prove that the original big-step semantics and our small-step
semantics are equivalent in the usual sense.
\BigstepToSmallstep
\BigstepFromSmallstep

\subsection{Contract Execution and Execution Chains}\label{sec:contract-execution}

The \AFun{prog-step} function can execute any Michelson program, not only those that comply
to the typing restrictions of a contract.
But it does not provide a mechanism to update the blockchain after successful contract execution
nor one to execute other blockchain operations which might be emitted by a contract.

\begin{comment}
When a contract execution terminates, the final stack interpretation will contain a pair
of a list of blockchain operations to be emitted by the contract as well as the updated
storage value of the contract.
Also contract execution is triggered by transfering some amount of Tezos tokens to it,
so it's balance and storage has to be updated and the emitted operations
must be staged for execution.
\end{comment}


To implement these aspects of contract execution, the \ADT{ProgState}
is augmented with further information as shown in Figure~\ref{fig:contract-execution-state}.
\begin{figure}[tp]
  \ConcretePrgRunning
  \ConcreteTransaction
  \ConcreteRunMode
  \ConcreteExecState
  \caption{Contract execution state}
  \label{fig:contract-execution-state}
\end{figure}
The record \ADT{PrgRunning} holds the contracts involved in the current
execution: \ACon{self} is the current contract and \ACon{sender} is
the sender (the account that started the current contract).
% (and makes sure that the executed program will terminate with the expected stacks,
% see \listref{PrgRunning}).
The \ADT{ExecState} holds the \ADT{Blockchain}, where contract execution results are saved,
and a list of pending blockchain transactions to be executed. A value
of type \ADT{Transaction} comprises a list of operations and the
address of the sender of these operations.
The field \ADT{MPstate} encodes the current mode of
execution. \ACon{Run} indicates that a contract is currently executing
the program in \ADT{PrgRunning} where we can take a step. \ACon{Cont}
indicates the transition between one contract and the next; execution
proceeds with the next pending blockchain operation. The $\mathcal{F}$
argument is used by the abstract execution to propagate information
between contract invocations. \ACon{Fail} indicates a failure along
with an error code in its $\mathcal{G}$ argument.

\begin{figure}[tp]
  \ConcreteExecStepProgram
  \caption{Program execution}
  \label{fig:exec-step-1}
\end{figure}
The function {\ConcreteExecStep} maps an execution state to its successor state
just like \AFun{prog-step} did for program states.
It only implements the features mentioned above that cannot be modeled
by the program state alone.
Its definition is too big to include it in full; instead
we briefly explain its implementation, giving each case in the same
order as in the implementation.
% \modulel{03-concrete-execution}{195}.

Figure~\ref{fig:exec-step-1} contains the cases when a contract is executing.
\begin{enumerate}
\item When execution of the current contract has terminated
  (i.e., \ADT{MPstate} is $\ACon{Run}~pr$ and \ADT{ProgState.prg} matches \ADT{end}),
  then intrinsic typing ensures that  the stack interpretation
  contains the emitted blockchain operations \textit{new-ops}
  paired with the new storage value \textit{new-storage}.
  We add the emitted operations to the \ACon{pending} field,
  update the terminated contract's storage on the blockchain, and
  switch to \ADT{RunMode} \ACon{Cont}. 
  % Otherwise the balances of the involved contracts must be updated as well.
  % In either case the emitted operations are appended to the pending operations
  % (together with the address from which they were emitted for later references).
\item In all other cases of a running program, its \ADT{ProgState} evolves using \ADT{prog-step}.
\end{enumerate}
In the remaining cases \ADT{MPstate} is $\ACon{Cont}~\ACon{tt}$ which
means that no contract is currently executed. In this case
\ACon{pending} is checked for other operations to be executed. 
Our model only implements the \ACon{TRANSFER-TOKENS} operation
that initiates a new contract execution.
We perform the following checks in this case:
\begin{itemize}
\item we fail unless the operation was emitted from a valid account;
    % any call on pending has a valid account address
    % \item if the sender account is valid, it must be checked whether the
    %   transfer operation is for the same contract or a different one
    %   %   why
\item we fail unless the type of the parameter matches the input type of the called contract;
  % this is guaranteed for calls on pending
\item we fail unless the target is a valid account;
\item we fail unless the sender's balance
  contains sufficient tokens to support the transfer.
    % first check is needed, but the second is not.
\end{itemize}
The first three cases can never occur during an actual execution of
a Michelson smart contract execution chain:
The \ADT{TRANSFER-TOKENS} instruction only works for values of type
$\Acontract~t$, which ensures validity of the target address and that the
parameter type is $t$. Moreover, operations can only be emitted by valid accounts.
The checks are needed in our model because it does not maintain
information about which addresses are valid contract addresses.
We chose not to include this information as it adds complexity
without contributing to our goal of proving the soundness of symbolic
execution. 


\begin{comment}

\subsubsection{Shadow Stack and extended Instructions/Programs}\label{subsec:shadow}

\ADT{DIP} and \ADT{ITER} are special in that they need a second stack to be executed:
During the execution of the sub-program of \verb=DIP= the top \verb=n= elements must be stored
to be later retrieved when sub-program execution has terminated.
Likewise for \verb=ITER=, which consumes the list at the top of the stack
by executing its sub-program for every list element.
Here the currently remaining list has to be stored during sub-program execution.
Since the sub-programs can also contain such instructions that need to store data away for later,
a second stack is needed, as well as new instructions to operate on it:
To execute \texttt{DIP n prg ; \emph{remainingProg}}, the top \verb=n= elements of the
main stack are transfered to this second stack (we call it the \emph{shadow stack}) and
the instruction is replaced by \verb=prg= followed by the \emph{shadow instruction} \verb=DIP'=
which retrieves them from the shadow stack and puts them back onto the mainstack
before continuing execution with \texttt{\emph{remainingProg}}.
For the instruction \verb=ITER prg= the list on top of the main stack will be placed
onto the shadow stack and the instruction will be replaced by its shadow version \verb=ITER' prg=
which does all the actual work: It checks if the list at the top of the shadow stack is empty.
If so, it will be dropped and execution continues.
If not, the first element in the list will be moved to the top of the main stack and
\verb=prg= will be executed. After that \verb=ITER'= is executed again to check the list
until all elements have been iterated over.

These new shadow instructions must therefore be parameterized by 4 stacks:
main input stack, shadow input stack, main output stack and shadow output stack.
Analogous to the abbreviations in our code we will call the main stack \emph{real stack}.

Shadow programs are programs containing ``real'' and shadow instructions (see \listref{shadow}).


Operators to concatenate programs and shadow programs are given in the canonical way:
\begin{listing}[!ht]
\begin{minted}{agda}
_;;_ : ∀ {Si So Se} → Program Si So → Program So Se → Program Si Se
_;∙_ : ∀ {ri rn si ro so}
     → Program ri rn → ShadowProg rn si ro so → ShadowProg ri si ro so
\end{minted}
\caption{program concatenations}
\label{concat}
\end{listing}

\end{comment}

%%% Local Variables:
%%% mode: latex
%%% TeX-master: "itp2024"
%%% End:

\section{Dynamic Logic for Michelson}
\label{sec:DL}

To obtain a dynamic logic suitable for symbolic execution we follow
the Key approach \cite{KeY3} and extend
first order logic with a modality $[p]$, where $p$ is a program state. The
intuitive meaning is that $[p]\Psi$ holds for a formula $\Psi$, if
running $p$ terminates in a state such that $\Psi$ holds. That is, the
formula $\Phi \to [p]\Psi$ has a similar meaning as the Hoare triple
$\{\Phi\}~p~\{ \Psi\}$.

In the following, we concentrate on the proof rules for the
modality. For instance (and ignoring the details of the program state
for now), $\Phi \to [end]\Psi \equiv \Phi \to \Psi$ if the program is
empty. Many simple proof rules have the form $\Phi \to [i; p]\Psi \equiv \Phi_i
\wedge\Phi \to  [p]\Psi$ where the formula $\Phi_i$ describes the effect of
instruction $i$. If the instruction is a branch instruction on a
predicate $Q$, like $\mathtt{if}~Q~p_1~p_2$, the resulting fomula is a
disjunction as in $\Phi \to [(\mathtt{if}~Q~p_1~p_2); p]\Psi \equiv Q
\wedge \Phi \to [p_1; p]\Psi \vee \neg Q \wedge \Phi \to  [p_2; p]\Psi$.

We start by defining the formulas of the logic in \autoref{sec:terms-formulas}.

\subsection{Terms and Formulas}\label{sec:terms-formulas}
\begin{figure}[tp]
  \AbstractTerm
  \AbstractFormula
  \caption{Terms and formulas}
  \label{fig:terms-and-formulas}
\end{figure}

At the core of any symbolic execution there are symbolic (i.e., logical) variables
representing the unknown operands.
We represent such variables by a typed deBruijn index into a given
{\AbstractContext}.
An abstract stack is then a list of typed variables:
\AbstractMatch

\autoref{fig:terms-and-formulas} shows the terms and formulas used for the logic.
Term comprise variables, constants of base type and of contract type,
and simple functions. Here, ``function'' stands for proper functions as
well as data constructors. For convenience, we restrict function arguments
to variables and rely on variable equality in the formulas to specify
complex terms. 
% The term for subtracting mutez' is not representable with a functional Michelson instruction
% and therefore defined as a separate term.

Formulas are restricted to express equality of a variable with a term
and to impose order on tokens. The latter is used for token transfers
where we have to know that the sender has sufficient tokens to satisfy
the requirements of the transfer. The reader may wonder about 
conjunction and disjunction: the proof rules only generate them in the
form of a disjunction of conjunctions of simple formulas.  We
represent this structure as a list of lists of simple formulas.


\subsection{Representing Michelson Programs in DL}\label{sec:abstract-states}

We simplify the handling of formulas of the form $\Phi \to [p]\Psi$ by
reusing our previous definition of the type \ADT{ProgState} in a
different mode as an \emph{abstract} state.
\AbstractAMode
That is, we replace the normal representation of values by symbolic
variables, in $\mathcal{F}$ we maintain a list (i.e., conjunction) of
formulas, and in $\mathcal{G}$ we maintain a tagged list of formulas. 

The meaning of an abstract state is a conjunction that specifies the
value for \ACon{AMOUNT} and \ACon{BALANCE} in the environment, it
specifies the size of the stack and all values on it, and it collects
further constraints generated by application of the proof rules.



\begin{comment}
  The main focus of this thesis is the developement of symbolic
  execution rules that are used to reduce the programs in such
  formulas successively until one is left with a purely first formula
  that does not include any more program statements.  We only give the
  formalization of these symbolic execution rules in Agda since the
  resulting first order formulas do not contain Michelson specific
  aspects and can be easily formalized like any other first order
  logic.  See \cite{KeY2,DL} for more information on that topic.

  Analogous to the two different execution models for Michelson, one
  only for programs and one for contract execution chains, the
  symbolic execution rules are given for programs and contract
  execution chains as well.  In case of contract execution chains,
  this formalizes a dynamic logic that extends the DL about Michelson
  programs to one about blockchain operations.

  For a DL in general the preconditions can be empty, but our
  formalization in Agda will always assume some preconditions, which
  makes sense in both cases: Even when reasoning about a program that
  does not consume any stack elements (e.g. proving something about
  the values the program pushed onto the stack), we can only ever
  prove something about the resulting values of stack elements, and to
  reference their position on the stack we would assume that the
  initial stack is empty.
  % This formalization generalizes to be correct for any unaffected
  % substack.  For Contract executions, they terminate when all
  % pending operations have been accounted for,
  To save the results from modeling a contract execution, it must at
  least be assumed that the contract under consideration is present in
  the blockchain.  So both formalizations always include at least some
  preconditions on the input stacks or the state of the blockchain.
\end{comment}
\begin{comment}
Most rules of the presented Michelson DL calculus have the form:
\begin{align*}
  Pre  &\Longrightarrow \langle i ; p \rangle Ass
  \\ \leftrightarrow \qquad I \land  Pre' &\Longrightarrow \langle     p \rangle Ass
\end{align*}

where $I$ is a new clause containing new variables and variables from
$Pre$, and $Pre'$ only differs from $Pre$ in clauses about the state
of the stacks.  For control flow instructions the rules may contain
disjunctions and subprogram expansions:
\begin{align*}
  Pre  &\Longrightarrow \langle i    ; p \rangle Ass
  \\ \leftrightarrow \qquad (I \land Pre'  \Longrightarrow \langle p'  ;; p \rangle Ass)
       &\lor  (J \land Pre'' \Longrightarrow \langle p'' ;; p \rangle Ass)
\end{align*}

Notice that $Ass$ is never affected.  This is unlike the calculus of
Dynamic Logics for other programming languages where an assignment
might affect the value of a variable contained in $Ass$.  But
Michelson does not have any variables and therefore no assignments to
them, so we are able to neglect $Ass$ when formalizing the symbolic
execution rules.  So our formalization will specify how we get from
$Pre$ to $I \land Pre'$ when symbolically executing instruction $i$,
where $Pre$ is a conjunction of formulas that will always include
statements about the program stacks as well as the execution
environment (which is necessary to execute \verb/env-func/
instructions).  That is when we only deal with symbolic program
execution; in case of contract execution chains, $Pre$ will always
contain statements about the state of the blockchain, as well as some
additional information in cases where a currently running contract
execution is formalized.
\end{comment}
\begin{comment}
  These rules never affect the goal formula to be asserted, and
  because (which must always be given), and $Ass$ can be any formula.
  Our construct specifys $Pre \Longrightarrow \langle p \rangle$ but
  not $Ass$ since it neither changes, nor has any implications for the
  validity of our rules \draft{other than the premisse holds with
    $Ass$ \emph{iff} the conclusion holds with it (i guess)}.

  This matches the sequent rule (R8) from \todo{cite lecture notes
    2010} for functional Michelson instructions, since here $x$ will
  be the new top stack element, which our DL always represents with a
  new variable, thus $\phi_x^{\hat{x}} \equiv \phi$.  \todo{not sure
    about control flow instructions \ldots}

  We will proof the soundness of these rules by showing that if an
  interpretation for $Pre$ is a valid interpretation for a concrete
  execution state with $i ; p$ left to be executed (i.e. stack values
  match and $Pre$ otherwise doesn't contradict itself), we can give an
  extension of that interpretation such that $I \land Pre'$ is a valid
  interpretation of the concrete state after execution of $i$.

  Since the extension of the interpretation is only for possible new
  variables that are not present in $Ass$, it follows from our
  soundness proof \draft{that our calculus is sound \ldots}.
\end{comment}

Informally, an abstract program state
represents $\Theta \Longrightarrow \langle prg \rangle \Psi$ where
\begin{align*}
	\Theta&\equiv	\text{\emph{state of environment}} = \alpha en
\land	\text{\emph{state of stack}} = rSI
\land	\bigwedge_{\phi \in \Phi} \phi
\end{align*}


This encoding makes the formalization of symbolic execution very similar to the concrete execution model
presented in~\autoref{sec:refImpl}.
This similarity in turn makes the soundness proof easier and more concise.
All constructs for concrete execution are reused in the abstract by
instantiating their \ADT{MODE} parameter.
Thus, they  are automatically parameterized by a \verb/Context Γ/ and the names of the
structures are the same as for concrete execution but prefixed with an \verb/α/
(only the abstract blockchain is called \verb/βlockchain/).


Symbolic execution can lead to disjunctions over such states, which is presented using
a list of abstract program states.
\AbstractUProgState
% \mint{agda}|⊎Prog-state = λ {ro} {so} → List (∃[ Γ ] αProg-state Γ {ro} {so})|

Using Agda lists to represent conjunctions and disjunctions is
convenient for two reasons.
\begin{enumerate}
\item Conjunctions and disjunctions do not mix: $\Phi$ always represents a conjunction over its elements
  and disjunctions can only occur as a result of some
  symbolic execution rules that implement control flow. 
  In this case, the disjunction always affects every aspect of the abstract program state
  (i.e., the remaining programs will always differ).
\item Agda's ``element of'' relation for lists makes the
  implementation of the rules of the calculus simpler and more efficient.
\end{enumerate}

\subsection{Calculus for Michelson DL}\label{sec:calculus}

The rules for symbolic execution are formalized by a function that
maps an abstract program state into a set (list) of abstract program states.
\AbstractProgStep

It mimics \AFun{prog-step} and gives a deterministic way of symbolic execution.
Every (non-environmental) functional instruction can be executed concretely
with a single rule as shown in \autoref{fig:exec-step-1}.
During symbolic execution, the only thing that is guaranteed is
that the stacks contain values of the expected type.
For example, if the next instruction is \ACon{ADDnn}, we can conclude
that there are two values of type \ACon{nat} on top of the stack
before the instruction and one value of type \ACon{nat}
afterwards. Moreover, we can say that this value is the sum of the two
values that were on top of the stack before, but we have to express
that with a constraint, i.e., a logical formula.

That is, symbolic exection of \ACon{ADDnn} introduces a new variable $v_r$ that replaces the
variables $v_x$ and  $v_y$ from the top of the stack,
and adds a clause that equates this new variable with the sum of the former two:
\[	v_r := \ACon{ADDnn}\; v_x\; v_y
\]

In this way, we can give a single symbolic execution rule for all
functional instructions that return a single result.
\AbstractProgStepDOne

Let's decompress this definition. We pattern match against the current
(abstract) state to obtain the environment {\Aaen}, the current
instruction, the stack {\Aast}, and the formula $\Phi$. The
constructor \ACon{fct} indicates a functional instruction and the
constructor \ACon{D1} indicates that $f$ returns a single result of
type \emph{result}.

As the instruction does not implement any control flow, there is only
a single next state. Its description starts with the extended context
$result :: \Gamma$, which introduces a new variable of type $result$
for the result. The name, rather the deBruijn address, of this
variable is \AZERO, which denotes the first entry in the context. The
second component describes the new state, which (ignoring the
\AFun{wk} functions for the moment) keeps the environment, moves to
the rest of the program, pushes the result on the stack after removing
the arguments using \AFun{H.rest}, and pushes a new equation that
defines the value of {\AZERO} as the result of applying $f$ to the
front of the stack.
The functions \AFun{H.front} and \AFun{H.rest} operate on heterogenous
lists and are defined such that
$\Aast \equiv \AFun{H.front}~\Aast~\AFun{H.++}~\AFun{H.rest}~\Aast$ where
the actual division is driven by the type of $f$. The operation
$\AFun{H.++}$ is concatenation of heterogenous lists.
The \AFun{wk} functions are a consequence of using deBruijn indices
for variables: if we introduce new variables, all existing variables
have to be incremented by the number of new variables (i.e.,
weakened). We do not show their definition, 
as this manipulation of deBruijn indices is standard.

%% listing ruler max width ------------------------------------------------|?X

\begin{figure}[tp]
  \AbstractProgStepUNPAIR
  \AbstractProgStepSWAP
  \AbstractProgStepPUSH
  \caption{Functional instructions (excerpt)}
  \label{fig:aprog-step-func}
\end{figure}

\begin{comment}
  The matching representing the stack can be split implicitly just
  like the stack interpretation during concrete execution.  \verb/n∈/
  is another Agda pattern synonym for the n'th element of the context.
  Whenever the context is extended during symbolic execution, all
  elements of the abstract program state that do not change must be
  weakened so they can be parameterized by the new context.
\end{comment}
We do not have a general mechanism for the other functional
instructions (see~\autoref{fig:aprog-step-func}), as they behave very differently in a symbolic context:
\ACon{UNPAIR} requires two new variables and clauses, while
\ACon{SWAP} only changes the position of two stack values
so the entire change is encoded in the state of the stack
and no new clauses or variables are necessary.

The instruction \ACon{PUSH} needs special treatment because it can
handle arbitrarily complex compound values.
When pushing a value $x$ of primitive type, it is sufficient to add a new variable
and a clause which sets this variable equal to the term $\ACon{const}~x$.
But if $x$ has a list type or an option type, its value cannot be
expressed with a \ACon{const} term, in general.
To this end, the function $\AFun{unfold}~P~x$ creates all clauses required
to express the value $x$. This process defines a list of new variables
of types defined by $\AFun{expandΓ}~ P~ x$.\footnote{We do not include
the tedious definitions of these auxiliary functions here, but
encourage the interested reader to check the supplementary material.}
For example, $\ACon{PUSH}~\{\ACon{list ty}\}~P~(y~\ACon{∷}~ ys)$ gives
rise to two new variables $r_y$ of type
\ACon{ty} for $y$ and $r_{ys}$ of type \ACon{list ty} for $ys$ and an
equation $r := \ACon{func}~(\ACon{CONS}~[r_y, r_{ys}])$, where $r$ is the variable
for the result. The function \AFun{unfold} proceeds recursively:
if $ys = []$, its variable can be set to $\ACon{func}~(\ACon{NIL}~ty)~[]$,
otherwise it will be further decomposed.
Similarly for $y$: if $ty$ is a primitive type, it can be set to $\ACon{const}~y$,
otherwise it must be further decomposed as well.
% This process is arbitrarily complex, but of course finite ;)

\begin{comment}
  When \verb/UNPAIR/ is executed on a variable \verb/p∈/, and it is
  known about \verb/p∈/ that it is a \verb/PAIR/ of two other
  variables \verb/v₁∈/ and \verb/v₂∈/, there is no need to introduce
  two new variables and express their values with respect to \verb/p∈/
  as the generic \verb/αprog-step/ would (see
  \listref{aprog-step-func}).  These new variables would be equal to
  \verb/v₁∈/ and \verb/v₂∈/ and it is better to use them instead.
\end{comment}
\begin{comment}
  Also, when the argument values for a onedimensional function are all
  given as constant terms (which is formalized by the
  \verb/MatchConst Φ Margs/ construct that associates every argument
  variable \verb/v∈/ from \verb/Margs/ with a clause
  \verb/v∈ := const x/ in \verb/Φ/), it is not necessary to express
  the value of the result in a functional term since it can just be
  calculated.
\end{comment}
\begin{comment}
  \listref{a-ITER'} shows some symbolic execution rules for the
  \verb/ITER'/ instruction: Generally, when no further information
  about the relevant list variable is present, symbolic execution will
  lead to a disjunction that considers both possibilities.  But if
  some relevant information is known about that variable, the
  disjunction as well as the context extension can be avoided.

  \begin{listing}[!ht]
\begin{minted}{agda}
αprog-step {Γ} α@(αstate αen (ITER' {ty} x ∙ prg) rVM (l∈ ∷ sVM) Φ)
  = [ _ , record α{ prg = prg ; sVM = sVM 
                  ;   Φ = [ l∈ := func (NIL ty) [M] // Φ ] }
    / [ ty / list ty // Γ ]
    , αstate (wkαE αen) (x ;∙ ITER' x ∙ prg) (0∈ ∷ wkM rVM) (1∈ ∷ wkM sVM)
             [ 2+ l∈ := func CONS (0∈ ∷ 1∈ ∷ [M]) // wkΦ Φ ] ]

  ITER'c : ∀ {αen ty l∈ x∈ xs∈ Φ rS sS iterate prg rVM sVM}
         → l∈ := func CONS (x∈ ∷ xs∈ ∷ [M])  ∈  Φ
         → αρ-special
               (αstate αen (ITER' {ty} {rS} {sS} x ∙ prg) rVM  (l∈ ∷ sVM) Φ)
           (_ , αstate αen (x ;∙ ITER' x ∙ prg)    (x∈ ∷ rVM) (xs∈ ∷ sVM) Φ)
\end{minted}
\caption{Exemplary symbolic execution rules for ITER'}
\label{a-ITER'}
\end{listing}

It is possible to reach the same logical conclusions as
\verb/αρ-special/ by first performing a generic symbolic execution
step with \verb/αprog-step/ and then applying some first order
conclusions on the resulting formula.  For example in the case of
\verb/ITER'/, if it is known that \verb/l∈/ is \verb/CONS/ and the
first disjunct from \verb/αprog-step/ adds the clause that sets it
equal to \verb/NIL/, a contradiction can be concluded from these two
clauses and the entire disjunct can be set to false and discarded.

Such rules and their soundness have already been successfully
implemented and proven in an earlier version of the model.  But since
they come at a great performance cost by requiring more conclusion
steps and creating new redundant variables most of the time, these
rules where replaced by the given special relation transitions during
a major rework of the model.
\end{comment}

\begin{figure}[tp]
  \AbstractProgStepIFNONE
  \caption{Symbolic execution of \ACon{IF-NONE}}
  \label{fig:sym-exec-if-none}
\end{figure}
We finish with the abstract execution of the conditional instruction
\ACon{IF-NONE} (see~\autoref{fig:sym-exec-if-none}). This instruction
expects a value of type $\ACon{option}~t$ on top of the stack. Here we have two
possible next states, depending on whether the value is present. The
first disjunct deals with the case where the value is \ACon{NONE}. In
this case, the stack is popped, the \emph{thn} branch is taken, and
the equation enforcing the value to be \ACon{NONE} is added. There are
no new variables, so there is no weakening in this disjunct.

The second disjunct models the case where the value on top of the stack is
$\ACon{SOME}~y$. Here we need a new variable of type $t$ for $y$, pop
the stack and push the new variable, we take the \emph{els} branch,
and add an equation that forces the value to be $\ACon{SOME}~y$.

\subsection{Calculus for the DL on blockchain operations}
\label{sec:calc-dl-blockch}

Just like the symbolic execution rules for the Michelson DL,
those for the DL on blockchain operations
are given analogously.
\AbstractUExecState
\AbstractAexecStep


%% listing ruler max width ------------------------------------------------|?X
% \begin{listing}[!ht]
% \begin{minted}{agda}
% record αExec-state Γ : Set where
%   constructor αexc
%   field
%     αccounts : βlockchain Γ
%     αρ⊎Φ     : αPrg-running Γ ⊎ List (Formula Γ)
%     pending  : List (list ops ∈ Γ × ⟦ base addr ⟧)

% ⊎Exec-state = List (∃[ Γ ] αExec-state Γ)

% αexec-step : ∀ {Γ} → αExec-state Γ → ⊎Exec-state

% data ασ-special {Γ} : αExec-state Γ → ⊎Exec-state → Set where
% \end{minted}
% \caption{Symbolic execution rules for abstract execution states}
% \label{aexec-all}
% \end{listing}

The switch from concrete to abstract execution state is achieve by
changing the $Mode$ parameter of the \ADT{ExecState} (see
\autoref{fig:contract-execution-state}). Its $\mathcal{F}$ field
replaces concrete semantics by abstract semantics throughout all state
components. 


Unfortunately \AFun{αexec-step} cannot represent \AFun {exec-step}
exactly, if \AField{MPstate} is \ACon{Cont}~$\Phi$, that is: a
contract has terminated and we need to check the \AField{pending}
field for further operations to be executed. 
At this point, the predicate $\Phi$ has to supply sufficient information about the values of the variables
representing the pending operations to proceed in a meaningful way.
The \AField{pending} list contains pairs of a list of operations and a
sender address. While the latter is a concrete address, the former is
a variable of type \verb/list operation ∈ Γ/. To proceed, we have to
know if the list  is empty (so that we can proceed to the next block
of pending operations) or not. In the latter case, we need to ensure
that the first element of the operation list is a
\ACon{TRANSFER-TOKENS}, and so on.

To this end, we defined several auxiliary functions to inspect the
constraints in $\Phi$ for patterns that restrict the models
sufficiently. For example, the function \AFun{find-tt-list} takes a
conjunction of formulas and a variable of type $\ACon{list}~t$ and
tries to find a formula that restricts this variable to \ACon{NIL} or
\ACon{CONS}:
\AbstractFindTTList
\SoundnessFindTTList

We only show the soundness lemma for \ACon{NIL}, as the one for
\ACon{CONS} is analogous. This approach is not complete as the
implementation of \AFun{find-tt-list} is tailored to the constraints
as they are produced by symbolic execution. 

The full implementation is quite involved and relies on several
further lemmas that examine constraints (for example if the current
balance of a sender is sufficient for a token transfer) in a similar
way. We refer the interested reader to the supplement.

The remaining cases deal with a terminated contract execution where
the new state is written back to the blockchain or the execution of an abstract program step for the contract under execution.
The first case is similar to the concrete implementation where new variables are introduced
for the updated values.
The second case is more complicated because the context extensions from the abstract program step
are encoded in the list of resulting disjunctions,
so an additional term has to be supplied proving that these contexts are actually an extension
of the original context.


\begin{comment}
\verb/αPrg-running/ is the same as its non-abstract counter part holding abstract contracts
and program states instead of concrete ones.

Besides using an abstract variant of \verb/blockchain/, the abstract version of the execution state
also differes in its other fields.
In the concrete setting, the new values for the contracts storage and balance
are written directly to the \verb/accounts/,
but the abstract contracts only hold variables for these entities, who's values are only
described by $\Phi$.
So when an abstract contract execution terminates, all of the information stored in
\verb/αPrg-running/ can be discarded, except for the formulas which will be stored
in case no contract is executed in the current state.
Also, since the emitted blockchain operations will also be stored as variables,
\verb/pending/ holds such variables instead of actual operations.

%% listing ruler max width ------------------------------------------------|?X
\begin{listing}[!ht]
\begin{minted}{agda}
record αProg-state Γ {ro so : Stack} : Set where
  constructor αstate
  field
    {ri si} : Stack
    αen : αEnvironment Γ
    prg : ShadowProg ri si ro so
    rVM : Match Γ ri
    sVM : Match Γ si
    Φ   : List (Formula Γ)

record αExec-state Γ : Set where
  constructor αexc
  field
    αccounts : βlockchain Γ
    αρ⊎Φ     : αPrg-running Γ ⊎ List (Formula Γ)
    pending  : List (list ops ∈ Γ × ⟦ base addr ⟧)
\end{minted}
\caption{Single abstract program and execution states}
\label{abstract-states}
\end{listing}

During symbolic execution, unlike in a concrete setting, execution is not deterministic
since the values for variables necessary for case distinctions may not be present in the current
execution state, i.e. $\Phi$.
This is modelled by giving abstracts states
as disjunctions of the states from \listref{abstract-states}:
%% single line listing ruler max width ------------------------------------------------|X
\mint{agda}|⊎Prog-state = λ {ro} {so} → List (∃[ Γ ] αProg-state Γ {ro} {so})|
\mint{agda}|⊎Exec-state = List (∃[ Γ ] αExec-state Γ)|

It is necessary to place the disjunctions on the level of abstract program and execution states
rather than defining disjunctions on the level of formulas,
because when they occure due to the symbolic execution of a given abstract state,
the resulting states also differ in the resulting stack matchings and rest program
(in case of a program state) or the resulting current execution state \verb/αρ⊎Φ/.

Using Agda Lists to represent conjunctions and disjunctions is conveniet because
\begin{enumerate}
	\item	within any of the given listing constructs ($\Phi$, \verb/αXxxx-state/)
		all elements are interpreted exclusively as con- or disjuncts,
		so defining them separately would only yield another list like structure.
	\item	some rules of the calculus as well as the semantics for our model
		can make efficient use of the element relation for Agda Lists.
\end{enumerate}

Note that while the storage and balance of abstract contract are modeled with respective variables,
their addresses on the abstract blockchain are \emph{NOT} modeled with variables but rather
with given values like in the concrete setting.
Abstracting them would be possible, but it wouldn't make the logic any more expressive
since cases where an address value on the stack is not known can still be modelled,
and having abstract contract addresses would only effect the symbolic execution of a
\verb/TRANSFERE-TOKEN/ operation \todo{annoying to formulate \ldots!!!!!!}

We give the calculus in a different way (as mentioned in \secref{sec:abstract-states})
by giving rules that will add new formulas to those already saved in the abstract states
for the abstract execution of the next instruction (or operation).
Or rather it will give a new abstract state disjunction that represents a formula
which is valid if the formula represented by the initial state
was valid before instruction execution (as we will proof in \chapref{chap:soundness}).

They are given in two different forms:

One gives the conclusion formula in a deterministic way that can always be applied to any
possible Michelson instruction, regardless of the values that may or may not be
provided for the relevant variables by the premise formula.
This resembles the deterministic execution of concrete states and is achieved
by accordingly named functions:
%% single line listing ruler max width ------------------------------------------------|X
\mint{agda}|αprog-step : ∀ {Γ ro so} → αProg-state Γ {ro} {so} → ⊎Prog-state {ro} {so}|
\mint{agda}|αexec-step : ∀ {Γ} → αExec-state Γ → ⊎Exec-state|

While in the concrete case, any kind of functional instruction can be performed by the compact
definition given in \listref{prog-step}, this can only be achieved for the subset of
onedimensional functions here (see \listref{aprog-step}):

%% listing ruler max width ------------------------------------------------|?X
\begin{listing}[!ht]
\begin{minted}{agda}
αprog-step {Γ} (αstate αen (fct (D1 {result} f) ; prg) rVM sVM Φ)
  = [ [ result // Γ ]
    , (αstate (wkαE αen) prg (0∈ ∷ wkM (Mbot rVM)) (wkM sVM)
              [ 0∈ := wk⊢ (func f (Mtop rVM)) // wkΦ Φ ]) ]

αprog-step {Γ} (αstate αen (fct (Dm (UNPAIR {t1} {t2})) ; prg)
                       (p∈ ∷ rVM) sVM Φ)
  = [ [ t1 / t2 // Γ ]
    , (αstate (wkαE αen) prg (0∈ ∷ 1∈ ∷ wkM rVM) (wkM sVM)
              [ 0∈ := wk⊢ (func CAR (p∈ ∷ [M]))
              / 1∈ := wk⊢ (func CDR (p∈ ∷ [M])) // wkΦ Φ ]) ]
αprog-step α@(αstate αen (fct (Dm SWAP) ; prg) (x∈ ∷ y∈ ∷ rVM) sVM Φ)
  = [ _ , record α{ prg = prg ; rVM = y∈ ∷ x∈ ∷ rVM } ]
αprog-step α@(αstate αen (fct (Dm DUP)  ; prg) (x∈      ∷ rVM) sVM Φ)
  = [ _ , record α{ prg = prg ; rVM = x∈ ∷ x∈ ∷ rVM } ]

αprog-step {Γ} (αstate αen (fct (PUSH P x) ; prg) rVM sVM Φ)
  = [ (expandΓ P x ++ Γ)
    , (αstate (wkαE αen) prg ((∈wk (0∈exΓ P)) ∷ wkM rVM) (wkM sVM)
              (Φwk (unfold P x) ++ wkΦ Φ)) ]
\end{minted}
\caption{Deterministic symbolic program state execution}
\label{aprog-step}
\end{listing}

It is possible for those functions since they double as the functional terms
that can be expressed in the DL.
A new variable of the result type is added to the Context and a new conjunct to $\Phi$,
stating that the new variable equals the term
that applies the argument variables from the top of the stack matching to that function.
The weakening operators \verb/wk/\emph{XY} accommodate for the new context.

For other functions, no generic symbolic execution is possible:
In case of the \verb/UNPAIR/ instruction, two new variables and corresponding clauses must be added,
while for \verb/SWAP/ and \verb/DUP/ nothing has to be added
(hence the context doesn't change, which Agda knows implicitly , and no weakenings are applied)
and only the variables on the top of the main stack matching must be changed.

\verb/PUSH P x/ is a special case for the abstract execution as mentioned before:
When pushing a basic type, an according variable and a clause
setting it equal to the \verb/const/ term \verb/x/ (see \listref{terms-formulas}) can be added.
In case of a complex type, \verb/x/ must be decomposed into
several new variables and clauses for those that represent \verb/x/
(e.g.: \verb/PUSH (list {ty} P) (y ∷ ys)/ would be decomposed into two new variables of type
\verb/ty/ and \verb/list ty/ where the first would be set to \verb/y/ and the second to \verb/ys/;
in case of \verb/ys = []/ that would be \verb/func (NIL ty) [M]/,
otherwise \verb/ys/ must be further decomposed,
as well as \verb/y/ in case \verb/ty/ is also a complex/composite type).

Abstract execution of \verb/ITER/ works just like the concrete case (see \listref{aITER}),
but while the top element of the shadow stack interpretation can be matched
to the only two possible values for the concrete execution
(an empty or non-empty list, see \listref{prog-step}),
this isn't possible for the variable \verb/l∈/
on top of the shadow stack matching in the abstract case.
So instead of giving two possible execution rules, only one is given that introduces a
disjunction for the two possibilities that can arise:

%% listing ruler max width ------------------------------------------------|?X
\begin{listing}[!ht]
\begin{minted}{agda}
αprog-step α@(αstate αen (ITER x ; prg) (l∈ ∷ rVM) sVM Φ)
  = [ _ , record α{ prg = ITER' x ∙ prg ; rVM = rVM ; sVM = l∈ ∷ sVM } ]

αprog-step {Γ} α@(αstate αen (ITER' {ty} x ∙ prg) rVM (l∈ ∷ sVM) Φ)
  = [ _ , record α{ prg = prg ; sVM = sVM 
                  ;   Φ = [ l∈ := func (NIL ty) [M] // Φ ] }
    / [ ty / list ty // Γ ]
    , αstate (wkαE αen) (x ;∙ ITER' x ∙ prg) (0∈ ∷ wkM rVM) (1∈ ∷ wkM sVM)
             [ 2+ l∈ := func CONS (0∈ ∷ 1∈ ∷ [M]) // wkΦ Φ ] ]
\end{minted}
\caption{Generic abstract execution for ITER}
\label{aITER}
\end{listing}

However, it may be possible that the premise includes additional information about \verb/l∈/,
if a clause is present in $\Phi$ where its stated to be equal to some term.
If that term happens to be either \verb/func (NIL ty) [M]/ or \verb/func CONS (x∈ ∷ xs∈ ∷ [M])/,
one case can savely be discarded.
To enable the DL to reach such conclusions, the following relation on abstract program states
is defined:

%% listing ruler max width ------------------------------------------------|?X
\begin{listing}[!ht]
\begin{minted}{agda}
data αρ-special {Γ ro so} :         αProg-state        Γ  {ro} {so}
                          → ∃[ Γ` ] αProg-state (Γ` ++ Γ) {ro} {so}
                          → Set where

  ITER'c : ∀ {αen ty l∈ x∈ xs∈ Φ rS sS iterate prg rVM sVM}
         → l∈ := func CONS (x∈ ∷ xs∈ ∷ [M])  ∈  Φ
         → αρ-special
               (αstate αen (ITER' {ty} {rS} {sS} x ∙ prg) rVM  (l∈ ∷ sVM) Φ)
           (_ , αstate αen (x ;∙ ITER' x ∙ prg)    (x∈ ∷ rVM) (xs∈ ∷ sVM) Φ)
\end{minted}
\caption{Special case (e.g. ITER' for CONS case)}
\label{arho-special}
\end{listing}

This also has the advantage that no new variables have to be introduced.

The same conclusion could have been drawn from first performing the generic case that will
create a disjunction, and then concluding a contradiction from the other clause
by a rule that sets $\Phi$\verb/ = `false/ when two contradicting clauses like
\verb/l∈ := func (NIL ty) [M]/ and \verb/l∈ := func CONS (x∈ ∷ xs∈ ∷ [M])/ for the same variable
are contained in $\Phi$.

Such rules have already been successfully implemented and proofen
in an earlier version of the model.
But since they come at a great performance cost by requiring 3 conclusion steps instead of one
and often creating new variables that make additional conclusions necessary to connect them,
these rules where not reimplemented during a major rework of the code.

As for abstract program states, the symbolic execution of abstract execution states
is also given once in a deterministic manner as mentioned above,
but also by a similar relation for special cases.

There is a significant difference however: 
When checking \verb/pending/ for further operations to be executed,
without further knowledge about the variables representing these operations,
no meaningful action can be performed to advance the symbolic execution process.
To execute any succeeding operation, it must be at least given that
\verb/list ops ∈ Γ/ is \verb/CONS/ of such an operation variable,
which in turn must also be given to have the value of a \verb/TRANSFERE-TOKENS/ term,
for which the \verb/contract/ variable must be given a value as well.
Without such additional information, which cannot be considered for the deterministic case,
the symbolic execution can only create a case distinction between the list of operations
being either empty or filled with at least one operation which still cannot be processed
due to the missing information.
Therefore the abstract execution step function cannot reflect the exact behaviour of its
concrete counterpart, and only for those cases where the abstract state contains a
contract under execution will the soundness of the symbolic execution be prooven.

\end{comment}


%%% Local Variables:
%%% mode: latex
%%% TeX-master: "itp2024"
%%% End:

\section{Semantics and Soundness}
\label{sec:semantics-soundness}

\subsection{Values and models}\label{sec:values&models}
% For the semantics \todo{check for precise usage of semantics \ldots} we 

\begin{figure}[tp]
  \SemanticsTerms
  \SemanticsFormulas
  \caption{Semantics of terms and formulas}
  \label{fig:semantics-terms}
\end{figure}

As a context is just a list of types like a stack, its interpretation
is also a heterogeneous list of values as defined by \ADT{Int}.
For a given context interpretation $\gamma$, the semantics of a term
and a formula is defined as usual
(see \autoref{fig:semantics-terms}).

For a given context interpretation \verb/γ/ and abstract and concrete (program or execution) states,
the predicates \AFun{modρ} and \AFun{modσ} express that
under this interpretation the given abstract state models the concrete state.
This is the case when the formulas in \verb/Φ/ are true under \verb/γ/ and
the real and variable values are the same for the stacks and every
other element.
\SemanticsMODELING
\SemanticsModRho

They all have a similar structure expressed by the \AFun{MODELING}
function as they relate an abstract thing with a concrete thing. They
are implemented by several auxiliary \verb/mod/$X$ predicates for every
subcomponent of program and execution states. For example, \AFun{ModE}
relates execution environments, \AFun{modprg} relates shadow programs,
\AFun{modS} relates stacks, and \AFun{modΦ} checks that the formulas
are all true. The definition of \AFun{modσ} is similar.

To show that a disjunction of abstract states models a concrete state,
we show that one of the states in the disjunction models the state:
\SemanticsModUS

% \begin{listing}[!ht]
% \begin{minted}{agda}
% mod⊎σ : ∀ {Γ} → Int Γ → ⊎Exec-state → Exec-state → Set
% mod⊎σ {Γ} γ ⊎σ σ = ∃[ ασ ] (Γ , ασ) ∈ ⊎σ × modσ γ ασ σ
% \end{minted}
% \caption{Modeling disjunction of states, e.g., for execution states}
% \label{modUsigma}
% \end{listing}

% \draft{maybe also show how formulas of states are modelled? \ldots or maybe i don't ;p}

\subsection{Soundness of the DL}\label{soundness}

We prove the soundness of the logic by showing that
when an abstract state models a concrete one,
the result of one-step symbolic execution models the result from
concrete execution of the same step.
Here are the types of the proof terms for program steps and execution
steps.
\SoundnessProgStep
\SoundnessExecStep


%% listing ruler max width ------------------------------------------------|?X
% \begin{listing}[!ht]
% \begin{minted}{agda}
% soundness : ∀ {Γ ro so} γ αρ ρ → modρ {Γ} {ro} {so} γ αρ ρ
%           → ∃[ Γ` ] ∃[ γ` ]
%             mod⊎ρ {Γ` ++ Γ} (γ` +I+ γ) (αprog-step αρ) (prog-step ρ)

% soundness : ∀ {Γ : Context} γ ασ σ → modσ {Γ} γ ασ σ
%           → Exec-state.MPstate σ ≡ nothing
%           ⊎ ∃[ Γ` ] ∃[ γ` ]
%             mod⊎σ {Γ` ++ Γ} (γ` +I+ γ) (αexec-step ασ) (exec-step σ)

% soundness : ∀ {Γ ro so γ αρ Γ` αρ`}
%           → αρ-special αρ (Γ` , αρ`)
%           → (ρ : Prog-state {ro} {so})
%           → modρ {Γ} {ro} {so} γ αρ ρ
%           → ∃[ γ` ] modρ (γ` +I+ γ) αρ` (prog-step ρ)

% soundness : ∀ {Γ γ ασ ⊎σ`}
%           → ασ-special ασ ⊎σ`
%           → (σ : Exec-state)
%           → modσ {Γ} γ ασ σ
%           → ∃[ Γ` ] ∃[ γ` ] mod⊎σ {Γ` ++ Γ} (γ` +I+ γ) ⊎σ` (exec-step σ)
% \end{minted}
% \caption{Soundness terms}
% \label{soundness}
% \end{listing}

\begin{figure}[tp]
\SoundnessCaseIfNone
  \caption{Prog-step soundness for IF-NONE (excerpt)}
  \label{fig:prog-step-soundness-if-none}
\end{figure}

% XXX =\module{23-prog-step-soundness}
The first \AFun{soundness} statement addresses soundness of \verb/αprog-step/.
As the modeling relation is mostly composed of equalities, the proof
gets accepted by Agda, once we supply sufficiently precise arguments to match the cases in the definition
of \AFun{αprog-step}. We pattern match against \ACon{refl} and parts of the arguments,
as well as we show that the weakened parts of the formula
are still modeled with the extended context (if new variables were
introduced in the case).

\autoref{fig:prog-step-soundness-if-none} shows the case for
the \ACon{IF-NONE} instruction. Without going into details, it is easy
to spot the handling of the concrete and abstract stack and that the
outcome of the test determines which of the possibilities of the
abstract outcome is chosen (cf. $0\in$ and $1\in$).

%% listing ruler max width ------------------------------------------------|?X
% \begin{listing}[!ht]
% \begin{minted}{agda}
% soundness γ (αstate αen (IF-NONE thn els ; prg) (o∈ ∷ rVM) sVM Φ)
%              (state en .(IF-NONE thn els ; prg) (just x ∷ rSI) sSI)
%              (refl , refl , mE , refl , (o≡ , mrS) , msS , mΦ)
%   = _ , x ∷ [I] , _ , 1∈ , (refl , refl , wkmodE mE , refl ,
%     (refl , wkmodS mrS) , wkmodS msS , (o≡ , wkmodΦ mΦ))
% \end{minted}
% \caption{Prog-step soundness example for IF-NONE}
% \label{ps-sound-IF-NONE}
% \end{listing}
\begin{figure}[tp]
  \SoundnessCaseDOne
  \caption{Prog-step soundness for scalar functions (excerpt)}
  \label{fig:ps-sound-D1}
\end{figure}

The most complicated case of this proof establishes soundness for any
scalar function (see \autoref{fig:ps-sound-D1}).
It works by showing that applying the front of the previous
stack interpretation to the given function yields the same result as applying
the extended interpretation of the top of the previous stack matching to it.

%% listing ruler max width ------------------------------------------------|?X
% \begin{listing}[!ht]
% \begin{minted}{agda}
% soundness γ (αstate αen (fct (D1 x) ; prg) rVM sVM Φ)
%              (state en .(fct (D1 x) ; prg) rSI sSI)
%              (refl , refl , mE , refl , mrS , msS , mΦ)
%   with modS++ rVM rSI mrS
% ... | mtop , mbot = _ , appD1 x (Itop rSI) ∷ [I] , _ , 0∈ , (refl , refl ,
%     wkmodE mE , refl , (refl , wkmodS mbot) , wkmodS msS ,
%     cong (appD1 x) (trans (sym (modIMI mtop))
%                           (wkIMI {γ` = appD1 x (Itop rSI) ∷ [I]})) ,
%     wkmodΦ mΦ)
% \end{minted}
% \caption{Prog-step soundness of onedimensional functions}
% \label{ps-sound-D1}
% \end{listing}

% XXX =\module{25-exec-step-soundnes}s
The second \AFun{soundness} statement establishes soundness for those cases of \AFun{αexec-step}
where a contract execution is active.
This part appears simple because it only covers two cases:
Either we are in the middle of running a contract, in which case we
reuse the soundness proof for program state execution, or
the current contract execution has terminated and we have to prove that
the blockchain and the pending list are updated correctly.
The first case is straightforward, but tedious because we need to copy
parts of the previous proof.
The second case is fairly technical as it involves getting the proof in
sync with the definitions of concrete and abstract execution.

% Unfortunately we still have to supply Agda with all relevant details about the program state
% so it can instanciate the result from that proof, but the proof term is always exactly the
% same except for the context extension of that case,
% as well as which clause of the disjunction models the result, which is sometimes the second clause.
% \listref{exec2progstep-soundness} gives an example: To be able to instanciate \verb/px/ 
% in lines 13 and 22 with \verb/refl/,
% we show that extending \verb/Γ/ with the correct context extension for the
% particular case is equal to \verb/Γ`/.

%% listing ruler max width ------------------------------------------------|?X
% \begin{listing}[!ht]
% \begin{minted}[linenos]{agda}
% soundness {Γ} γ 
%   ασ@(αexc αccounts 
%           (inj₁ (αpr {s = s} αcurrent αsender
%                      αρ@(αstate αen (IF-NONE thn els ; prg) 
%                                     (v∈ ∷ rVM) sVM Φ)))
%           αpending)
%   σ@(exc accounts (just (pr current sender ρ)) pending)
%   ( mβ
%   , ( refl , refl , refl , refl , mc , ms
%     , mρ@(refl , refl , mE , refl , mρrest))
%   , mp)
%   with ρ-sound γ αρ ρ mρ
% ... | Γ` , γ` , _ , here px , mρ`
%   with ++-cancelʳ Γ Γ` [] (,-injectiveˡ px)
% soundness γ ασ σ (mβ , ( refl , refl , refl , refl , mc , ms , mρ) , mp)
%   | _ , γ` , _ , 0∈ , mρ` | refl
%   = inj₂ ( _ , γ` , _ , 0∈ , (wkmodβ mβ) , (refl , refl , refl , refl
%          , wkmodC {γ` = γ`} mc , (wkmodC {γ` = γ`} ms) , mρ`) , wkmodp mp)
% soundness {Γ} γ ασ@(αexc αccounts (inj₁ (αpr {s = s} αcurrent αsender
%    αρ@(αstate αen (IF-NONE {ty} thn els ; prg) (v∈ ∷ rVM) sVM Φ))) αpending)
%   σ (mβ , ( refl , refl , refl , refl , mc , ms , mρ) , mp)
%   | Γ` , γ` , _ , there (here px) , mρ`
%   with ++-cancelʳ Γ Γ` [ ty ] (,-injectiveˡ px)
% soundness γ ασ σ (mβ , ( refl , refl , refl , refl , mc , ms , mρ) , mp)
%   | _ , γ` , _ , 1∈ , mρ` | refl
%   = inj₂ ( _ , γ` , _ , 1∈ , (wkmodβ mβ) , (refl , refl , refl , refl
%          , wkmodC {γ` = γ`} mc , (wkmodC {γ` = γ`} ms) , mρ`) , wkmodp mp)
% \end{minted}
% \caption{Proofing exec step soundness with prog step soundness}
% \label{exec2progstep-soundness}
% \end{listing}

\begin{comment}
  The proof of the special cases for program state transitions is
  similar in that almost every case has the same structure: Extracting
  the conjunct from \verb/Φ/ that is used to apply the special case
  transition and rewriting by it makes the soundness obvious and only
  \verb/refl/ pracements necessary.  Only the cases for the
  \verb/CONTRACT/ instruction need a special rewrite technique to
  match the concrete and abstract blockchain lookups, as well as some
  handling of cases that are actually impossible for the given
  transition:

  %% listing ruler max width
  %% ------------------------------------------------|?X
  \begin{listing}[!ht]
\begin{minted}{agda}
soundness (UNPAIR φ∈Φ) (state en _ (p ∷ rSI) sSI)
          (refl , refl , mE , refl , (refl , mrS) , msS , mΦ)
  with modφ∈Φ φ∈Φ mΦ
... | p∈≡p rewrite p∈≡p
  = _ , refl , refl , mE , refl , (refl , refl , mrS) , msS , mΦ

soundness (CTR¬p {p' = p'} φ∈Φ ≡j p≢p') (state en _ (a ∷ rSI) sSI)
          (refl , refl , mE@(mβ , mErest) , refl , (refl , mrS) , msS , mΦ)
  with modφ∈Φ φ∈Φ mΦ | mβ a
... | refl | mMCa
  rewrite ≡j
  with Environment.accounts en a | mMCa
... | just (p , s , c) | refl , mpC
  with p ≟ p'
... | yes refl = ⊥-elim (p≢p' refl)
... | no x
  = (nothing ∷ [I]) , refl , refl , wkmodE mE , refl
  , (refl , wkmodS mrS) , wkmodS msS , (refl , wkmodΦ mΦ)
\end{minted}
\caption{Soundness of special prog state transitions}
\label{prog-step-SC}
\end{listing}

The special program state transitions can also occur during contract
chain executions and make up one of the special cases for execution
state transitions.  As with prog state execution within execution
state execution, these special cases are proven by reusing their
soundness proof for program state transitions.  The remaining cases
are concerned with termination of a contract execution or the handling
of pending operations and require a combination of the techniques used
for special program state transitions on a larger scale.

\end{comment}
%%% Local Variables:
%%% mode: latex
%%% TeX-master: "itp2024"
%%% End:

\section{Related work}
\label{sec:related-work}

Research on formal verification of blockchain-based applications
has experienced rapid growth in the last decade. Various techniques
and frameworks have been applied to enhance the safety of smart
contracts. In this section, we 
discuss some key approaches, particularly those employing symbolic
execution in the context of smart contracts. 
 
Symbolic execution is a powerful technique for
systematically exploring program paths and identifying potential
vulnerabilities in smart contracts. Most of the existing tools focus
on the Ethereum platform. Tsankov et al. introduced
SECURIFY \cite{securify}, a tool that utilizes symbolic execution to
perform practical security analysis on Ethereum smart contracts. It
targets common vulnerability security patterns specified in a
designated domain-specific language. SECURIFY symbolically encodes the
dependence graph of the contract in stratified Datalog to extract
semantic information from the code. After obtaining semantic facts, it
checks whether the security patterns hold or not. Similarly, Manticore
\cite{manticore} and KEVM \cite{kevm} use symbolic execution to
analyze Ethereum smart contracts. KEVM is an executable formal
specification built with the K Framework for the Ethereum virtual
machine's bytecode (EVM), a stack-based and low-level smart contract
language for the Ethereum blockchain. Since tokens can hold a
significant amount of value, they are often targeted for
attacks. Therefore, several tools \cite{kevm,park} conduct case
studies for the implementations of token standards. 

Several approaches use existing formal verification frameworks to
ensure the correctness and security of smart contracts. Amani et
al. \cite{isabelle} proposed the formal verification of Ethereum smart
contracts in Isabelle/HOL. Hirai \cite{hirai} formalizes the EVM using
Lem, a language to specify semantic definitions. The formal
verification of smart contracts is achieved using the Isabelle proof
assistant. 
Mi-cho-Coq~\cite{mi-cho-coq} is a framework for the proof assistant Coq to verify
functional correctness of Michelson smart contracts. They formalize
the semantics of a Michelson in Coq using a weakest precondition
calculus and verify several contracts.
It provides full coverage of the language whereas our goal is to give
a blueprint for a soundness proof of symbolic execution.

There are several tools for automated verification including
solc-verify \cite{solc}, VerX \cite{verx}, and Oyente
\cite{oyente}. solc-verify processes smart contracts written in
Solidity and discharges verification conditions using modular program
analysis and SMT solving. It operates at the level of the contract
source code, with properties specified as contract invariants and
function pre- and post-conditions provided as annotations in the code
by the developer. This approach offers a scalable, automated, and
user-friendly formal verification solution for Solidity smart
contracts. The core of solc-verify involves translating Solidity
contracts to Boogie IVL (Intermediate Verification Language), a
language designed for verification.  

Nishida et al. \cite{helmholtz} developed HELMHOLTZ, an automated
verification tool for Michelson. While both research efforts aim to
build a verification tool for smart contracts written in Michelson,
HELMHOLTZ is based on  refinement types, whereas we consider symbolic
execution.
HELMHOLTZ has better coverage of Michelson instructions than we
currently have, but it can only verify a single contract whereas our
model and soundness proof covers full inter-contract verification.
The HELMHOLTZ developers plan to extend Helmholtz with inter-contract behavior.

% iContract targets the Solidity language. It also utilizes pre and
% post conditions, similar to our tool, to specify user properties. It
% locally installs the Solidity compiler to compile a user-provided
% Solidity file into a JSON file containing the typed abstract syntax
% tree (AST). Then, iContract analyzes the AST to encode contracts
% into predicates using the Z3 library. They leverage the NatSpec
% format to define their own specifications. 


% A lot of different tools have been developed to aid programmers in verifying smart contracts.
% Most of them focus on the \emph{Ethereum} platform and many employ symbolic execution tools,
% like Manticore~\cite{manticore}, Oyente~\cite{oyente}, or Securify~\cite{securify}.

Bau et al.~\cite{abstract-interpretation} implement a static analyzer for Michelson
within the modular static analyser MOPSA that is based on abstract interpretation.
It is able to infer invariants on a contract's storage over several calls
and it can prove the absence of errors at run time.

Da Horta et al.~\cite{why3} aim at automating as much of the verification process as possible
by automatically translating a Michelson contract into an equivalent program
for the deductive program verification platform WHY3.
However, they found that sometimes user intervention was required,
and their tool can only verify single contracts individually.

% The Helmholtz verifier~\cite{helmholtz} uses refinement types to prove
% user-defined specifications of Michelson programs with the SMT solver Z3.


%%% from Abstract Interpretation of Michelson Smart-Contracts %%%%%%%%%
% The Micse project [6] allows for automated
% static analysis, using the Z3 SMT solver. The Tezla [30]
% project allows translating the Michelson instructions into a
% suitable intermediate representation for dataflow analysis.

\section{Conclusion}

We presented a dynamic logic for Michelson as well as its extension for blockchain operations
on a small but representative subset of Michelson.
The goal was to create a core model that covers instances of all kinds
of operations and that can be easily extended with further Michelson instructions.
We achieved full coverage of scalar functional instructions, the majority of Michelson
instructions.
To include any further scalar instruction,
one only has to specify its typing rule and its implementation in Agda.
The symbolic execution rule and the soundness proof for that rule is
already provided by our model.
Further instructions that retrieve information from the execution environment can be added
easily as well by extending the \ADT{Environment} record and its subcomponents
to include such information.

We cover three exemplary instructions for control flow,
because most other conditional and looping instructions
are either very similar or very simple and thus easy to include in the presented model.
One aspect of Michelson that is not covered is first-class functions.
Including them might require some reworking of the current model to store such values on the stack.

Efficient symbolic execution is \textbf{not} a goal of this work:
Agda can normalize a symbolic execution state to present the current values after the execution 
steps, but it will internally always save the state as an application of every preceeding
symbolic execution step.
Since these states involve many weakenings of all subcomponents,
the computational load becomes excessive after
less than ten symbolic execution steps.
We plan to use our soundness proof as the basis for an efficient
symbolic interpreter for Michelson in ongoing work.


%%% Local Variables:
%%% mode: latex
%%% TeX-master: "itp2024"
%%% End:


\bibliography{ecoop2024,bib/refs}
\end{document}
