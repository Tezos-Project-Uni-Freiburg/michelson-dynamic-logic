\section{Dynamic Logic for Michelson}
\label{sec:DL}

\subsection{Terms and Formulas}\label{sec:terms-formulas}

Variables are defined as elements of a given \verb/Context = List Type/,
which is represented in Agda the same way as \verb/Stack/'s are, but with a different purpose.

The datatype \verb/data Match (Γ : Context) : Stack → Set/ represents a subset of the
current context that matches the types of a given \verb/Stack/ and is also used
to represent the Michelson execution stacks during symbolic execution.
It is implemented in the same way as the stack interpretation \verb/Int/ with functions
for concatenation and to retrieve the top and bottom of it when indexed by a list concatenation.

\listref{terms-formulas} shows the terms and formulas used for the logic.
We only allow constant terms for primitive types and contracts.
The onedimensional functions from \secref{sec:instructions} are used to construct
terms representing these functions, where the argument terms are restricted to being variables.
The term for subtracting mutez' is not representable with a functional Michelson instruction
and therefore defined as a separate term.

Formulas are restricted mostly to express equality of a variable with a term.

%% listing ruler max width ------------------------------------------------|?X
\begin{listing}[!ht]
\begin{minted}{agda}
data _⊢_ (Γ : Context)    : Type → Set where
  const : ∀ {bt}          → ⟦ base bt ⟧               → Γ ⊢ base bt
  func  : ∀ {args result} → 1-func args result
                          → Match Γ args              → Γ ⊢ result
  var   : ∀ {ty}          → ty ∈ Γ                    → Γ ⊢ ty
  contr : ∀ {ty P}        → ⟦ contract {ty} P ⟧       → Γ ⊢ contract P
  _∸ₘ_  : base mutez ∈ Γ  → base mutez ∈ Γ            → Γ ⊢ base mutez

data Formula (Γ : Context) : Set where
  `false : Formula Γ
  _:=_   : ∀ {ty} → ty ∈ Γ → Γ ⊢ ty   →  Formula Γ
  _<ₘ_   : base mutez ∈ Γ → base mutez ∈ Γ   →  Formula Γ
  _≥ₘ_   : base mutez ∈ Γ → base mutez ∈ Γ   →  Formula Γ
\end{minted}
\caption{Typed terms and Formulas for a Context Γ}
\label{terms-formulas}
\end{listing}

\subsection{Representing Michelson Programs in DL}\label{sec:abstract-states}

The way in which programs are included in the syntax of a Dynamic Logic is
by prexifing formulas with the modal operators $[p]$ or $\langle p\rangle$
for a program or program fragment $p$.
A formula $\langle p\rangle \phi$ means that $p$ will terminate in a state where $\phi$ holds,
and the goal is to be able to proof formulas like
$Pre \Longrightarrow \langle p \rangle Ass$ to know that when the given preconditions $Pre$ hold,
program $p$ will terminate and the assertions $Ass$ are guaranteed.

The main focus of this thesis is the developement of symbolic execution rules that are used to
reduce the programs in such formulas successively until one is left
with a purely first formula that does not include any more program statements.
We only give the formalization of these symbolic execution rules in Agda
since the resulting first order formulas do not contain Michelson specific aspects
and can be easily formalized like any other first order logic.
See \cite{KeY2,DL} for more information on that topic.

Analogous to the two different execution models for Michelson,
one only for programs and one for contract execution chains,
the symbolic execution rules are given for programs and contract execution chains as well.
In case of contract execution chains, this formalizes a dynamic logic that extends the DL
about Michelson programs to one about blockchain operations.

For a DL in general the preconditions can be empty,
but our formalization in Agda will always assume some preconditions,
which makes sense in both cases:
Even when reasoning about a program that does not consume any stack elements
(e.g. proving something about the values the program pushed onto the stack),
we can only ever prove something about the resulting values of stack elements,
and to reference their position on the stack we would assume that the initial stack is empty.
% This formalization generalizes to be correct for any unaffected substack.
% For Contract executions, they terminate when all pending operations have been accounted for,
To save the results from modeling a contract execution, it must at least be assumed
that the contract under consideration is present in the blockchain.
So both formalizations always include at least some preconditions on the input stacks
or the state of the blockchain.

\begin{comment}
\draft{And when we're perfectly honest (hence this is just a draft ;p),
what we have developed is only a precursor to a full Dynamic Logic
(i'll hopefully find a formulation somewhat like above with
``the main focus \ldots therefore here we only give \ldots'').
We give a calculus for the following parts:}
\end{comment}

Most rules of the presented Michelson DL calculus have the form:
\begin{align*}
			           Pre  &\Longrightarrow \langle i ; p \rangle Ass
\\ \leftrightarrow \qquad I \land  Pre' &\Longrightarrow \langle     p \rangle Ass
\end{align*}

where $I$ is a new clause containing new variables and variables from $Pre$,
and $Pre'$ only differs from $Pre$ in clauses about the state of the stacks.
For control flow instructions the rules may contain disjunctions and subprogram expansions:
\begin{align*}
			           Pre  &\Longrightarrow \langle i    ; p \rangle Ass
\\ \leftrightarrow \qquad (I \land Pre'  \Longrightarrow \langle p'  ;; p \rangle Ass)
		   &\lor  (J \land Pre'' \Longrightarrow \langle p'' ;; p \rangle Ass)
\end{align*}

Notice that $Ass$ is never affected.
This is unlike the calculus of Dynamic Logics for other programming languages 
where an assignment might affect the value of a variable contained in $Ass$.
But Michelson does not have any variables and therefore no assignments to them,
so we are able to neglect $Ass$ when formalizing the symbolic execution rules.
So our formalization will specify how we get from $Pre$ to $I \land Pre'$
when symbolically executing instruction $i$, where $Pre$ is a conjunction of formulas 
that will always include statements about the program stacks
as well as the execution environment (which is necessary to execute \verb/env-func/ instructions).
That is when we only deal with symbolic program execution; in case of contract execution chains,
$Pre$ will always contain statements about the state of the blockchain, as well as some additional
information in cases where a currently running contract execution is formalized.

\begin{comment}
These rules never affect the goal formula to be asserted, and because
(which must always be given), and $Ass$ can be any formula.
Our construct specifys $Pre \Longrightarrow \langle p \rangle$ but not $Ass$ since it
neither changes, nor has any implications for the validity of our rules
\draft{other than the premisse holds with $Ass$ \emph{iff} the conclusion holds with it (i guess)}.

This matches the sequent rule (R8) from \todo{cite lecture notes 2010}
for functional Michelson instructions, since here $x$ will be the new top stack element,
which our DL always represents with a new variable, thus $\phi_x^{\hat{x}} \equiv \phi$.
\todo{not sure about control flow instructions \ldots}

We will proof the soundness of these rules by showing that
if an interpretation for $Pre$ is a valid interpretation for a concrete execution state
with $i ; p$ left to be executed
(i.e. stack values match and $Pre$ otherwise doesn't contradict itself),
we can give an extension of that interpretation
such that $I \land Pre'$ is a valid interpretation of the concrete state after execution of $i$.

Since the extension of the interpretation is only for possible new variables that are not present
in $Ass$, it follows from our soundness proof \draft{that our calculus is sound \ldots}.
\end{comment}

This makes the formalization of symbolic execution very similar to the concrete execution model
presented in \chapref{chap:refImpl}.
This is by design to make the soundness proof easier and more concise.
All constructs for concrete execution have an abstract counterpart.
They are always paremeterized by a \verb/Context Γ/ and the constructs names are the same as for
concrete execution but prefixed with an \verb/α/
(only the abstract blockchain is called \verb/βlockchain/).

The abstract program state from \listref{aProg-state}
represents $Pre \Longrightarrow \langle p \rangle Ass$ for an arbitrary assertion where
\begin{align*}
	Pre	&\equiv	\text{\emph{state of environment}} = \alpha en
\\		&\land	\text{\emph{state of real stack}} = rVM
\\		&\land	\text{\emph{state of shadow stack}} = sVM
\\		&\land	\bigwedge_{\phi \in \Phi} \phi
\\	p	&\equiv	prg
\end{align*}

\begin{listing}[!ht]
\begin{minted}{agda}
record αProg-state Γ {ro so : Stack} : Set where
  constructor αstate
  field
    {ri si} : Stack
    αen : αEnvironment Γ
    prg : ShadowProg ri si ro so
    rVM : Match Γ ri
    sVM : Match Γ si
    Φ   : List (Formula Γ)

⊎Prog-state = λ {ro} {so} → List (∃[ Γ ] αProg-state Γ {ro} {so})
\end{minted}
\caption{Abstract program state}
\label{aProg-state}
\end{listing}

Symbolic execution can lead to disjunctions over such states, which is presented using
a list of abstract program states.
% \mint{agda}|⊎Prog-state = λ {ro} {so} → List (∃[ Γ ] αProg-state Γ {ro} {so})|

Using Agda Lists to represent conjunctions and disjunctions is conveniet because
\begin{enumerate}
	\item	They do not mix: $\Phi$ always represents a conjunction over its elements
		and disjunctions can only occur as a result of some symbolic execution rules
		where it always affects every aspect of the abstract program state
		(e.g. the remaining programs will always differ).
	\item	Agda comes with a very powerful element relation for its list.
		This makes implementation of some rules of the calculus
		as well as the semantics for our model very efficient.
\end{enumerate}

\subsection{Calculus for Michelson DL}\label{sec:calculus}

The rules for symbolic execution are formalized in two ways presented in \listref{aprog-all}.
%% listing ruler max width ------------------------------------------------|?X
\begin{listing}[!ht]
\begin{minted}{agda}
αprog-step : ∀ {Γ ro so} → αProg-state Γ {ro} {so} → ⊎Prog-state {ro} {so}

data αρ-special {Γ ro so} :         αProg-state        Γ  {ro} {so}
                          → ∃[ Γ` ] αProg-state (Γ` ++ Γ) {ro} {so}
                          → Set where
\end{minted}
\caption{Symbolic execution rules for abstract program states}
\label{aprog-all}
\end{listing}

The first one mimics \verb/prog-step/ and gives a deterministic way of symbolic execution.
The second one is for special cases where the precondition contains additional information
about the abstract state
that makes it possible to deduce a more concise formula than \verb/αprog-step/ would.
For example, when symbolically executing \verb/ITER'/, \verb/αprog-step/ would produce
a true disjunction, but when $\Phi$ contains information about the top element of the
shadow stack, a disjunction can be avoided.

For concrete execution, every (non-environmental) functional instruction could be executed
with a single rule as demonstrated in \listref{prog-step}.
During symbolic execution, the only thing that is guaranteed is
that the stacks contain some values of certain given type.
Other than that, nothing more can be assumed of these values.
If for example the next instruction is \verb/ADDnn/, we can conclude that the top of the stack
will be some value of type \verb/base nat/, and in this case we can say a little bit more about
it, particularly that it is the sum of the two values that were on top of the stack before.

This is achieved by introducing a new variable $v_n$ that replaces the
variables $v_x, v_y$ from the top of the stack,
and adding a clause where this new variable is set to be equal to the sum of the former two,
which is expressed by the formula
\[	v_n := ADD_{nn}\; v_x\; v_y
\]

We cannot give a concise symbolic execution rule for all functional instructions,
but it is possible for the subset of onedimensional functions since their
function types are also used to define most terms.
This is shown in \listref{aprog-step-func} alongside some exemplary rules for other functional
instructions.

%% listing ruler max width ------------------------------------------------|?X
\begin{listing}[!ht]
\begin{minted}{agda}
αprog-step {Γ} (αstate αen (fct (D1 {result} f) ; prg) rVM sVM Φ)
  = [ [ result // Γ ]
    , (αstate (wkαE αen) prg (0∈ ∷ wkM (Mbot rVM)) (wkM sVM)
              [ 0∈ := wk⊢ (func f (Mtop rVM)) // wkΦ Φ ]) ]

αprog-step {Γ} (αstate αen (fct (Dm (UNPAIR {t1} {t2})) ; prg)
                       (p∈ ∷ rVM) sVM Φ)
  = [ [ t1 / t2 // Γ ]
    , (αstate (wkαE αen) prg (0∈ ∷ 1∈ ∷ wkM rVM) (wkM sVM)
              [ 0∈ := wk⊢ (func CAR (p∈ ∷ [M]))
              / 1∈ := wk⊢ (func CDR (p∈ ∷ [M])) // wkΦ Φ ]) ]

αprog-step α@(αstate αen (fct (Dm SWAP) ; prg) (x∈ ∷ y∈ ∷ rVM) sVM Φ)
  = [ _ , record α{ prg = prg ; rVM = y∈ ∷ x∈ ∷ rVM } ]

αprog-step {Γ} (αstate αen (fct (PUSH P x) ; prg) rVM sVM Φ)
  = [ (expandΓ P x ++ Γ)
    , (αstate (wkαE αen) prg ((∈wk (0∈exΓ P)) ∷ wkM rVM) (wkM sVM)
              (Φwk (unfold P x) ++ wkΦ Φ)) ]
\end{minted}
\caption{Deterministic symbolic execution of abstract program states for functional instructions}
\label{aprog-step-func}
\end{listing}

The matching representing the stack can be split implicitly
just like the stack interpretation during concrete execution.
\verb/n∈/ is another Agda pattern synonym for the n'th element of the context.
Whenever the context is extended during symbolic execution, all elements of the abstract
program state that do not change must be weakened so they can be parameterized by the new context.

For the other functional instructions, no general mechanism can be given,
since they behave very differently in a symbolic context:
\verb/UNPAIR/ requires two new variables and clauses, while
\verb/SWAP/ only changes the position of two stack values, that is two variables,
so the entire change is encoded in the state of the stack
and no new clauses or variables are necessary.

\verb/PUSH/ needs special treatment because it is a legal Michelson instruction
for any, arbitrarily complex compound type.
% \verb/PUSH P x/ is a special case for the abstract execution as mentioned before:
When pushing a value \verb/x/ of a primitive type, it is enough to add a new variable
and a clause which sets this variable equal to the term \verb/const x/.
But when \verb/x/ is a compound type, its value cannot be expressed with a \verb/const/ term
since this is restricted to primitive types (see \listref{terms-formulas}).
So \verb/unfold P x/ will create all clauses necessary
to express the value \verb/x/, using a list of new variables \verb/expandΓ P x/ and the function
types \verb/CONS/, \verb/PAIR/ etc for constructing compound types.
For example \verb/PUSH {list ty} P (y ∷ ys)/ would be decomposed into two new variables of type
\verb/ty/ for \verb/y/ and \verb/list ty/ for \verb/ys/.
If \verb/ys = []/, its variable can be set to \verb/func (NIL ty) [M]/,
otherwise it will be further decomposed.
Similarly for \verb/y/: If \verb/ty/ is a primitive type, it can be set to \verb/const y/,
otherwise it must be further decomposed as well.
% This process is arbitrarily complex, but of course finite ;)

\listref{ap-special-func} shows two special case examples for functional instructions:

%% listing ruler max width ------------------------------------------------|?X
\begin{listing}[!ht]
\begin{minted}{agda}
  UNPAIR : ∀ {αen ty₁ ty₂ v₁∈ v₂∈ S si p∈ Φ prg rVM sVM}
         → p∈ := func (PAIR {ty₁} {ty₂}) (v₁∈ ∷ v₂∈ ∷ [M])  ∈  Φ
         → αρ-special 
                (αstate {si = si} αen (fct {S = S} (Dm UNPAIR) ; prg)
                        (p∈ ∷ rVM) sVM Φ)
           ([] , αstate αen prg (v₁∈ ∷ v₂∈ ∷ rVM) sVM Φ)

  app-bf : ∀ {αen args result S si prg rVM sVM Φ} {Margs : Match Γ args}
         → {bf : 1-func args (base result)}
         → (MCargs : MatchConst Φ Margs)
         → αρ-special
            (αstate {si = si} αen (fct {S = S} (D1 bf) ; prg)
                    (Margs +M+ rVM) sVM Φ)
           ( [ base result ]
           , αstate (wkαE αen) prg (0∈ ∷ (wkM rVM)) (wkM sVM) 
                    [ 0∈ := const (appD1 bf (getInt MCargs)) // wkΦ Φ ] )
\end{minted}
\caption{Symbolic execution of functional instructions with additional information}
\label{ap-special-func}
\end{listing}

When \verb/UNPAIR/ is executed on a variable \verb/p∈/,
and it is known about \verb/p∈/ that it is a \verb/PAIR/
of two other variables \verb/v₁∈/ and \verb/v₂∈/,
there is no need to introduce two new variables and express their values with respect to \verb/p∈/
as the generic \verb/αprog-step/ would (see \listref{aprog-step-func}).
These new variables would be equal to \verb/v₁∈/ and \verb/v₂∈/
and it is better to use them instead.

Also, when the argument values for a onedimensional function are all given as constant terms
(which is formalized by the \verb/MatchConst Φ Margs/ construct that associates every argument
variable \verb/v∈/ from \verb/Margs/ with a clause \verb/v∈ := const x/ in \verb/Φ/),
it is not necessary to express the value of the result in a functional term
since it can just be calculated.

\listref{a-ITER'} shows some symbolic execution rules for the \verb/ITER'/ instruction:
Generally, when no further information about the relevant list variable is present,
symbolic execution will lead to a disjunction that considers both possibilities.
But if some relevant information is known about that variable,
the disjunction as well as the context extension can be avoided.

\begin{listing}[!ht]
\begin{minted}{agda}
αprog-step {Γ} α@(αstate αen (ITER' {ty} x ∙ prg) rVM (l∈ ∷ sVM) Φ)
  = [ _ , record α{ prg = prg ; sVM = sVM 
                  ;   Φ = [ l∈ := func (NIL ty) [M] // Φ ] }
    / [ ty / list ty // Γ ]
    , αstate (wkαE αen) (x ;∙ ITER' x ∙ prg) (0∈ ∷ wkM rVM) (1∈ ∷ wkM sVM)
             [ 2+ l∈ := func CONS (0∈ ∷ 1∈ ∷ [M]) // wkΦ Φ ] ]

  ITER'c : ∀ {αen ty l∈ x∈ xs∈ Φ rS sS iterate prg rVM sVM}
         → l∈ := func CONS (x∈ ∷ xs∈ ∷ [M])  ∈  Φ
         → αρ-special
               (αstate αen (ITER' {ty} {rS} {sS} x ∙ prg) rVM  (l∈ ∷ sVM) Φ)
           (_ , αstate αen (x ;∙ ITER' x ∙ prg)    (x∈ ∷ rVM) (xs∈ ∷ sVM) Φ)
\end{minted}
\caption{Exemplary symbolic execution rules for ITER'}
\label{a-ITER'}
\end{listing}

It is possible to reach the same logical conclusions as \verb/αρ-special/
by first performing a generic symbolic execution step with \verb/αprog-step/
and then applying some first order conclusions on the resulting formula.
For example in the case of \verb/ITER'/,
if it is known that \verb/l∈/ is \verb/CONS/
and the first disjunct from \verb/αprog-step/ adds the clause that sets it equal to \verb/NIL/,
a contradiction can be concluded from these two clauses and the entire disjunct
can be set to false and discarded.

Such rules and their soundness have already been successfully implemented and proven
in an earlier version of the model.
But since they come at a great performance cost by requiring more conclusion steps
and creating new redundant variables most of the time,
these rules where replaced by the given special relation transitions
during a major rework of the model.

\subsection{Calculus for the DL on blockchain operations}

Just like the symbolic execution rules for the Michelson DL,
those for the DL on blockchain operations (shown in \listref{aexec-all})
are also given in a generic way as well as with a transition relation for special cases.

%% listing ruler max width ------------------------------------------------|?X
\begin{listing}[!ht]
\begin{minted}{agda}
record αExec-state Γ : Set where
  constructor αexc
  field
    αccounts : βlockchain Γ
    αρ⊎Φ     : αPrg-running Γ ⊎ List (Formula Γ)
    pending  : List (list ops ∈ Γ × ⟦ base addr ⟧)

⊎Exec-state = List (∃[ Γ ] αExec-state Γ)

αexec-step : ∀ {Γ} → αExec-state Γ → ⊎Exec-state

data ασ-special {Γ} : αExec-state Γ → ⊎Exec-state → Set where
\end{minted}
\caption{Symbolic execution rules for abstract execution states}
\label{aexec-all}
\end{listing}

The abstract execution state needs a field to store its list of formulas \verb/Φ/.
When it represents a state where a contract is currently under execution,
\verb/αρ⊎Φ/ contains \verb/αPrg-running/ which in turn contains an abstract program state
that has a field \verb/Φ/ for this purpose.
If it represents a state where this is not the case, \verb/αρ⊎Φ/ contains \verb/Φ/ instead.

Unfortunately \verb/αexec-step/ cannot exactly represent \verb/exec-step/:
When checking \verb/pending/ for further operations to be executed,
without further knowledge about the variables representing these operations,
no meaningful action can be performed to advance the symbolic execution process.
To execute any succeeding operation, it must be at least given that the first
\verb/list ops ∈ Γ/ variable in \verb/pending/ is \verb/CONS/ of such an operation variable,
which in turn must also be given to have the value of a \verb/TRANSFERE-TOKENS/ term,
for which the \verb/contract/ variable must be given a value as well.
This additional information cannot be considered for the generic case,
and because of that we only prove the soundness of \verb/αexec-step/ for those cases that
are not concerned with executing pending operations, that is only those cases where a
contract is currently under execution.

These cases are either a terminated contract execution where the results are written back
to the blockchain, or execution of an abstract program step for the contract under execution.
The first case is similar to the concrete implementation where new variables are introduced
for the updated values.
The second case is more complicated because the context extensions from the abstract program step
are encoded in the list of resulting disjunctions,
so an additional term has to be supplied proving that these contexts are actually an extension
of the original context.

The special transition relation for abstract execution states
contains all special transitions for abstract program states,
ways to symbolically execute the blockchain update that take additional information about
the final stack into account,
as well as many different special cases that reflect some of the possible cases
when executing the list of pending operations.
It relates the execution states to a disjunction of execution states to reflect the possibility
that all necessary information to execute the next contract is given except for the
relevant balances.
Contract executions are only started when the balance of the calling account can support
the amount of tokens that should be transfered.
In an abstract scenario this balance or the amount of tokens to be transfered may not be known,
and in that case the symbolic execution rule creates a disjunction to cover either case.

\begin{comment}
\verb/αPrg-running/ is the same as its non-abstract counter part holding abstract contracts
and program states instead of concrete ones.

Besides using an abstract variant of \verb/blockchain/, the abstract version of the execution state
also differes in its other fields.
In the concrete setting, the new values for the contracts storage and balance
are written directly to the \verb/accounts/,
but the abstract contracts only hold variables for these entities, who's values are only
described by $\Phi$.
So when an abstract contract execution terminates, all of the information stored in
\verb/αPrg-running/ can be discarded, except for the formulas which will be stored
in case no contract is executed in the current state.
Also, since the emitted blockchain operations will also be stored as variables,
\verb/pending/ holds such variables instead of actual operations.

%% listing ruler max width ------------------------------------------------|?X
\begin{listing}[!ht]
\begin{minted}{agda}
record αProg-state Γ {ro so : Stack} : Set where
  constructor αstate
  field
    {ri si} : Stack
    αen : αEnvironment Γ
    prg : ShadowProg ri si ro so
    rVM : Match Γ ri
    sVM : Match Γ si
    Φ   : List (Formula Γ)

record αExec-state Γ : Set where
  constructor αexc
  field
    αccounts : βlockchain Γ
    αρ⊎Φ     : αPrg-running Γ ⊎ List (Formula Γ)
    pending  : List (list ops ∈ Γ × ⟦ base addr ⟧)
\end{minted}
\caption{Single abstract program and execution states}
\label{abstract-states}
\end{listing}

During symbolic execution, unlike in a concrete setting, execution is not deterministic
since the values for variables necessary for case distinctions may not be present in the current
execution state, i.e. $\Phi$.
This is modelled by giving abstracts states
as disjunctions of the states from \listref{abstract-states}:
%% single line listing ruler max width ------------------------------------------------|X
\mint{agda}|⊎Prog-state = λ {ro} {so} → List (∃[ Γ ] αProg-state Γ {ro} {so})|
\mint{agda}|⊎Exec-state = List (∃[ Γ ] αExec-state Γ)|

It is necessary to place the disjunctions on the level of abstract program and execution states
rather than defining disjunctions on the level of formulas,
because when they occure due to the symbolic execution of a given abstract state,
the resulting states also differ in the resulting stack matchings and rest program
(in case of a program state) or the resulting current execution state \verb/αρ⊎Φ/.

Using Agda Lists to represent conjunctions and disjunctions is conveniet because
\begin{enumerate}
	\item	within any of the given listing constructs ($\Phi$, \verb/αXxxx-state/)
		all elements are interpreted exclusively as con- or disjuncts,
		so defining them separately would only yield another list like structure.
	\item	some rules of the calculus as well as the semantics for our model
		can make efficient use of the element relation for Agda Lists.
\end{enumerate}

Note that while the storage and balance of abstract contract are modeled with respective variables,
their addresses on the abstract blockchain are \emph{NOT} modeled with variables but rather
with given values like in the concrete setting.
Abstracting them would be possible, but it wouldn't make the logic any more expressive
since cases where an address value on the stack is not known can still be modelled,
and having abstract contract addresses would only effect the symbolic execution of a
\verb/TRANSFERE-TOKEN/ operation \todo{annoying to formulate \ldots!!!!!!}

We give the calculus in a different way (as mentioned in \secref{sec:abstract-states})
by giving rules that will add new formulas to those already saved in the abstract states
for the abstract execution of the next instruction (or operation).
Or rather it will give a new abstract state disjunction that represents a formula
which is valid if the formula represented by the initial state
was valid before instruction execution (as we will proof in \chapref{chap:soundness}).

They are given in two different forms:

One gives the conclusion formula in a deterministic way that can always be applied to any
possible Michelson instruction, regardless of the values that may or may not be
provided for the relevant variables by the premise formula.
This resembles the deterministic execution of concrete states and is achieved
by accordingly named functions:
%% single line listing ruler max width ------------------------------------------------|X
\mint{agda}|αprog-step : ∀ {Γ ro so} → αProg-state Γ {ro} {so} → ⊎Prog-state {ro} {so}|
\mint{agda}|αexec-step : ∀ {Γ} → αExec-state Γ → ⊎Exec-state|

While in the concrete case, any kind of functional instruction can be performed by the compact
definition given in \listref{prog-step}, this can only be achieved for the subset of
onedimensional functions here (see \listref{aprog-step}):

%% listing ruler max width ------------------------------------------------|?X
\begin{listing}[!ht]
\begin{minted}{agda}
αprog-step {Γ} (αstate αen (fct (D1 {result} f) ; prg) rVM sVM Φ)
  = [ [ result // Γ ]
    , (αstate (wkαE αen) prg (0∈ ∷ wkM (Mbot rVM)) (wkM sVM)
              [ 0∈ := wk⊢ (func f (Mtop rVM)) // wkΦ Φ ]) ]

αprog-step {Γ} (αstate αen (fct (Dm (UNPAIR {t1} {t2})) ; prg)
                       (p∈ ∷ rVM) sVM Φ)
  = [ [ t1 / t2 // Γ ]
    , (αstate (wkαE αen) prg (0∈ ∷ 1∈ ∷ wkM rVM) (wkM sVM)
              [ 0∈ := wk⊢ (func CAR (p∈ ∷ [M]))
              / 1∈ := wk⊢ (func CDR (p∈ ∷ [M])) // wkΦ Φ ]) ]
αprog-step α@(αstate αen (fct (Dm SWAP) ; prg) (x∈ ∷ y∈ ∷ rVM) sVM Φ)
  = [ _ , record α{ prg = prg ; rVM = y∈ ∷ x∈ ∷ rVM } ]
αprog-step α@(αstate αen (fct (Dm DUP)  ; prg) (x∈      ∷ rVM) sVM Φ)
  = [ _ , record α{ prg = prg ; rVM = x∈ ∷ x∈ ∷ rVM } ]

αprog-step {Γ} (αstate αen (fct (PUSH P x) ; prg) rVM sVM Φ)
  = [ (expandΓ P x ++ Γ)
    , (αstate (wkαE αen) prg ((∈wk (0∈exΓ P)) ∷ wkM rVM) (wkM sVM)
              (Φwk (unfold P x) ++ wkΦ Φ)) ]
\end{minted}
\caption{Deterministic symbolic program state execution}
\label{aprog-step}
\end{listing}

It is possible for those functions since they double as the functional terms
that can be expressed in the DL.
A new variable of the result type is added to the Context and a new conjunct to $\Phi$,
stating that the new variable equals the term
that applies the argument variables from the top of the stack matching to that function.
The weakening operators \verb/wk/\emph{XY} accommodate for the new context.

For other functions, no generic symbolic execution is possible:
In case of the \verb/UNPAIR/ instruction, two new variables and corresponding clauses must be added,
while for \verb/SWAP/ and \verb/DUP/ nothing has to be added
(hence the context doesn't change, which Agda knows implicitly , and no weakenings are applied)
and only the variables on the top of the main stack matching must be changed.

\verb/PUSH P x/ is a special case for the abstract execution as mentioned before:
When pushing a basic type, an according variable and a clause
setting it equal to the \verb/const/ term \verb/x/ (see \listref{terms-formulas}) can be added.
In case of a complex type, \verb/x/ must be decomposed into
several new variables and clauses for those that represent \verb/x/
(e.g.: \verb/PUSH (list {ty} P) (y ∷ ys)/ would be decomposed into two new variables of type
\verb/ty/ and \verb/list ty/ where the first would be set to \verb/y/ and the second to \verb/ys/;
in case of \verb/ys = []/ that would be \verb/func (NIL ty) [M]/,
otherwise \verb/ys/ must be further decomposed,
as well as \verb/y/ in case \verb/ty/ is also a complex/composite type).

Abstract execution of \verb/ITER/ works just like the concrete case (see \listref{aITER}),
but while the top element of the shadow stack interpretation can be matched
to the only two possible values for the concrete execution
(an empty or non-empty list, see \listref{prog-step}),
this isn't possible for the variable \verb/l∈/
on top of the shadow stack matching in the abstract case.
So instead of giving two possible execution rules, only one is given that introduces a
disjunction for the two possibilities that can arise:

%% listing ruler max width ------------------------------------------------|?X
\begin{listing}[!ht]
\begin{minted}{agda}
αprog-step α@(αstate αen (ITER x ; prg) (l∈ ∷ rVM) sVM Φ)
  = [ _ , record α{ prg = ITER' x ∙ prg ; rVM = rVM ; sVM = l∈ ∷ sVM } ]

αprog-step {Γ} α@(αstate αen (ITER' {ty} x ∙ prg) rVM (l∈ ∷ sVM) Φ)
  = [ _ , record α{ prg = prg ; sVM = sVM 
                  ;   Φ = [ l∈ := func (NIL ty) [M] // Φ ] }
    / [ ty / list ty // Γ ]
    , αstate (wkαE αen) (x ;∙ ITER' x ∙ prg) (0∈ ∷ wkM rVM) (1∈ ∷ wkM sVM)
             [ 2+ l∈ := func CONS (0∈ ∷ 1∈ ∷ [M]) // wkΦ Φ ] ]
\end{minted}
\caption{Generic abstract execution for ITER}
\label{aITER}
\end{listing}

However, it may be possible that the premise includes additional information about \verb/l∈/,
if a clause is present in $\Phi$ where its stated to be equal to some term.
If that term happens to be either \verb/func (NIL ty) [M]/ or \verb/func CONS (x∈ ∷ xs∈ ∷ [M])/,
one case can savely be discarded.
To enable the DL to reach such conclusions, the following relation on abstract program states
is defined:

%% listing ruler max width ------------------------------------------------|?X
\begin{listing}[!ht]
\begin{minted}{agda}
data αρ-special {Γ ro so} :         αProg-state        Γ  {ro} {so}
                          → ∃[ Γ` ] αProg-state (Γ` ++ Γ) {ro} {so}
                          → Set where

  ITER'c : ∀ {αen ty l∈ x∈ xs∈ Φ rS sS iterate prg rVM sVM}
         → l∈ := func CONS (x∈ ∷ xs∈ ∷ [M])  ∈  Φ
         → αρ-special
               (αstate αen (ITER' {ty} {rS} {sS} x ∙ prg) rVM  (l∈ ∷ sVM) Φ)
           (_ , αstate αen (x ;∙ ITER' x ∙ prg)    (x∈ ∷ rVM) (xs∈ ∷ sVM) Φ)
\end{minted}
\caption{Special case (e.g. ITER' for CONS case)}
\label{arho-special}
\end{listing}

This also has the advantage that no new variables have to be introduced.

The same conclusion could have been drawn from first performing the generic case that will
create a disjunction, and then concluding a contradiction from the other clause
by a rule that sets $\Phi$\verb/ = `false/ when two contradicting clauses like
\verb/l∈ := func (NIL ty) [M]/ and \verb/l∈ := func CONS (x∈ ∷ xs∈ ∷ [M])/ for the same variable
are contained in $\Phi$.

Such rules have already been successfully implemented and proofen
in an earlier version of the model.
But since they come at a great performance cost by requiring 3 conclusion steps instead of one
and often creating new variables that make additional conclusions necessary to connect them,
these rules where not reimplemented during a major rework of the code.

As for abstract program states, the symbolic execution of abstract execution states
is also given once in a deterministic manner as mentioned above,
but also by a similar relation for special cases.

There is a significant difference however: 
When checking \verb/pending/ for further operations to be executed,
without further knowledge about the variables representing these operations,
no meaningful action can be performed to advance the symbolic execution process.
To execute any succeeding operation, it must be at least given that
\verb/list ops ∈ Γ/ is \verb/CONS/ of such an operation variable,
which in turn must also be given to have the value of a \verb/TRANSFERE-TOKENS/ term,
for which the \verb/contract/ variable must be given a value as well.
Without such additional information, which cannot be considered for the deterministic case,
the symbolic execution can only create a case distinction between the list of operations
being either empty or filled with at least one operation which still cannot be processed
due to the missing information.
Therefore the abstract execution step function cannot reflect the exact behaviour of its
concrete counterpart, and only for those cases where the abstract state contains a
contract under execution will the soundness of the symbolic execution be prooven.

\end{comment}


%%% Local Variables:
%%% mode: latex
%%% TeX-master: "itp2024"
%%% End:
