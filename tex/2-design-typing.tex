\section{Michelson}
\label{sec:michelson}
\label{sec:Mtype}


Michelson~\cite{Mref,devres} is the native language for smart contracts on the Tezos blockchain.
It is a stack-based, low level, strongly-typed functional programming
language. That is, all computation is driven by transforming an input
stack into an output stack. There are no mutable data structures;
blockchain transactions are handled outside of Michelson.
All contracts are statically typed to avoid type errors during runtime.
Michelson is simply-typed as there is neither universal nor
existential quantification.

Each Michelson instruction converts a given input stack into an output stack
where some of its values have been changed, added or removed.
For example, the \verb=ADD= instruction will accept any stack
whose two top most elements are numeric values,
and return a stack where these two values have been replaced by their sum
and the remaining stack is unchanged:
\begin{align*}
	\text{ADD} :: 15 :: 27 :: \text{\emph{remainingStack}}
	\mapsto	           42 :: \text{\emph{remainingStack}}
\label{equ:ADD}
\end{align*}

\subsection{Types}
\label{sec:michelson-types}

Michelson supports the usual data types like numbers, pairs, and lists as well as
some blockchain-specific types for tokens and contracts. 
Figure~\ref{fig:Type} contains Agda definitions for selected subset of
Michelson types {\AType}. As some base types can be treated alike later on, we
represent them with a separate type {\ABaseType}.

\begin{figure}[tp]
  \begin{subfigure}{0.48\textwidth}
    \noindent
    \TypesType
    \TypesPatterns
    \caption{Syntax}
    \label{fig:Type}
  \end{subfigure}
  \begin{subfigure}{0.48\textwidth}
  \TypesSemantics
  \caption{Semantics}
  \label{fig:Type-Semantics}
\end{subfigure}
\caption{Michelson Types}
\label{Type}
\end{figure}

Most types' names are self explanatory. The base type {\Amutez} stands
for tokens, {\Aaddr} stands for blockchain addresses in  Tezos. We use
some obvious shorthand patterns for base types.
The type {\Aoperation} consists of blockchain operations that can
be emitted during contract execution. This mechanism implements token
transfers from the current contract to other accounts or contracts.
The type {\Acontract~\AP} represents such a contract
which accepts a parameter of type {\Aty} represented by {\APassable~\Aty}.
The type predicate {\TypesPassable} is declared mutually recursive with {\AType}
and characterizes types that can be passed as parameters to contracts.

The semantics of types is defined by a mapping to Agda types. Most
Michelson types have  obvious Agda counterparts, except {\Aaddr},
{\Acontract}, and {\Aoperation}.  Addresses and contracts are both
represented by natural numbers. We only define one alternative of the
{\AOperation} datatype: {$\AtransferTokens~v~m~c$}, which models a
token transfer to contract $c$ while passing the parameter value $v$
and tokens $m$.

\subsection{Programs and Instructions}
\label{sec:michelson-programs}


Michelson programs are intrinsically typed and represented accordingly in
Agda by a datatype {\AProgram} indexed by the types on the input stack
and the types on the output stack. We assume that \TypesStack.
\SyntaxProgram

Instructions are indexed in the same way:
If  instruction \verb/inst/ maps an input stack $Si$ to an output stack $So$
and \verb/prg/ maps that output stack $So$ to the final stack $Se$,
then \verb/inst ; prg/ is a program that maps $Si$ to $Se$.
The empty program {\Aend} does not transform the stack.

\begin{figure}[tp]
  \SyntaxInstruction  
  \caption{Instructions of Core Michelson}
  \label{fig:core-michelson-instructions}
\end{figure}
We discuss a representative subset of Michelson instructions shown in Figure~\ref{fig:core-michelson-instructions}.
The definition of {\AInstructionPlus} implements the pattern that most
instructions only transform the top elements of the input stack and are parametric in the rest. 

The first group of instructions operates on a fixed number of values on
the stack and pushes the result. All arithmetic operations belong to
this group and we just give two examples, {\AADDnn}  and {\AADDm},
which perform addition of natural numbers and tokens, respectively.
The original Michelson language overloads arithmetic operators. As
overloading is not supported by Agda, we supply separate
instructions. We come back to this issue at the end of this section. 

\ACon{CAR}, \ACon{CDR}, and \ACon{PAIR} are the standard operations on
pairs. \ACon{NONE} and \ACon{SOME} are the constructors for the
\ACon{option} datatype, and \ACon{NIL} and \ACon{CONS} construct
lists. The constructors for ``empty'' containers, \ACon{NONE} and
\ACon{NIL} are indexed by the element type, otherwise that type can be
inferred from the context.

The last instruction in this group is \ACon{TRANSFER-TOKENS}. Despite
the name, this instruction does \textbf{not} directly transfer tokens
to another account. It rather constructs a value
{$\AtransferTokens~v~m~c$} of type {\Aoperation} from its arguments.

The instructions in the next group differ in that they push zero or more
values on the output stack. \ACon{DROP} pops the stack, \ACon{DUP} duplicates the top of the
stack, \ACon{SWAP} swaps the top entries, and \ACon{UNPAIR} eliminates
a pair and pushes its contents. \ACon{UNPAIR} is a convenience
instruction as it is equivalent to the instruction sequence
\ACon{DUP}; \ACon{CDR}; \ACon{SWAP}; \ACon{CAR}. 

The next group contains instructions that are blockchain
specific. \ACon{AMOUNT} returns the tokens that were transferred with
the currently running contract invocation and \ACon{BALANCE}  returns
the tokens currently owned by it. The \ACon{CONTRACT} instruction is
indexed by a type $t$ that must be {\APassable}. It takes an address
and check on the blockchain whether this address is associated to a
contract that accepts arguments of type $t$. The result is
communicated as an \ACon{option} types. That is, the {\Acontract} type
carries a verified address.

The \ACon{PUSH} instruction pushes a value of type $t$ on the stack
assuming that value is {\APushable} (another predicate an types). The
value is encoded by a type-indexed datatype \AgdaDatatype{Data} for
pushable values. We elide its obvious definition.

The last group of instructions showcases control structures and an
instruction that operates in a non-uniform way on the stack. The
instruction \ACon{IF-NONE} eliminates a value of option type from the
top of the stack. Its parameters are programs that implement the branches for case \ACon{None}
and \ACon{Some}. The latter takes the value wrapped in the \ACon{Some}
constructor as an argument on top of the stack.

The instruction \ACon{ITER} runs a sub-program on every element of its
argument list. The instruction $\ACon{DIP}~n$ runs a sub-program at
depth $n$ on the input stack, that is, it skips over the first $n$
elements of the stack, runs the sub-program, and reattaches those
elements. The extra machinery in the implicit argument of the
instruction makes sure that there are at least $n$ elements on the
stack. This mechanism is called reflection in the  PLFA textbook
\cite{plfa}.

Earlier, we remarked that Agda does not allow overloading of
constructors in the same datatype. However, we can use reflection to
define a ``smart constructor'' that almost suits the purpose.
\SyntaxOverloading

The definition exploits the fact that the input stack of an
instruction is always known in a Michelson program. The same fact also
enables overloading in Michelson's implementation to work.
The function \AFun{overADD} specifies the resolution of overloading
for the \AFun{ADD} instruction. If the argument types are both
\ACon{nat}, then the result type is \ACon{nat} and the chosen instruction is
\ACon{ADDnn}; and so on.\footnote{Full Michelson has ten different
  overloadings of \AFun{ADD}.} If no  overloading is known for a combination
of arguments, the function return \ACon{nothing}.
The smart constructor \AFun{ADD} takes a proof that the overloading is
defined on a given pair of input types. Then it extracts the selected
instruction from the overloading.

Compared to ``real'' Michelson, the smart constructor requires an
extra argument to work:
\SyntaxOverloadingExample

\subsection{Blockchain interface}
\label{sec:blockchain-interface}

A contract on the Tezos blockchain is indexed by a parameter type $p$
and a store type $s$. The type $p$ must be {\APassable} and the type
$s$ must be {\AStorable}. Moreover, each contract comes with a current
balance of tokens and a store of type $s$. The implementation of the
contract is a program that maps a $\ACon{pair}~p~s$ to a
$\ACon{pair}~(\ACon{list}~\Aoperation)~s$, that is, it consumes the
parameter paired with the current store and produces a list of operations (e.g., to invoke further
contracts) paired with the updated store. The program itself is pure;
any side effects, i.e., store update and contract calls, are managed
by the blockchain run-time.
\ConcreteContract

The $Mode$ argument abstracts over the semantics of types. Its type
has two components, one $\mathcal{M}$ for the semantics and the other $\mathcal{F}$ will be
explained in the context of the abstract semantics in Section~\ref{sec:abstract-semantics}.
\ConcreteMODE
Its instantiation for the concrete semantics installs the 
standard semantics from Section~\ref{sec:michelson-types}.
\ConcreteCMode

With this definition, the contract store of the blockchain is just a partial mapping from addresses
to contracts.
\ConcreteBlockchain

To start executing a contract, we make a blockchain transaction to its
address, i.e., we ask the blockchain runtime to transfer tokens to it along with its parameter.
Once a contract has terminated, the runtime updates the
stored value and processes the list of operations.

On the Tezos blockchain a normal account with deposit $init$ is a
contract with a unit parameter, unit store, and a trivial program that issues no operations.
\ConcreteAccount


\begin{comment}
\subsection{Michelson instructions}\label{sec:instructions}

The way Michelson is represented in this thesis is heavily tailored towards
the soundness proof and making it as easy and concise as possible in Agda.

The first consequence of this paradigm is that rather than defining the instructions
and subsequently associating them with their respective typing rules,
we use an approach of \emph{intrinsically-typed} instructions similar to
intrinsically-typed terms as described in~\cite{plfa}:
Our Agda definition for instructions is indexed by the input and output stack
that the instruction can operate on, and it is not possible to give an instruction
without also specifying its in- and output stacks:
\mint{agda}|data Instruction : Stack → Stack → Set|
Note that \verb/Stack/ in the Agda code is a list of \verb/Type/'s representing
the type of a given stack, corresponding to the meaning of ``stack'' throughout this chapter.

Michelson programs are represented as a concatenation of instructions for matching stacks:
%% listing ruler max width ------------------------------------------------|?X
\begin{figure}[tp]
\begin{minted}{agda}
data Program : Stack → Stack → Set where
  end  : ∀ {S} → Program S S
  _;_  : ∀ {Si So Se}
       → Instruction  Si   So
       → Program      So   Se
       → Program      Si   Se
\end{minted}
\caption{Programs are lists of instructions}
\label{Program}
\end{figure}
If \verb/inst/ maps an input stack \verb/Si/ to an output stack \verb/So/
and \verb/prg/ maps that output stack \verb/So/ to the ``end'' stack \verb/Se/,
then \verb/inst ; prg/ is a program that maps \verb/Si/ to \verb/Se/.
This way Agdas typechecker will enforce that only well typed Michelson programs can be entered.

Another consequence is that instructions are further subdivided into different categories.
The main categories are \emph{functional} and \emph{control flow} instructions.
The former are all instructions that represent a function that takes the top of the
current stack as arguments and maps it to some result, while leaving the remaining stack unchanged.
All instructions from the example program in \listref{simple-example} are from that category.
They can be executed in a single step and will be further subdivided into several categories.

Because they do not change the remaining stack and only require the top of the stack to match
their argument types, their subcategories will be given as function types indexed by their
argument and result types, ignoring the remaining unchanged stack.
When defining \verb/Instruction/,
these will be mapped to instructions that work on any remaining substack.

The biggest subcategory of functions combines all functions
whose output only depends on the input arguments from the top of the stack
and no further knowlegde of the execution environment is necessary for their execution.

This group is further subdivided according to their role during symbolic execution:
Multidimensional functions, as well as the \verb/PUSH/ instruction, require special
treatment when they are symbolically executed, which is explained in \secref{sec:calculus}.
All other onedimensional functions serve an additional purpose:
They will later be reused in the Dynamic Logic to define most of the terms for the logic, which,
unlike all other functional instructions, also gives them a universal symbolic execution scheme.

The function types of these instructions is given in \listref{func-type}.
We employ some Agda pattern synonyms for short lists \verb|[ a ] [ a / b ]|
of one or two elements.
\verb/⟦ ty ⟧/ is the set of values of type \verb/ty/.
Notice, that there are two definitions of \verb/ADD/ for naturals and for mutez tokens.
This is necessary for our intrinsic typing scheme, but since the typing rule for each instruction
of a contract is unambiguous, our implementation does not loose expressiveness by this restriction.
% out implementation of Michelson does not loose expressiveness

%% listing ruler max width ------------------------------------------------|?X
\begin{figure}[tp]
\begin{minted}{agda}
data 1-func : Stack → Type → Set where
  ADDnn  :                1-func  [ base   nat / base   nat ]  (base   nat)
  ADDm   :                1-func  [ base mutez / base mutez ]  (base mutez)
  CAR    :  ∀ {t1 t2}  →  1-func               [ pair t1 t2 ]           t1
  CDR    :  ∀ {t1 t2}  →  1-func               [ pair t1 t2 ]           t2
  PAIR   :  ∀ {t1 t2}  →  1-func                  [ t1 / t2 ]  (pair t1 t2)
  NIL    :  ∀  ty      →  1-func                           []  (list    ty)
  NONE   :  ∀  ty      →  1-func                           []  (option  ty)
  SOME   :  ∀ {ty}     →  1-func                       [ ty ]  (option  ty)
  CONS   :  ∀ {ty}     →  1-func             [ ty / list ty ]  (list    ty)
  TRANSFER-TOKENS : ∀ {ty pt}
            →  1-func  [ ty / base mutez / contract {ty} pt ]          ops

data m-func : Stack → Stack × Type → Set where
  UNPAIR  :  ∀ {t1 t2}  →  m-func  [ pair t1 t2 ]   ([ t1 ] , t2)
  SWAP    :  ∀ {t1 t2}  →  m-func     [ t1 / t2 ]   ([ t2 ] , t1)
  DUP     :  ∀ {ty}     →  m-func          [ ty ]   ([ ty ] , ty)

data func-type : Stack → Stack × Type → Set where
  D1   : ∀ {res  args} → 1-func args res      →  func-type args ([] , res)
  Dm   : ∀ {args ress} → m-func args ress     →  func-type args       ress
  PUSH : ∀ {ty}        → Pushable ty → ⟦ ty ⟧ →  func-type []   ([] ,  ty)
\end{minted}
\caption{Function types}
\label{func-type}
\end{figure}

Besides those, \listref{env-func} defines functions that provide information about the
current execution environment,
like the amount of tokens that was transfered when starting the execution,
or the contract stored at a given address.
Their typing scheme does not differ from \verb/func-type/ functions,
but since their execution requires this information,
they have to be executed differently and will be given their own category.

\begin{figure}[tp]
\begin{minted}{agda}
data env-func : Stack → Type → Set where
  AMOUNT    :             env-func             []               (base mutez)
  BALANCE   :             env-func             []               (base mutez)
  CONTRACT  :  ∀ {ty} P → env-func  [ base addr ] (option (contract {ty} P))
\end{minted}
\caption{Functions for blockchain operations}
\label{env-func}
\end{figure}

Control flow instructions are those that take subprogram as arguments.
Their execution is defined in big step semantics,
and since their subprograms can be arbitrarily complex,
their execution requires additional features, which are explained in \chapref{chap:refImpl}.

Their typing rules are defined in the \verb/Instruction/ datatype in \listref{Instruction},
together with the functional instructions.
The different function types are mapped to their corresponding instructions
that work for any possible stack where the top matches the function argument types.
The Agda pattern \verb|[ x // xs ]| stands for the normal list constructor \verb/x ∷ xs/.
Since \verb/results/ of \verb/func-type/ has the type \verb/Stack × Type/,
it must be mapped to a list before it is concatenated with the substack.
\verb/DROP/ is the only 0-dimensional function and therefore provided with its own constructor.

%% listing ruler max width ------------------------------------------------|?X
\begin{figure}[tp]
\begin{minted}{agda}
data Instruction where
  enf       : ∀ {args result S}
            → env-func args result
            → Instruction  (       args ++ S )       [ result // S ]
  fct       : ∀ {args results S}
            → func-type args results
            → Instruction  (       args ++ S )  ([× results ] ++ S)
  DROP      : ∀ {ty S}
            → Instruction  [         ty // S ]                   S
  ITER      : ∀ {ty S}
            → Program      [         ty // S ]                   S
            → Instruction  [    list ty // S ]                   S
  DIP       : ∀ {S Se} n → {T (n ≤ᵇ length S)}
            → Program              (drop n S)                    Se
            → Instruction                  S        (take n S ++ Se)
  IF-NONE   : ∀ {ty S Se}
            → Program                      S                     Se
            → Program      [         ty // S ]                   Se
            → Instruction  [  option ty // S ]                   Se
\end{minted}
\caption{Intrinsically typed Michelson instructions}
\label{Instruction}
\end{figure}

\listref{simple-example-our} shows the example program from \listref{simple-example}
programmed in our Michelson representation.
\begin{figure}[tp]
\begin{minted}{agda}
example : Program [ pair (base nat) (base nat) ] [ pair (list ops) (base nat) ]
example = fct (Dm UNPAIR) ;
          fct (D1 ADDnn) ;
          fct (D1 (NIL ops)) ;
          fct (D1 PAIR) ; end
\end{minted}
\caption{Example program in our Michelson representation}
\label{simple-example-our}
\end{figure}

\end{comment}
\begin{comment}
The biggest and most importand subcategory of functional instructions is that of
those instructions whose typing rule 
These are all onedimensional functions whose execution only requires knowledge
of the current stack values that serve as the arguments of these functions,
like \verb/ADD/, \verb/PAIR/ or \verb/NIL ty/ from the example above.
However there is one exception: \verb/PUSH/ technically matches these requirenments,
but its symbolic execution will require some special treatment as we will see later.
Therefore it is given its own category.
Together with 

They will be used to represent certain terms of their result type
We will give these subcategories and they motivation in the same order they are defined in the
\model{02-Functions-Interpretations} module before explaining the \verb/Instruction/ datatype.

The biggest subcategory is that of onedimensional functions whose implementation
only requires the knowledge of 

% Functions to retrieve information from the execution environment,
% will require these additional informations besides the current state of the stack.
Control flow instructions are those that take subprogram as arguments.
Their execution is defined in big step semantics,
and since these subprograms can be arbitarily complex, their execution is a bit involved
and will be discussed in \chapref{chap:refImpl}.
% Their typing rules however are straight forward and can be easily implemented.

The first subcategory of functional instructions contains instructions 
to retrieve blockchain related information,
like the amount of tokens that was transfered when starting the current execution,
or the contract stored at a given address.
Their typing rules don't differ from other functional instructions,
but will be defined separately since their execution requires these informations
from the current execution environment.
All other functional instructions operate on the current stack values alone
and are further divided:
Onedimensional functions have a special role since they will later be reused in the
Dynamic Logic to define most of its terms, which will also make their symbolic execution
straight forward.
Multidimensional functions (like \verb/UNPAIR/ in the example above) need to be treated
individually during symbolic execution.
\verb/PUSH/ is technically a onedimensional function,
but its symbolic execution requires a special treatment as we will see later,
so it is given its own cathegory as well.

All functional instructions only require the top of the input stack to match the
required argument types and leave the rest of the stack unchanged.
This makes it convenient to define each functional subcategory by their argument and result types
and extend that typing when constructing instructions from them.

are represented separately since their execution
are given its own datatype \verb/env-func/ in the modell.
All other functional instructions operate only on the current stack values


into functions that only require the current
stack values to calculate its results, and functions
whose function operates solely on the values from the current stack
and those that 


\subsection{Design rational}\label{sec:desigr}
The guiding idea for out code design was reusability of typing rules
for the reference implementation in the dynamic logic
to make the soundness proofs as concise and easy as possible.

While the reference implementation will work on a stack of concrete values that are
provided for and manipulated during execution,
the DL will operate on a stack of variables for such values that may or may not be
accompanied by formulas that assign values to them.

So the Michelson instructions are defined by their typing rules,
and during concrete execution those typing rules will be matched to their corresponing
implementation rules (semantic rules),
while during symbolic execution a new variable will be introduced and a formula added
that describes its value by using its typing rule.

So for the \verb=ADD= instruction, the typing rule would be
\[			\text{nat} :: \text{nat} :: \text{\emph{remStack}}
	\Longrightarrow		      \text{nat} :: \text{\emph{remStack}}
\]
which will be mapped to its natural implementation $+ : \bN \to \bN \to \bN$ for concrete execution
as well as to a formula that describes a variable of type $\bN$ as being $\text{ADD} x y$
for two variables $x$ and $y$ of type $\bN$.

Notice that in Michelson some instructions, like \verb=ADD=, are not monomorphic.
However (due to Michelson being \emph{strongly typed}),
because the input and output stacks for smart contracts \emph{are} monomorphic,
the typing rule for every instruction within a given smart contract is fixed.
For that reason, we chose to give a monomorphic reference implementation
instead of
% which eliminates the teadious work of 
matching each instruction with the
typing rule that would be applied for a given contract,
since this reusability scheme applies to the typing rules anyways
and we don't loose generality by giving several add-instructions.
Any given Michelson contract can be compiled to one in our reference implementation
in linear time of its size due to Michelsons strict typing rules.

For \verb=ADD=, we define it for the two typing rules
for those Michelson types we chose to implement, with is \verb=ADDnn= for naturals
and \verb=ADDm= for adding to \verb=mutez= values.

Also, not all instructions represent unary functions \todo{is \emph{unary} correct here?}
like \verb=ADD= does, but only those can be reused for formulas.
So the implemented Michelson instructions are given in certain subcategories
according to their properties.

Of course we want to express in formulas that a given variable has some constant value,
but constant values should be restricted to variables of primitive types.
Constant values for complex types will be represented
with formulas of their according introduction instructions
like \verb=PAIR= for pairs or \verb=CONS= and \verb=NIL= for lists
and more formulas \draft{that fix the variables used there}.
This make reasoning about complex types easier, if not possible in the first place:
In an earlier model where this restriction was not present,
it turned out near impossible to proof that iterating over a list of 3 elements would terminate
without giving concrete values for those 3 elements.
But formulating such conclusions in general without giving the exact values
should be acchievable with a DL.

\subsection{Typing System}\label{sec:typing}

\subsubsection{Types}\label{subsec:types}

The Agda module \todo{link to github} \verb=01-Types= defines a data type \verb=BaseType=
to represent the primitive data types that have been implemented:
\begin{figure}[tp]
\begin{minted}{agda}
data BaseType : Set where
  unit  : BaseType
  nat   : BaseType
  mutez : BaseType
  addr  : BaseType
\end{minted}
\caption{Basic Types}
\label{BaseType}
\end{figure}

A Michelson type in our model is either a \verb=BaseType= or a complex type,
which can be a higher order type like \verb=pair=, \verb=list= or \verb=option=
or blockchain specific types like an \verb=operation= (abbreviated to \verb=ops=)
or a \verb=contract=:
\begin{figure}[tp]
\begin{minted}{agda}
data Type where
  ops          :               Type
  base         : BaseType    → Type
  pair         : Type → Type → Type
  list         : Type        → Type
  option       : Type        → Type
  contract     : ∀ {ty} → Passable ty → Type
\end{minted}
\caption{Michelson Types}
\label{Type}
\end{figure}

For correct implementation of the typing restrictions the 3 predicates
\verb=Pushable=, \verb=Passable= and \verb=Storable= are defined.
There are a lot more restrictions present in Michelson, but only these are needed
for the types implemented here.

After defining the \verb=DecidableEquality= operator on \verb=Types=,
a \verb=Stack= is defined to be a list of \verb=Types=.
To more easily differentiate between the types and values of a Michelson stack
as well as a Stack of DL variables that represent the program stack in the DL,
we will use these following terms:
\begin{description}
	\item[Stack]
		unless stated explicitly, \textbf{Stack} will always referre
		to a Stack of Michelson types (rather than a stack of values
		as in most documents regarding Michelson)
	\item[Interpretation]
		will mean a stack of values, corresponding to the \verb=Stack= indexed
		data type \verb=Int= in the module \verb=02-Functions-Interpretations=
	\item[Matching]
		lastly will mean a stack of DL variables representing a program stack
		during symbolic execution. It corresponds to the \verb=Match= data type
		defined in \verb=11-abstract-representation-and-weakening=,
		which is also indexed by a \verb=Stack=
\end{description}
The Agda constructors for the latter two work exactly like Agda List constructors
and the standard list operations \verb=take=, \verb=drop= and \emph{concatenation}
are implemented for those the same way as for lists,
together with operators to retrieve the top or bottom of an Interpretation or Matching
if they are indexed by a concatenation of two Stacks.

\subsubsection{Functions}\label{subsec:functions}

The module \verb=02-Functions-Interpretations= first defines the different kind of functions
that will be used:
\begin{itemize}
	\item	one-dimensional functions are the most common and those that will be
		reused for formulas. \\
		they are parameterized by the \verb=Stack= of required input types
		and the \emph{single} resulting output type (see \listref{1-func}).
	\item	there are some multidimensional functions (\listref{m-func}).
		% (\listref{m-func} \ldots making the second parameter a product of
		% \verb=Stack= and \verb=Type= enables ).
		The reference implementation won't make a difference between these
		(that's also why the second parameter of \verb=m-func= must be
		the product of \verb=Stack= and \verb=Type=, although only a \verb=Stack=
		should suffice)
		but for the DL implementation special cases are needed here.
	\item	the functions so far work on the current stack alone,
		but instructions for blockchain operations require informations
		about the current blockchain environment they are executed in.
		They don't differ from \verb=1-func= regarding typing
		but have to be treated differently during execution later on (listref{env-func}).
	\item	\emph{there is a 2-dimensional blockchain operation which was'n implemented
		in this}%  model and will require some extensions or reworking of it}
\end{itemize}

All functions that work on stack elements alone are combined in the \verb=func-type=
which also includes the \verb=PUSH= instruction (\listref{func-type}).
It's a onedimensional function regarding its typing rule, but Michelson allows to push
compounded types that can't necessarily be expressed by a single formula,
so it is separated out from the other function types for special care during symbolic execution.

%% listing ruler max width ------------------------------------------------|?X
\begin{figure}[tp]
\begin{minted}{agda}
data func-type : Stack → Stack × Type → Set where
  D1   : ∀ {args res} → 1-func args res      →  func-type args ([] , res)
  Dm   : ∀ {args res} → m-func args res      →  func-type args       res
  PUSH : ∀ {ty}       → Pushable ty → ⟦ ty ⟧ →  func-type []   ([] ,  ty)
\end{minted}
\caption{functions for blockchain operations}
\label{func-type}
\end{figure}


\subsubsection{Shadow Stack and extended Instructions/Programs}\label{subsec:shadow}

\verb=DIP= and \verb=ITER= are special in that they need a second stack to be executed:
During the execution of the sub-program of \verb=DIP= the top \verb=n= elements must be stored
to be later retrieved when sub-program execution has terminated.
Likewise for \verb=ITER=, which consumes the list at the top of the stack
by executing its sub-program for every list element.
Here the currently remaining list has to be stored during sub-program execution.
Since the sub-programs can also contain such instructions that need to store data away for later,
a second stack is needed, as well as new instructions to operate on it:
To execute \texttt{DIP n prg ; \emph{remainingProg}}, the top \verb=n= elements of the
main stack are transfered to this second stack (we call it the \emph{shadow stack}) and
the instruction is replaced by \verb=prg= followed by the \emph{shadow instruction} \verb=DIP'=
which retrieves them from the shadow stack and puts them back onto the mainstack
before continuing execution with \texttt{\emph{remainingProg}}.
For the instruction \verb=ITER prg= the list on top of the main stack will be placed
onto the shadow stack and the instruction will be replaced by its shadow version \verb=ITER' prg=
which does all the actual work: It checks if the list at the top of the shadow stack is empty.
If so, it will be dropped and execution continues.
If not, the first element in the list will be moved to the top of the main stack and
\verb=prg= will be executed. After that \verb=ITER'= is executed again to check the list
until all elements have been iterated over.

These new shadow instructions must therefore be parameterized by 4 stacks:
main input stack, shadow input stack, main output stack and shadow output stack.
Analogous to the abbreviations in our code we will call the main stack \emph{real stack}.

Shadow programs are programs containing ``real'' and shadow instructions (see \listref{shadow}).

%% listing ruler max width ------------------------------------------------|?X
\begin{figure}[tp]
\begin{minted}[linenos]{agda}
data ShadowInst : Stack → Stack → Stack → Stack → Set where
  ITER'     : ∀ {ty rS sS}
            → Program      [ ty // rS ]                              rS
            → ShadowInst           rS   [ list ty // sS ]            rS  sS
  DIP'      : ∀ top {rS sS}
            → ShadowInst           rS        (top ++ sS )    (top ++ rS) sS

data ShadowProg : Stack → Stack → Stack → Stack → Set where
  end  : ∀ {rS sS} → ShadowProg rS sS rS sS
  _;_  : ∀ {ri rn si ro so}
       → Instruction ri     rn
       → ShadowProg  rn si  ro so
       → ShadowProg  ri si  ro so
  _∙_  : ∀ {ri si rn sn ro so}
       → ShadowInst  ri si  rn sn
       → ShadowProg  rn sn  ro so
       → ShadowProg  ri si  ro so
\end{minted}
\caption{shadow instructions and programs}
\label{shadow}
\end{figure}

Operators to concatenate programs and shadow programs are given in the canonical way:
\begin{figure}[tp]
\begin{minted}{agda}
_;;_ : ∀ {Si So Se} → Program Si So → Program So Se → Program Si Se
_;∙_   : ∀ {ri rn si ro so}
       → Program ri rn → ShadowProg rn si ro so → ShadowProg ri si ro so
\end{minted}
\caption{program concatenations}
\label{concat}
\end{figure}

\end{comment}

%%% Local Variables:
%%% mode: latex
%%% TeX-master: "itp2024"
%%% End:
