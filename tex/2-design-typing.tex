\section{Types and Programs}\label{sec:Mtype}


Michelson~\cite{Mref,devres} is the native language for smart contracts on the Tezos blockchain.
It is a stack based, low level, strongly-typed functional programming language
that supports high level data types like pairs and lists as well as
some blockchain-specific types for tokens and contracts.
All contracts are statically typed
to avoid type errors during runtime.
Michelson is simply-typed as there is neither universal nor
existential quantification.

Each Michelson instruction converts a given input stack into an output stack
where some of its values have been changed, added or removed.
For example, the \verb=ADD= instruction will accept any stack
whose two top most elements are numeric values,
and return a stack where these two values have been replaced by their sum
and the remaining stack is unchanged:
\begin{align*}
	\text{ADD} :: 15 :: 27 :: \text{\emph{remainingStack}}
	\mapsto	           42 :: \text{\emph{remainingStack}}
\label{equ:ADD}
\end{align*}

\textbf{****************************************}

\subsection{Michelson Types}

Michelson contains primitive datatypes as well as higher level types.
This is represented in the Agda model with the datatypes \verb/BaseType/ for the primitive ones
and \verb/Type/ to represent any Michelson type where the primitive types are included:

\begin{figure}[tp]
\TypesType
% \begin{minted}{agda}
% data BaseType : Set where
%   unit  : BaseType
%   nat   : BaseType
%   mutez : BaseType
%   addr  : BaseType
%
% data Type where
%   ops          :               Type
%   base         : BaseType    → Type
%   pair         : Type → Type → Type
%   list         : Type        → Type
%   option       : Type        → Type
%   contract     : ∀ {ty} → Passable ty → Type
% \end{minted}
\caption{Michelson Types}
\label{Type}
\end{figure}

This is the subset of Michelson types we chose to demonstrate most of its complexity.
Besides the \verb/base/ constructor for primitive types,
there are complex types like pairs, lists and optional types
that can form an arbitrarily complex compound Michelson type.
\verb/ops/ is the abbreviation of the special Michelson type \verb/operation/
for the blockchain operations that can be emitted during contract execution.
\verb/contract/ is another blockchain related Michelson type that represents a Tezos contract
which accepts data of type \verb/ty/ as parameter.
The type predicate \verb/data Passable : Type → Set/ is declared mutually recursive with \verb/Type/
and characterizes those types that can be passed as parameters.

Michelson enforces strict static type checking,
so during program execution the types of the in- and output stacks of each instruction
are unambiguous
(this chapter is mostly concerned with the type of the Michelson stack under consideration,
i.e. the types of its values,
so throughout the chapter, unless otherwise statet, a ``stack'' will refer to
the type of a stack or a stack of Michelson types).

However, Michelson instructions are not monomorphic, for example the \verb=ADD= instruction
has several typing rules for different numeric input types,
and the output stack can only be determined unambiguously if the input stack is known.

This is possible due to the typing restrictions enforced by the contract's parameter and storage
types:
When a contract is originated (i.e. when its account is created on the blockchain),
its program is checked against the parameter and storage type and will only be accepted
if it complies with the following restriction:
It must accept a stack with a single element \verb/pair parameter storage/,
and it must return a stack with \verb/pair (list ops) storage/ as its single element.
Since the initial stack will be known, the output stack resulting from each instruction
is unambiguous and therefore every stack during the entire contract execution will be monomorphic,
although individual Michelson instructions may be polymorphic.

\listref{simple-example} gives an example program to demonstrate these properties:
\begin{figure}[tp]
\begin{minted}{agda}
UNPAIR; ADD; NIL ops; PAIR
\end{minted}
\caption{Michelson example program}
\label{simple-example}
\end{figure}

When an execution is initialized, the parameter that was sent will be paired with the
value that the contract has saved in its storage and placed on the otherwise empty stack.
This program would unpair this initial pair, resulting in a stack where the parameter is
on top and the contracts storage below. \verb/ADD/ will add these two stack elements together
and leave their sum. \verb/NIL ty/ pushes an empty list of type \verb/ty/ onto the stack,
which in this case will then contain an empty list of operations
above the sum of parameter and storage,
and \verb/PAIR/ finally combines those two stack elements into one pair, leaving the expected
pair of emitted operations (in this case the contract does not emit any) and the updated
storage of the contract.
%%%%% SUGAR ;P \todo{maybe show this in a table!!!}

Without knowing the parameter and storage types, this program could be executed on any stack
where the initial pair contains any combination of numeric types that \verb/ADD/ would accept
and it is not clear which typing rules would apply here.
But since Michelson programs can only be executed as part of a Tezos smart contracts,
these parameters and therefore the specific typing rule to apply for this \verb/ADD/ instruction
will always be known and unambiguous.


\subsection{Smart contract features}\label{sec:smart}

Besides features for basic data manipulation like arithmetic operations and control flow,
Michelson naturally contains some special blockchain related features.

A program can only be executed as part of a smart contract,
which is characterized by its parameter and storage types.
The parameter specifies the type of data that has to be sent to the contract when starting
an execution.
Each contract can store one Michelson data value of its storage type
on the blockchain between executions.
Besides updating its storage, a smart contract can also emit new blockchain operations
which will be triggered after its successful execution.
Smart contracts are also just accounts on the Tezos blockchain and hold some amount of Tezos tokens
like any other blockchain account.
So, to execute a contract, one makes a blockchain transaction to that contract,
i.e. send some tokens to it, together with the parameter it requires.

These characteristics constitute some typing restrictions on the programs of smart contracts
which will be discussed after introducing some basics of Michelsons type system.

\subsection{Michelson instructions}\label{sec:instructions}

The way Michelson is represented in this thesis is heavily tailored towards
the soundness proof and making it as easy and concise as possible in Agda.

The first consequence of this paradigm is that rather than defining the instructions
and subsequently associating them with their respective typing rules,
we use an approach of \emph{intrinsically-typed} instructions similar to
intrinsically-typed terms as described in~\cite{plfa}:
Our Agda definition for instructions is indexed by the input and output stack
that the instruction can operate on, and it is not possible to give an instruction
without also specifying its in- and output stacks:
\mint{agda}|data Instruction : Stack → Stack → Set|
Note that \verb/Stack/ in the Agda code is a list of \verb/Type/'s representing
the type of a given stack, corresponding to the meaning of ``stack'' throughout this chapter.

Michelson programs are represented as a concatenation of instructions for matching stacks:
%% listing ruler max width ------------------------------------------------|?X
\begin{figure}[tp]
\begin{minted}{agda}
data Program : Stack → Stack → Set where
  end  : ∀ {S} → Program S S
  _;_  : ∀ {Si So Se}
       → Instruction  Si   So
       → Program      So   Se
       → Program      Si   Se
\end{minted}
\caption{Programs are lists of instructions}
\label{Program}
\end{figure}
If \verb/inst/ maps an input stack \verb/Si/ to an output stack \verb/So/
and \verb/prg/ maps that output stack \verb/So/ to the ``end'' stack \verb/Se/,
then \verb/inst ; prg/ is a program that maps \verb/Si/ to \verb/Se/.
This way Agdas typechecker will enforce that only well typed Michelson programs can be entered.

Another consequence is that instructions are further subdivided into different categories.
The main categories are \emph{functional} and \emph{control flow} instructions.
The former are all instructions that represent a function that takes the top of the
current stack as arguments and maps it to some result, while leaving the remaining stack unchanged.
All instructions from the example program in \listref{simple-example} are from that category.
They can be executed in a single step and will be further subdivided into several categories.

Because they do not change the remaining stack and only require the top of the stack to match
their argument types, their subcategories will be given as function types indexed by their
argument and result types, ignoring the remaining unchanged stack.
When defining \verb/Instruction/,
these will be mapped to instructions that work on any remaining substack.

The biggest subcategory of functions combines all functions
whose output only depends on the input arguments from the top of the stack
and no further knowlegde of the execution environment is necessary for their execution.

This group is further subdivided according to their role during symbolic execution:
Multidimensional functions, as well as the \verb/PUSH/ instruction, require special
treatment when they are symbolically executed, which is explained in \secref{sec:calculus}.
All other onedimensional functions serve an additional purpose:
They will later be reused in the Dynamic Logic to define most of the terms for the logic, which,
unlike all other functional instructions, also gives them a universal symbolic execution scheme.

The function types of these instructions is given in \listref{func-type}.
We employ some Agda pattern synonyms for short lists \verb|[ a ] [ a / b ]|
of one or two elements.
\verb/⟦ ty ⟧/ is the set of values of type \verb/ty/.
Notice, that there are two definitions of \verb/ADD/ for naturals and for mutez tokens.
This is necessary for our intrinsic typing scheme, but since the typing rule for each instruction
of a contract is unambiguous, our implementation does not loose expressiveness by this restriction.
% out implementation of Michelson does not loose expressiveness

%% listing ruler max width ------------------------------------------------|?X
\begin{figure}[tp]
\begin{minted}{agda}
data 1-func : Stack → Type → Set where
  ADDnn  :                1-func  [ base   nat / base   nat ]  (base   nat)
  ADDm   :                1-func  [ base mutez / base mutez ]  (base mutez)
  CAR    :  ∀ {t1 t2}  →  1-func               [ pair t1 t2 ]           t1
  CDR    :  ∀ {t1 t2}  →  1-func               [ pair t1 t2 ]           t2
  PAIR   :  ∀ {t1 t2}  →  1-func                  [ t1 / t2 ]  (pair t1 t2)
  NIL    :  ∀  ty      →  1-func                           []  (list    ty)
  NONE   :  ∀  ty      →  1-func                           []  (option  ty)
  SOME   :  ∀ {ty}     →  1-func                       [ ty ]  (option  ty)
  CONS   :  ∀ {ty}     →  1-func             [ ty / list ty ]  (list    ty)
  TRANSFER-TOKENS : ∀ {ty pt}
            →  1-func  [ ty / base mutez / contract {ty} pt ]          ops

data m-func : Stack → Stack × Type → Set where
  UNPAIR  :  ∀ {t1 t2}  →  m-func  [ pair t1 t2 ]   ([ t1 ] , t2)
  SWAP    :  ∀ {t1 t2}  →  m-func     [ t1 / t2 ]   ([ t2 ] , t1)
  DUP     :  ∀ {ty}     →  m-func          [ ty ]   ([ ty ] , ty)

data func-type : Stack → Stack × Type → Set where
  D1   : ∀ {res  args} → 1-func args res      →  func-type args ([] , res)
  Dm   : ∀ {args ress} → m-func args ress     →  func-type args       ress
  PUSH : ∀ {ty}        → Pushable ty → ⟦ ty ⟧ →  func-type []   ([] ,  ty)
\end{minted}
\caption{Function types}
\label{func-type}
\end{figure}

Besides those, \listref{env-func} defines functions that provide information about the
current execution environment,
like the amount of tokens that was transfered when starting the execution,
or the contract stored at a given address.
Their typing scheme does not differ from \verb/func-type/ functions,
but since their execution requires this information,
they have to be executed differently and will be given their own category.

\begin{figure}[tp]
\begin{minted}{agda}
data env-func : Stack → Type → Set where
  AMOUNT    :             env-func             []               (base mutez)
  BALANCE   :             env-func             []               (base mutez)
  CONTRACT  :  ∀ {ty} P → env-func  [ base addr ] (option (contract {ty} P))
\end{minted}
\caption{Functions for blockchain operations}
\label{env-func}
\end{figure}

Control flow instructions are those that take subprogram as arguments.
Their execution is defined in big step semantics,
and since their subprograms can be arbitrarily complex,
their execution requires additional features, which are explained in \chapref{chap:refImpl}.

Their typing rules are defined in the \verb/Instruction/ datatype in \listref{Instruction},
together with the functional instructions.
The different function types are mapped to their corresponding instructions
that work for any possible stack where the top matches the function argument types.
The Agda pattern \verb|[ x // xs ]| stands for the normal list constructor \verb/x ∷ xs/.
Since \verb/results/ of \verb/func-type/ has the type \verb/Stack × Type/,
it must be mapped to a list before it is concatenated with the substack.
\verb/DROP/ is the only 0-dimensional function and therefore provided with its own constructor.

%% listing ruler max width ------------------------------------------------|?X
\begin{figure}[tp]
\begin{minted}{agda}
data Instruction where
  enf       : ∀ {args result S}
            → env-func args result
            → Instruction  (       args ++ S )       [ result // S ]
  fct       : ∀ {args results S}
            → func-type args results
            → Instruction  (       args ++ S )  ([× results ] ++ S)
  DROP      : ∀ {ty S}
            → Instruction  [         ty // S ]                   S
  ITER      : ∀ {ty S}
            → Program      [         ty // S ]                   S
            → Instruction  [    list ty // S ]                   S
  DIP       : ∀ {S Se} n → {T (n ≤ᵇ length S)}
            → Program              (drop n S)                    Se
            → Instruction                  S        (take n S ++ Se)
  IF-NONE   : ∀ {ty S Se}
            → Program                      S                     Se
            → Program      [         ty // S ]                   Se
            → Instruction  [  option ty // S ]                   Se
\end{minted}
\caption{Intrinsically typed Michelson instructions}
\label{Instruction}
\end{figure}

\listref{simple-example-our} shows the example program from \listref{simple-example}
programmed in our Michelson representation.
\begin{figure}[tp]
\begin{minted}{agda}
example : Program [ pair (base nat) (base nat) ] [ pair (list ops) (base nat) ]
example = fct (Dm UNPAIR) ;
          fct (D1 ADDnn) ;
          fct (D1 (NIL ops)) ;
          fct (D1 PAIR) ; end
\end{minted}
\caption{Example program in our Michelson representation}
\label{simple-example-our}
\end{figure}


\begin{comment}
The biggest and most importand subcategory of functional instructions is that of
those instructions whose typing rule 
These are all onedimensional functions whose execution only requires knowledge
of the current stack values that serve as the arguments of these functions,
like \verb/ADD/, \verb/PAIR/ or \verb/NIL ty/ from the example above.
However there is one exception: \verb/PUSH/ technically matches these requirenments,
but its symbolic execution will require some special treatment as we will see later.
Therefore it is given its own category.
Together with 

They will be used to represent certain terms of their result type
We will give these subcategories and they motivation in the same order they are defined in the
\model{02-Functions-Interpretations} module before explaining the \verb/Instruction/ datatype.

The biggest subcategory is that of onedimensional functions whose implementation
only requires the knowledge of 

% Functions to retrieve information from the execution environment,
% will require these additional informations besides the current state of the stack.
Control flow instructions are those that take subprogram as arguments.
Their execution is defined in big step semantics,
and since these subprograms can be arbitarily complex, their execution is a bit involved
and will be discussed in \chapref{chap:refImpl}.
% Their typing rules however are straight forward and can be easily implemented.

The first subcategory of functional instructions contains instructions 
to retrieve blockchain related information,
like the amount of tokens that was transfered when starting the current execution,
or the contract stored at a given address.
Their typing rules don't differ from other functional instructions,
but will be defined separately since their execution requires these informations
from the current execution environment.
All other functional instructions operate on the current stack values alone
and are further divided:
Onedimensional functions have a special role since they will later be reused in the
Dynamic Logic to define most of its terms, which will also make their symbolic execution
straight forward.
Multidimensional functions (like \verb/UNPAIR/ in the example above) need to be treated
individually during symbolic execution.
\verb/PUSH/ is technically a onedimensional function,
but its symbolic execution requires a special treatment as we will see later,
so it is given its own cathegory as well.

All functional instructions only require the top of the input stack to match the
required argument types and leave the rest of the stack unchanged.
This makes it convenient to define each functional subcategory by their argument and result types
and extend that typing when constructing instructions from them.

are represented separately since their execution
are given its own datatype \verb/env-func/ in the modell.
All other functional instructions operate only on the current stack values


into functions that only require the current
stack values to calculate its results, and functions
whose function operates solely on the values from the current stack
and those that 


\subsection{Design rational}\label{sec:desigr}
The guiding idea for out code design was reusability of typing rules
for the reference implementation in the dynamic logic
to make the soundness proofs as concise and easy as possible.

While the reference implementation will work on a stack of concrete values that are
provided for and manipulated during execution,
the DL will operate on a stack of variables for such values that may or may not be
accompanied by formulas that assign values to them.

So the Michelson instructions are defined by their typing rules,
and during concrete execution those typing rules will be matched to their corresponing
implementation rules (semantic rules),
while during symbolic execution a new variable will be introduced and a formula added
that desribes its value by using its typing rule.

So for the \verb=ADD= instruction, the typing rule would be
\[			\text{nat} :: \text{nat} :: \text{\emph{remStack}}
	\Longrightarrow		      \text{nat} :: \text{\emph{remStack}}
\]
which will be mapped to its natural implementation $+ : \bN \to \bN \to \bN$ for concrete execution
as well as to a formula that describes a variable of type $\bN$ as being $\text{ADD} x y$
for two variables $x$ and $y$ of type $\bN$.

Notice that in Michelson some instructions, like \verb=ADD=, are not monomorphic.
However (due to Michelson being \emph{strongly typed}),
because the input and output stacks for smart contracts \emph{are} monomorphic,
the typing rule for every instruction within a given smart contract is fixed.
For that reason, we chose to give a monomorphic reference implementation
instead of
% which eliminates the teadious work of 
matching each instruction with the
typing rule that would be applied for a given contract,
since this reusability scheme applies to the typing rules anyways
and we don't loose generality by giving several add-instructions.
Any given Michelson contract can be compiled to one in our reference implementation
in linear time of its size due to Michelsons strict typing rules.

For \verb=ADD=, we define it for the two typing rules
for those Michelson types we chose to implement, with is \verb=ADDnn= for naturals
and \verb=ADDm= for adding to \verb=mutez= values.

Also, not all instructions represent unary functions \todo{is \emph{unary} correct here?}
like \verb=ADD= does, but only those can be reused for formulas.
So the implemented Michelson instructions are given in certain subcategories
according to their properties.

Of course we want to express in formulas that a given variable has some constant value,
but constant values should be restricted to variables of primitive types.
Constant values for complex types will be represented
with formulas of their according introduction instructions
like \verb=PAIR= for pairs or \verb=CONS= and \verb=NIL= for lists
and more formulas \draft{that fix the variables used there}.
This make reasoning about complex types easier, if not possible in the first place:
In an earlier model where this restriction was not present,
it turned out near impossible to proof that iterating over a list of 3 elements would terminate
without giving concrete values for those 3 elements.
But formulating such conclusions in general without giving the exact values
should be acchievable with a DL.

\subsection{Typing System}\label{sec:typing}

\subsubsection{Types}\label{subsec:types}

The Agda module \todo{link to github} \verb=01-Types= defines a data type \verb=BaseType=
to represent the primitive data types that have been implemented:
\begin{figure}[tp]
\begin{minted}{agda}
data BaseType : Set where
  unit  : BaseType
  nat   : BaseType
  mutez : BaseType
  addr  : BaseType
\end{minted}
\caption{Basic Types}
\label{BaseType}
\end{figure}

A Michelson type in our model is either a \verb=BaseType= or a complex type,
which can be a higher order type like \verb=pair=, \verb=list= or \verb=option=
or blockchain specific types like an \verb=operation= (abbreviated to \verb=ops=)
or a \verb=contract=:
\begin{figure}[tp]
\begin{minted}{agda}
data Type where
  ops          :               Type
  base         : BaseType    → Type
  pair         : Type → Type → Type
  list         : Type        → Type
  option       : Type        → Type
  contract     : ∀ {ty} → Passable ty → Type
\end{minted}
\caption{Michelson Types}
\label{Type}
\end{figure}

For correct implementation of the typing restrictions the 3 predicates
\verb=Pushable=, \verb=Passable= and \verb=Storable= are defined.
There are a lot more restrictions present in Michelson, but only these are needed
for the types implemented here.

After defining the \verb=DecidableEquality= operator on \verb=Types=,
a \verb=Stack= is defined to be a list of \verb=Types=.
To more easily differentiate between the types and values of a Michelson stack
as well as a Stack of DL variables that represent the program stack in the DL,
we will use these following terms:
\begin{description}
	\item[Stack]
		unless stated explicitly, \textbf{Stack} will always referre
		to a Stack of Michelson types (rather than a stack of values
		as in most documents regarding Michelson)
	\item[Interpretation]
		will mean a stack of values, corresponding to the \verb=Stack= indexed
		data type \verb=Int= in the module \verb=02-Functions-Interpretations=
	\item[Matching]
		lastly will mean a stack of DL variables representing a program stack
		during symbolic execution. It corresponds to the \verb=Match= data type
		defined in \verb=11-abstract-representation-and-weakening=,
		which is also indexed by a \verb=Stack=
\end{description}
The Agda constructors for the latter two work exactly like Agda List constructors
and the standard list operations \verb=take=, \verb=drop= and \emph{concatenation}
are implemented for those the same way as for lists,
together with operators to retrieve the top or bottom of an Interpretation or Matching
if they are indexed by a concatenation of two Stacks.

\subsubsection{Functions}\label{subsec:functions}

The module \verb=02-Functions-Interpretations= first defines the different kind of functions
that will be used:
\begin{itemize}
	\item	one-dimensional functions are the most common and those that will be
		reused for formulas. \\
		they are parameterized by the \verb=Stack= of required input types
		and the \emph{single} resulting output type (see \listref{1-func}).
	\item	there are some multidimensional functions (\listref{m-func}).
		% (\listref{m-func} \ldots making the second parameter a product of
		% \verb=Stack= and \verb=Type= enables ).
		The reference implementation won't make a difference between these
		(that's also why the second parameter of \verb=m-func= must be
		the product of \verb=Stack= and \verb=Type=, although only a \verb=Stack=
		should suffice)
		but for the DL implementation special cases are needed here.
	\item	the functions so far work on the current stack alone,
		but instructions for blockchain operations require informations
		about the current blockchain environment they are executed in.
		They don't differ from \verb=1-func= regarding typing
		but have to be treated differently during execution later on (listref{env-func}).
	\item	\emph{there is a 2-dimensional blockchain operation which was'n implemented
		in this}%  model and will require some extensions or reworking of it}
\end{itemize}

All functions that work on stack elements alone are combined in the \verb=func-type=
which also includes the \verb=PUSH= instruction (\listref{func-type}).
It's a onedimensional function regarding its typing rule, but Michelson allows to push
compounded types that can't necessarily be expressed by a single formula,
so it is separated out from the other function types for special care during symbolic execution.

%% listing ruler max width ------------------------------------------------|?X
\begin{figure}[tp]
\begin{minted}{agda}
data func-type : Stack → Stack × Type → Set where
  D1   : ∀ {args res} → 1-func args res      →  func-type args ([] , res)
  Dm   : ∀ {args res} → m-func args res      →  func-type args       res
  PUSH : ∀ {ty}       → Pushable ty → ⟦ ty ⟧ →  func-type []   ([] ,  ty)
\end{minted}
\caption{functions for blockchain operations}
\label{func-type}
\end{figure}


\subsubsection{Shadow Stack and extended Instructions/Programs}\label{subsec:shadow}

\verb=DIP= and \verb=ITER= are special in that they need a second stack to be executed:
During the execution of the sub-program of \verb=DIP= the top \verb=n= elements must be stored
to be later retrieved when sub-program execution has terminated.
Likewise for \verb=ITER=, which consumes the list at the top of the stack
by executing its sub-program for every list element.
Here the currently remaining list has to be stored during sub-program execution.
Since the sub-programs can also contain such instructions that need to store data away for later,
a second stack is needed, as well as new instructions to operate on it:
To execute \texttt{DIP n prg ; \emph{remainingProg}}, the top \verb=n= elements of the
main stack are transfered to this second stack (we call it the \emph{shadow stack}) and
the instruction is replaced by \verb=prg= followed by the \emph{shadow instruction} \verb=DIP'=
which retrieves them from the shadow stack and puts them back onto the mainstack
before continuing execution with \texttt{\emph{remainingProg}}.
For the instruction \verb=ITER prg= the list on top of the main stack will be placed
onto the shadow stack and the instruction will be replaced by its shadow version \verb=ITER' prg=
which does all the actual work: It checks if the list at the top of the shadow stack is empty.
If so, it will be dropped and execution continues.
If not, the first element in the list will be moved to the top of the main stack and
\verb=prg= will be executed. After that \verb=ITER'= is executed again to check the list
until all elements have been iterated over.

These new shadow instructions must therefore be parameterized by 4 stacks:
main input stack, shadow input stack, main output stack and shadow output stack.
Analogous to the abbreviations in our code we will call the main stack \emph{real stack}.

Shadow programs are programs containing ``real'' and shadow instructions (see \listref{shadow}).

%% listing ruler max width ------------------------------------------------|?X
\begin{figure}[tp]
\begin{minted}[linenos]{agda}
data ShadowInst : Stack → Stack → Stack → Stack → Set where
  ITER'     : ∀ {ty rS sS}
            → Program      [ ty // rS ]                              rS
            → ShadowInst           rS   [ list ty // sS ]            rS  sS
  DIP'      : ∀ top {rS sS}
            → ShadowInst           rS        (top ++ sS )    (top ++ rS) sS

data ShadowProg : Stack → Stack → Stack → Stack → Set where
  end  : ∀ {rS sS} → ShadowProg rS sS rS sS
  _;_  : ∀ {ri rn si ro so}
       → Instruction ri     rn
       → ShadowProg  rn si  ro so
       → ShadowProg  ri si  ro so
  _∙_  : ∀ {ri si rn sn ro so}
       → ShadowInst  ri si  rn sn
       → ShadowProg  rn sn  ro so
       → ShadowProg  ri si  ro so
\end{minted}
\caption{shadow instructions and programs}
\label{shadow}
\end{figure}

Operators to concatenate programs and shadow programs are given in the canonical way:
\begin{figure}[tp]
\begin{minted}{agda}
_;;_ : ∀ {Si So Se} → Program Si So → Program So Se → Program Si Se
_;∙_   : ∀ {ri rn si ro so}
       → Program ri rn → ShadowProg rn si ro so → ShadowProg ri si ro so
\end{minted}
\caption{program concatenations}
\label{concat}
\end{figure}

\end{comment}

%%% Local Variables:
%%% mode: latex
%%% TeX-master: "itp2024"
%%% End:
