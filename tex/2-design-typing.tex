\section{Michelson}
\label{sec:michelson}
\label{sec:Mtype}


Michelson~\cite{michelson,devres} is the native language for smart contracts on the Tezos blockchain.
It is a low level, stack-based, simply-typed, purely functional programming
language. That is, all computation is driven by transforming an input
stack into an output stack. There are no mutable data structures;
blockchain transactions are handled outside of Michelson.
All contracts are statically typed to avoid run-time type errors.

Each Michelson instruction transforms a given input stack into an output stack
where some of its values have been changed, added, or removed.
For example, the \verb=ADD= instruction accepts any stack
whose two topmost elements are numbers,
and returns a stack where these two values have been replaced by their sum. The remaining stack is unchanged.
\begin{align*}
	\text{ADD} \quad:\quad 15 :: 27 :: \text{\emph{remainingStack}}
	\mapsto	           42 :: \text{\emph{remainingStack}}
\label{equ:ADD}
\end{align*}

\subsection{Types}
\label{sec:michelson-types}

Michelson supports the usual data types like numbers, pairs, and lists as well as
some blockchain-specific types for tokens and contracts. 
Figure~\ref{fig:Type} contains Agda definitions for a select subset of
Michelson types {\AType}. As some base types can be treated alike later on, we
represent them with a separate type {\ABaseType}.

\begin{figure}[tp]
  \begin{subfigure}{0.48\textwidth}
    \noindent
    % \TypesType
    % \TypesPatterns 
    \includegraphics[scale=0.45]{figures/TypesType}
    \caption{Syntax}
    \label{fig:Type}
  \end{subfigure}
  \begin{subfigure}{0.48\textwidth}
    % \TypesSemantics
  \includegraphics[scale=0.45]{figures/TypesSemantics}
  \caption{Semantics}
  \label{fig:Type-Semantics}
\end{subfigure}
\caption{Michelson Types}
\label{Type}
\end{figure}

Most types' names are self explanatory. The base type {\ABmutez} stands
for tokens, {\Aaddr} stands for blockchain addresses in  Tezos. We introduce
shorthand patterns for base types for readability.
The type {\Aoperation} consists of blockchain operations that can
be emitted during contract execution. This mechanism implements token
transfers from the current contract to other accounts or contracts.
The type {\Acontract~\AP} represents such a contract
which accepts a parameter of type {\Aty} represented by {\AP : \APassable~\Aty}.
The type predicate {\TypesPassable} originates from the Michelson
specification and characterizes types that can be passed as parameters
to contracts. Its declaration is mutually recursive with {\AType}.

The semantics of types is defined by a mapping to Agda types. Most
Michelson types have  obvious Agda counterparts, except {\Aaddr},
{\Acontract}, and {\Aoperation}.  Addresses and contracts are both
represented by natural numbers. The difference is that a value of type
{\Acontract} is known to be a valid address of a contract of suitable
type. We only define one alternative of the 
{\AOperation} datatype: {$\AtransferTokens~v~m~c$}, which models a
token transfer to contract $c$ while passing the parameter value $v$
and tokens $m$.\footnote{At the time of writing this paper, full Michelson also
  supports the operations \texttt{CREATE-CONTRACT}, \texttt{EMIT}
  (deliver an event to an external application), and
  \texttt{SET-DELEGATE} (delegate stakes to another account).}

\subsection{Programs and Instructions}
\label{sec:michelson-programs}


Michelson programs are intrinsically typed, that is, only well-typed
programs can be written. Accordingly, they are represented in
Agda by a datatype {\AProgram} indexed by the types on the input stack
and the types on the output stack. We assume that \TypesStack.
% \SyntaxProgram
\begin{flushleft}
  \includegraphics[scale=0.45]{figures/SyntaxProgram}
\end{flushleft}

Instructions are indexed in the same way:
If  instruction \verb/inst/ maps an input stack of type $Si$ to an
output stack of type $So$
and \verb/prg/ maps that output stack $So$ to the final stack of type $Se$,
then \verb/inst ; prg/ is a program that maps $Si$ to $Se$.
The empty program {\Aend} does not transform the stack.

\begin{figure}[t]
  % \SyntaxInstruction
  \includegraphics[scale=0.37]{figures/SyntaxInstruction}
  \caption{Instructions of Core Michelson}
  \label{fig:core-michelson-instructions}
\end{figure}
We discuss a representative subset of Michelson instructions shown in Figure~\ref{fig:core-michelson-instructions}.
The definition of {\AInstructionPlus} implements the pattern that most
instructions only transform a fixed number of elements on top of the input stack and are parametric in the rest. 

The first group of instructions operates on a fixed number of values on
the stack and pushes the result. All arithmetic operations belong to
this group and we just give two examples, {\AADDnn}  and {\AADDm},
which perform addition of natural numbers and tokens, respectively.
Michelson language overloads arithmetic operators, but as
overloading is not supported by Agda, we supply separate
instructions. We come back to this issue at the end of this section. 

\ACon{CAR}, \ACon{CDR}, and \ACon{PAIR} are the standard operations on
pairs. \ACon{NONE} and \ACon{SOME} are the constructors for the
\ACon{option} datatype, and \ACon{NIL} and \ACon{CONS} construct
lists. The constructors for ``empty'' containers, \ACon{NONE} and
\ACon{NIL} are indexed by the element type, otherwise that type can be
inferred from the context.

The last instruction in this group is \ACon{TRANSFER-TOKENS}. Despite
the name, this instruction does \textbf{not} directly transfer tokens
to another account. It rather constructs a value
{$\AtransferTokens~v~m~c$} of type {\Aoperation} from its arguments.

The instructions in the next group differ in that they push zero or more
values on the output stack. \ACon{DROP} pops the stack, \ACon{DUP} duplicates the top of the
stack, \ACon{SWAP} swaps the top entries, and \ACon{UNPAIR} eliminates
a pair and pushes its contents. \ACon{UNPAIR} is a convenience
instruction as it is equivalent to the instruction sequence
\ACon{DUP}; \ACon{CDR}; \ACon{SWAP}; \ACon{CAR}. 

The next group contains instructions that are blockchain
specific. \ACon{AMOUNT} returns the tokens that were transferred with
the currently running contract invocation and \ACon{BALANCE}  returns
the tokens currently owned by it. The \ACon{CONTRACT} instruction is
indexed by a type $t$ that must be {\APassable}. It takes an address
and checks on the blockchain whether this address is associated with a
contract that accepts arguments of type $t$. The result is
communicated as an \ACon{option} type. That is, the {\Acontract} type
carries a verified address.

The \ACon{PUSH} instruction pushes a value of type $t$ on the stack. The
value is encoded by a type-indexed datatype \AgdaDatatype{Data} for
pushable values. We elide its straightforward definition.

The last group of instructions showcases control structures and an
instruction that operates in a non-uniform way on the stack. The
instruction \ACon{IF-NONE} eliminates a value of \ACon{option} type from the
top of the stack. Its parameters are programs that implement the branches for case \ACon{None}
and \ACon{Some}. The latter takes the value wrapped in the \ACon{Some}
constructor as an argument on top of the stack.

The instruction \ACon{ITER} runs a sub-program on every element of its
argument list. The instruction $\ACon{DIP}~n$ runs a sub-program at
depth $n$ on the input stack, that is, it skips over the first $n$
elements of the stack, runs the sub-program, and reattaches those
elements. The extra machinery in the implicit argument of the
instruction makes sure that there are at least $n$ elements on the
stack. This mechanism is called reflection in the  PLFA textbook
\cite{plfa}.

Earlier, we remarked that Agda does not allow overloading of
constructors in the same datatype. However, we can use reflection to
define a ``smart constructor'' that almost suits the purpose.
% \SyntaxOverloading
\begin{flushleft}
  \includegraphics[scale=0.45]{figures/SyntaxOverloading}
\end{flushleft}

The definition exploits the fact that the input stack of an
instruction is always known in a Michelson program. The same fact also
enables overloading in Michelson's implementation to work.
The function \AFun{overADD} specifies the resolution of overloading
for the \AFun{ADD} instruction. If the argument types are both
\ACon{nat}, then the result type is \ACon{nat} and the chosen instruction is
\ACon{ADDnn}; and so on.\footnote{The full Michelson language has ten different
  overloadings of \AFun{ADD}.} If no  overloading is known for a combination
of arguments, the function returns \ACon{nothing}.
The smart constructor \AFun{ADD} takes a proof that the overloading is
defined on a given pair of input types. Then it extracts the selected
instruction from the overloading.

Compared to ``real'' Michelson, the smart constructor requires an
extra argument to work:
\SyntaxOverloadingExample

\subsection{Blockchain Interface}
\label{sec:blockchain-interface}

A contract on the Tezos blockchain is indexed by a parameter type $p$
and a store type $s$. The type $p$ must be {\APassable} and the type
$s$ must be {\AStorable}. Moreover, each contract comes with a current
balance of tokens and a store of type $s$. The implementation of the
contract is a program that maps a $\ACon{pair}~p~s$ to a
$\ACon{pair}~(\ACon{list}~\Aoperation)~s$, that is, it consumes the
parameter paired with the current store and produces a list of operations (e.g., to invoke further
contracts) paired with the updated store. The program itself is pure;
any side effects, i.e., store update and contract calls, are managed
by the blockchain runtime.
\ConcreteContract

The $Mode$ argument abstracts over the semantics of types. Its type
has three components, one $\mathcal{M}$ for the semantics and the others,
$\mathcal{F}$ and $\mathcal{G}$, are used by the abstract semantics in \autoref{sec:abstract-states}.
\ConcreteMODE
Its instantiation for the concrete semantics installs the 
standard semantics of types from Section~\ref{sec:michelson-types}. The
remaining components are instantiated to the unit type $\top$.
\ConcreteCMode

With this definition, the contract store of the blockchain is just a partial mapping from addresses
to contracts.
\ConcreteBlockchain

To start executing a contract, we initiate a blockchain transaction to its
address, i.e., we ask the blockchain runtime to transfer tokens to its
address along with its parameter.
Once a contract has terminated, the runtime updates the
stored value and processes the list of operations.

On the Tezos blockchain a normal account with deposit $init$
corresponds to a
contract with a unit parameter, unit store, and a trivial program that issues no operations.
\ConcreteAccount



%%% Local Variables:
%%% mode: latex
%%% TeX-master: "itp2024"
%%% End:
