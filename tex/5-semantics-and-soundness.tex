\section{Semantics and Soundness}
\label{sec:semantics-soundness}

\subsection{Values and models}\label{sec:values&models}
% For the semantics \todo{check for precise usage of semantics \ldots} we 

Since the context is just a list of \verb/Type/s like a stack, the \verb/Int/ data type
is reused as an Interpretation for a given context as well.

For a given context interpretation, variable and term values are defined in an obvious way
(see \listref{values}; \verb/IMI/ maps a (stack) matching to its (stack) interpretation
for the given context interpretation).

%% listing ruler max width ------------------------------------------------|?X
\begin{listing}[!ht]
\begin{minted}{agda}
val∈ : ∀ {ty Γ} → Int Γ → ty ∈ Γ → ⟦ ty ⟧
val∈ (x ∷ I) (here refl) = x
val∈ (x ∷ I) (there x∈)  = val∈ I x∈

IMI : ∀ {Γ S} → Int Γ → Match Γ S → Int S
IMI γ [M] = [I]
IMI γ (v∈ ∷ M) = (val∈ γ v∈) ∷ (IMI γ M)
  
val⊢ : ∀ {ty Γ} → Int Γ → Γ ⊢ ty → ⟦ ty ⟧
val⊢ γ (const x) = x
val⊢ γ (func d1f Margs) = appD1 d1f (IMI γ Margs)
val⊢ γ (var v∈) = val∈ γ v∈
val⊢ γ (contr {P = P} adr) = adr
val⊢ γ (m₁∈ ∸ₘ m₂∈) = val∈ γ m₁∈ ∸ val∈ γ m₂∈
\end{minted}
\caption{Values of variables and terms}
\label{values}
\end{listing}

For a given context interpretation \verb/γ/ and abstract and concrete (program or execution) states,
the predicates \verb/modρ/ and \verb/modσ/ express that
under this interpretation the given abstract state models the concrete state.
This is the case when the formulas in \verb/Φ/ are true under \verb/γ/ and
the real and variable values are the same for the stacks and every other element.

They are implemented by several \verb/mod/$X$ predicates for every
subcomponent of program and execution states.

To show that a given disjunction of states models a concrete state,
you show that one of the states in the disjunction models the state:

\begin{listing}[!ht]
\begin{minted}{agda}
mod⊎σ : ∀ {Γ} → Int Γ → ⊎Exec-state → Exec-state → Set
mod⊎σ {Γ} γ ⊎σ σ = ∃[ ασ ] (Γ , ασ) ∈ ⊎σ × modσ γ ασ σ
\end{minted}
\caption{Modeling disjunction of states, e.g., for execution states}
\label{modUsigma}
\end{listing}

% \draft{maybe also show how formulas of states are modelled? \ldots or maybe i don't ;p}

\subsection{Soundness of the DL}\label{soundness}

The soundness of the calculus is prooven by showing that
when an abstract state models a concrete one,
the result from symbolic execution will model the result from concrete execution.

The types for these proof terms are listed in \listref{soundness}.

%% listing ruler max width ------------------------------------------------|?X
\begin{listing}[!ht]
\begin{minted}{agda}
soundness : ∀ {Γ ro so} γ αρ ρ → modρ {Γ} {ro} {so} γ αρ ρ
          → ∃[ Γ` ] ∃[ γ` ]
            mod⊎ρ {Γ` ++ Γ} (γ` +I+ γ) (αprog-step αρ) (prog-step ρ)

soundness : ∀ {Γ : Context} γ ασ σ → modσ {Γ} γ ασ σ
          → Exec-state.MPstate σ ≡ nothing
          ⊎ ∃[ Γ` ] ∃[ γ` ]
            mod⊎σ {Γ` ++ Γ} (γ` +I+ γ) (αexec-step ασ) (exec-step σ)

soundness : ∀ {Γ ro so γ αρ Γ` αρ`}
          → αρ-special αρ (Γ` , αρ`)
          → (ρ : Prog-state {ro} {so})
          → modρ {Γ} {ro} {so} γ αρ ρ
          → ∃[ γ` ] modρ (γ` +I+ γ) αρ` (prog-step ρ)

soundness : ∀ {Γ γ ασ ⊎σ`}
          → ασ-special ασ ⊎σ`
          → (σ : Exec-state)
          → modσ {Γ} γ ασ σ
          → ∃[ Γ` ] ∃[ γ` ] mod⊎σ {Γ` ++ Γ} (γ` +I+ γ) ⊎σ` (exec-step σ)
\end{minted}
\caption{Soundness terms}
\label{soundness}
\end{listing}

% XXX =\module{23-prog-step-soundness}
The first term from XXX proves soundness of \verb/αprog-step/.
Since modeling is mostly done by proving equality,
when supplying the arguments sufficiently precise to match the cases in the definition
of \verb/αprog-step/, the proof is obvious for Agda and only requires
correct placements of \verb/refl/ and some parts of the arguments,
as well as showing that the weakened parts of the formula
are still modeled with the extended context.
An example is given in \listref{ps-sound-IF-NONE}.

%% listing ruler max width ------------------------------------------------|?X
\begin{listing}[!ht]
\begin{minted}{agda}
soundness γ (αstate αen (IF-NONE thn els ; prg) (o∈ ∷ rVM) sVM Φ)
             (state en .(IF-NONE thn els ; prg) (just x ∷ rSI) sSI)
             (refl , refl , mE , refl , (o≡ , mrS) , msS , mΦ)
  = _ , x ∷ [I] , _ , 1∈ , (refl , refl , wkmodE mE , refl ,
    (refl , wkmodS mrS) , wkmodS msS , (o≡ , wkmodΦ mΦ))
\end{minted}
\caption{Prog-step soundness example for IF-NONE}
\label{ps-sound-IF-NONE}
\end{listing}

The most complicated proof in this module is also the most powerful one, because it proves
the soundness for any onedimensional function.
It is given in \listref{ps-sound-D1} and works by showing that applying the top of the previous
stack interpretation to the given function yields the same result as applying
the extended interpretation of the top of the previous stack matching to it.

%% listing ruler max width ------------------------------------------------|?X
\begin{listing}[!ht]
\begin{minted}{agda}
soundness γ (αstate αen (fct (D1 x) ; prg) rVM sVM Φ)
             (state en .(fct (D1 x) ; prg) rSI sSI)
             (refl , refl , mE , refl , mrS , msS , mΦ)
  with modS++ rVM rSI mrS
... | mtop , mbot = _ , appD1 x (Itop rSI) ∷ [I] , _ , 0∈ , (refl , refl ,
    wkmodE mE , refl , (refl , wkmodS mbot) , wkmodS msS ,
    cong (appD1 x) (trans (sym (modIMI mtop))
                          (wkIMI {γ` = appD1 x (Itop rSI) ∷ [I]})) ,
    wkmodΦ mΦ)
\end{minted}
\caption{Prog-step soundness of onedimensional functions}
\label{ps-sound-D1}
\end{listing}

% XXX =\module{25-exec-step-soundnes}s
The second term from \listref{soundness} is implemented in XXX
and proves the soundness for those cases of \verb/αexec-step/
where a contract execution is active.
This is conceptually quite simple since it only containes two cases:
Either the current contract execution has terminated and we have to proof that
the blockchain is updatet correctly, which can be achieved with sufficient
argument precision an in most cases of the soundness proof for \verb/αprog-step/.
For all other cases, we reuse the soundness proof for program state execution.
Unfortunately we still have to supply Agda with all relevant details about the program state
so it can instanciate the result from that proof, but the proof term is always exactly the
same except for the context extension of that case,
as well as which clause of the disjunction models the result, which is sometimes the second clause.
\listref{exec2progstep-soundness} gives an example: To be able to instanciate \verb/px/ 
in lines 13 and 22 with \verb/refl/,
we show that extending \verb/Γ/ with the correct context extension for the
particular case is equal to \verb/Γ`/.

%% listing ruler max width ------------------------------------------------|?X
\begin{listing}[!ht]
\begin{minted}[linenos]{agda}
soundness {Γ} γ 
  ασ@(αexc αccounts 
          (inj₁ (αpr {s = s} αcurrent αsender
                     αρ@(αstate αen (IF-NONE thn els ; prg) 
                                    (v∈ ∷ rVM) sVM Φ)))
          αpending)
  σ@(exc accounts (just (pr current sender ρ)) pending)
  ( mβ
  , ( refl , refl , refl , refl , mc , ms
    , mρ@(refl , refl , mE , refl , mρrest))
  , mp)
  with ρ-sound γ αρ ρ mρ
... | Γ` , γ` , _ , here px , mρ`
  with ++-cancelʳ Γ Γ` [] (,-injectiveˡ px)
soundness γ ασ σ (mβ , ( refl , refl , refl , refl , mc , ms , mρ) , mp)
  | _ , γ` , _ , 0∈ , mρ` | refl
  = inj₂ ( _ , γ` , _ , 0∈ , (wkmodβ mβ) , (refl , refl , refl , refl
         , wkmodC {γ` = γ`} mc , (wkmodC {γ` = γ`} ms) , mρ`) , wkmodp mp)
soundness {Γ} γ ασ@(αexc αccounts (inj₁ (αpr {s = s} αcurrent αsender
   αρ@(αstate αen (IF-NONE {ty} thn els ; prg) (v∈ ∷ rVM) sVM Φ))) αpending)
  σ (mβ , ( refl , refl , refl , refl , mc , ms , mρ) , mp)
  | Γ` , γ` , _ , there (here px) , mρ`
  with ++-cancelʳ Γ Γ` [ ty ] (,-injectiveˡ px)
soundness γ ασ σ (mβ , ( refl , refl , refl , refl , mc , ms , mρ) , mp)
  | _ , γ` , _ , 1∈ , mρ` | refl
  = inj₂ ( _ , γ` , _ , 1∈ , (wkmodβ mβ) , (refl , refl , refl , refl
         , wkmodC {γ` = γ`} mc , (wkmodC {γ` = γ`} ms) , mρ`) , wkmodp mp)
\end{minted}
\caption{Proofing exec step soundness with prog step soundness}
\label{exec2progstep-soundness}
\end{listing}

The proof of the special cases for program state transitions is similar in that
almost every case has the same structure:
Extracting the conjunct from \verb/Φ/ that is used to apply the special case transition
and rewriting by it makes the soundness obvious and only \verb/refl/ pracements necessary.
Only the cases for the \verb/CONTRACT/ instruction need a special rewrite technique
to match the concrete and abstract blockchain lookups,
as well as some handling of cases that are actually impossible for the given transition:

%% listing ruler max width ------------------------------------------------|?X
\begin{listing}[!ht]
\begin{minted}{agda}
soundness (UNPAIR φ∈Φ) (state en _ (p ∷ rSI) sSI)
          (refl , refl , mE , refl , (refl , mrS) , msS , mΦ)
  with modφ∈Φ φ∈Φ mΦ
... | p∈≡p rewrite p∈≡p
  = _ , refl , refl , mE , refl , (refl , refl , mrS) , msS , mΦ

soundness (CTR¬p {p' = p'} φ∈Φ ≡j p≢p') (state en _ (a ∷ rSI) sSI)
          (refl , refl , mE@(mβ , mErest) , refl , (refl , mrS) , msS , mΦ)
  with modφ∈Φ φ∈Φ mΦ | mβ a
... | refl | mMCa
  rewrite ≡j
  with Environment.accounts en a | mMCa
... | just (p , s , c) | refl , mpC
  with p ≟ p'
... | yes refl = ⊥-elim (p≢p' refl)
... | no x
  = (nothing ∷ [I]) , refl , refl , wkmodE mE , refl
  , (refl , wkmodS mrS) , wkmodS msS , (refl , wkmodΦ mΦ)
\end{minted}
\caption{Soundness of special prog state transitions}
\label{prog-step-SC}
\end{listing}

The special program state transitions can also occur during contract chain executions
and make up one of the special cases for execution state transitions.
As with prog state execution within execution state execution,
these special cases are proven by reusing their soundness proof for program state transitions.
The remaining cases are concerned with termination of a contract execution
or the handling of pending operations and require a combination of the techniques used
for special program state transitions on a larger scale.


%%% Local Variables:
%%% mode: latex
%%% TeX-master: "itp2024"
%%% End:
