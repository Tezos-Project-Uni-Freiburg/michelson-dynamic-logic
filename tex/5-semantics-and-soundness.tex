\section{Semantics and Soundness}
\label{sec:semantics-soundness}

\subsection{Values and Models}\label{sec:values&models}
% For the semantics \todo{check for precise usage of semantics \ldots} we 

\begin{figure}[tp]
  \SemanticsTerms
  \SemanticsFormulas
  \caption{Semantics of terms and formulas}
  \label{fig:semantics-terms}
\end{figure}

As a context is just a list of types like a stack, its interpretation
is also a heterogeneous list of values as defined by \ADT{Int}.
For a given context interpretation $\gamma$, the semantics of a term
and a formula is defined as usual
(see \autoref{fig:semantics-terms}).

For a given context interpretation \verb/γ/ and abstract and concrete (program or execution) states,
the predicates \AFun{modρ} and \AFun{modσ} express that
under this interpretation the given abstract state models the concrete state.
This is the case when the formulas in \verb/Φ/ are true under \verb/γ/ and
the real and variable values are the same for the stacks and every
other element.
\SemanticsMODELING
\SemanticsModRho

They all have a similar structure expressed by the \AFun{MODELING}
function as they relate an abstract thing with a concrete thing. They
are implemented by several auxiliary \verb/mod/$X$ predicates for every
subcomponent of program and execution states. For example, \AFun{ModE}
relates execution environments, \AFun{modprg} relates shadow programs,
\AFun{modS} relates stacks, and \AFun{modΦ} checks that the formulas
are all true. The definition of \AFun{modσ} is similar.

To show that a disjunction of abstract states models a concrete state,
we show that one of the states in the disjunction models the state:
\SemanticsModUS

% \begin{listing}[!ht]
% \begin{minted}{agda}
% mod⊎σ : ∀ {Γ} → Int Γ → ⊎Exec-state → Exec-state → Set
% mod⊎σ {Γ} γ ⊎σ σ = ∃[ ασ ] (Γ , ασ) ∈ ⊎σ × modσ γ ασ σ
% \end{minted}
% \caption{Modeling disjunction of states, e.g., for execution states}
% \label{modUsigma}
% \end{listing}

% \draft{maybe also show how formulas of states are modelled? \ldots or maybe i don't ;p}

\subsection{Soundness of the DL}\label{soundness}

We prove the soundness of the logic by showing that
when an abstract state models a concrete one,
the result of one-step symbolic execution models the result from
concrete execution of the same step.
Here are the types of the proof terms for program steps and execution
steps.
\SoundnessProgStep
\SoundnessExecStep


%% listing ruler max width ------------------------------------------------|?X
% \begin{listing}[!ht]
% \begin{minted}{agda}
% soundness : ∀ {Γ ro so} γ αρ ρ → modρ {Γ} {ro} {so} γ αρ ρ
%           → ∃[ Γ` ] ∃[ γ` ]
%             mod⊎ρ {Γ` ++ Γ} (γ` +I+ γ) (αprog-step αρ) (prog-step ρ)

% soundness : ∀ {Γ : Context} γ ασ σ → modσ {Γ} γ ασ σ
%           → Exec-state.MPstate σ ≡ nothing
%           ⊎ ∃[ Γ` ] ∃[ γ` ]
%             mod⊎σ {Γ` ++ Γ} (γ` +I+ γ) (αexec-step ασ) (exec-step σ)

% soundness : ∀ {Γ ro so γ αρ Γ` αρ`}
%           → αρ-special αρ (Γ` , αρ`)
%           → (ρ : Prog-state {ro} {so})
%           → modρ {Γ} {ro} {so} γ αρ ρ
%           → ∃[ γ` ] modρ (γ` +I+ γ) αρ` (prog-step ρ)

% soundness : ∀ {Γ γ ασ ⊎σ`}
%           → ασ-special ασ ⊎σ`
%           → (σ : Exec-state)
%           → modσ {Γ} γ ασ σ
%           → ∃[ Γ` ] ∃[ γ` ] mod⊎σ {Γ` ++ Γ} (γ` +I+ γ) ⊎σ` (exec-step σ)
% \end{minted}
% \caption{Soundness terms}
% \label{soundness}
% \end{listing}

\begin{figure}[tp]
\SoundnessCaseIfNone
  \caption{Prog-step soundness for IF-NONE (excerpt)}
  \label{fig:prog-step-soundness-if-none}
\end{figure}

% XXX =\module{23-prog-step-soundness}
The first \AFun{soundness} statement addresses soundness of \verb/αprog-step/.
As the modeling relation is mostly composed of equalities, the proof
gets accepted by Agda, once we supply sufficiently precise arguments to match the cases in the definition
of \AFun{αprog-step}. We pattern match against \ACon{refl} and parts of the arguments,
as well as we show that the weakened parts of the formula
are still modeled with the extended context (if new variables were
introduced in the case).

\autoref{fig:prog-step-soundness-if-none} shows the case for
the \ACon{IF-NONE} instruction. Without going into details, it is easy
to spot the handling of the concrete and abstract stack and that the
outcome of the test determines which of the possibilities of the
abstract outcome is chosen (cf. $0\in$ and $1\in$).

%% listing ruler max width ------------------------------------------------|?X
% \begin{listing}[!ht]
% \begin{minted}{agda}
% soundness γ (αstate αen (IF-NONE thn els ; prg) (o∈ ∷ rVM) sVM Φ)
%              (state en .(IF-NONE thn els ; prg) (just x ∷ rSI) sSI)
%              (refl , refl , mE , refl , (o≡ , mrS) , msS , mΦ)
%   = _ , x ∷ [I] , _ , 1∈ , (refl , refl , wkmodE mE , refl ,
%     (refl , wkmodS mrS) , wkmodS msS , (o≡ , wkmodΦ mΦ))
% \end{minted}
% \caption{Prog-step soundness example for IF-NONE}
% \label{ps-sound-IF-NONE}
% \end{listing}
\begin{figure}[tp]
  \SoundnessCaseDOne
  \caption{Prog-step soundness for scalar functions (excerpt)}
  \label{fig:ps-sound-D1}
\end{figure}

The most complicated case of this proof establishes soundness for any
scalar function (see \autoref{fig:ps-sound-D1}).
It works by showing that applying the front of the previous
stack interpretation to the given function yields the same result as applying
the extended interpretation of the top of the previous stack matching to it.

%% listing ruler max width ------------------------------------------------|?X
% \begin{listing}[!ht]
% \begin{minted}{agda}
% soundness γ (αstate αen (fct (D1 x) ; prg) rVM sVM Φ)
%              (state en .(fct (D1 x) ; prg) rSI sSI)
%              (refl , refl , mE , refl , mrS , msS , mΦ)
%   with modS++ rVM rSI mrS
% ... | mtop , mbot = _ , appD1 x (Itop rSI) ∷ [I] , _ , 0∈ , (refl , refl ,
%     wkmodE mE , refl , (refl , wkmodS mbot) , wkmodS msS ,
%     cong (appD1 x) (trans (sym (modIMI mtop))
%                           (wkIMI {γ` = appD1 x (Itop rSI) ∷ [I]})) ,
%     wkmodΦ mΦ)
% \end{minted}
% \caption{Prog-step soundness of onedimensional functions}
% \label{ps-sound-D1}
% \end{listing}

% XXX =\module{25-exec-step-soundnes}s
The second \AFun{soundness} statement establishes soundness for those cases of \AFun{αexec-step}
where a contract execution is active.
This part appears simple because it only covers two cases:
Either we are in the middle of running a contract, in which case we
reuse the soundness proof for program state execution, or
the current contract execution has terminated and we have to prove that
the blockchain and the pending list are updated correctly.
The first case is straightforward, but tedious because we need to copy
parts of the previous proof.
The second case is fairly technical as it involves getting the proof in
sync with the definitions of concrete and abstract execution.

% Unfortunately we still have to supply Agda with all relevant details about the program state
% so it can instanciate the result from that proof, but the proof term is always exactly the
% same except for the context extension of that case,
% as well as which clause of the disjunction models the result, which is sometimes the second clause.
% \listref{exec2progstep-soundness} gives an example: To be able to instanciate \verb/px/ 
% in lines 13 and 22 with \verb/refl/,
% we show that extending \verb/Γ/ with the correct context extension for the
% particular case is equal to \verb/Γ`/.

%% listing ruler max width ------------------------------------------------|?X
% \begin{listing}[!ht]
% \begin{minted}[linenos]{agda}
% soundness {Γ} γ 
%   ασ@(αexc αccounts 
%           (inj₁ (αpr {s = s} αcurrent αsender
%                      αρ@(αstate αen (IF-NONE thn els ; prg) 
%                                     (v∈ ∷ rVM) sVM Φ)))
%           αpending)
%   σ@(exc accounts (just (pr current sender ρ)) pending)
%   ( mβ
%   , ( refl , refl , refl , refl , mc , ms
%     , mρ@(refl , refl , mE , refl , mρrest))
%   , mp)
%   with ρ-sound γ αρ ρ mρ
% ... | Γ` , γ` , _ , here px , mρ`
%   with ++-cancelʳ Γ Γ` [] (,-injectiveˡ px)
% soundness γ ασ σ (mβ , ( refl , refl , refl , refl , mc , ms , mρ) , mp)
%   | _ , γ` , _ , 0∈ , mρ` | refl
%   = inj₂ ( _ , γ` , _ , 0∈ , (wkmodβ mβ) , (refl , refl , refl , refl
%          , wkmodC {γ` = γ`} mc , (wkmodC {γ` = γ`} ms) , mρ`) , wkmodp mp)
% soundness {Γ} γ ασ@(αexc αccounts (inj₁ (αpr {s = s} αcurrent αsender
%    αρ@(αstate αen (IF-NONE {ty} thn els ; prg) (v∈ ∷ rVM) sVM Φ))) αpending)
%   σ (mβ , ( refl , refl , refl , refl , mc , ms , mρ) , mp)
%   | Γ` , γ` , _ , there (here px) , mρ`
%   with ++-cancelʳ Γ Γ` [ ty ] (,-injectiveˡ px)
% soundness γ ασ σ (mβ , ( refl , refl , refl , refl , mc , ms , mρ) , mp)
%   | _ , γ` , _ , 1∈ , mρ` | refl
%   = inj₂ ( _ , γ` , _ , 1∈ , (wkmodβ mβ) , (refl , refl , refl , refl
%          , wkmodC {γ` = γ`} mc , (wkmodC {γ` = γ`} ms) , mρ`) , wkmodp mp)
% \end{minted}
% \caption{Proofing exec step soundness with prog step soundness}
% \label{exec2progstep-soundness}
% \end{listing}

\begin{comment}
  The proof of the special cases for program state transitions is
  similar in that almost every case has the same structure: Extracting
  the conjunct from \verb/Φ/ that is used to apply the special case
  transition and rewriting by it makes the soundness obvious and only
  \verb/refl/ pracements necessary.  Only the cases for the
  \verb/CONTRACT/ instruction need a special rewrite technique to
  match the concrete and abstract blockchain lookups, as well as some
  handling of cases that are actually impossible for the given
  transition:

  %% listing ruler max width
  %% ------------------------------------------------|?X
  \begin{listing}[!ht]
\begin{minted}{agda}
soundness (UNPAIR φ∈Φ) (state en _ (p ∷ rSI) sSI)
          (refl , refl , mE , refl , (refl , mrS) , msS , mΦ)
  with modφ∈Φ φ∈Φ mΦ
... | p∈≡p rewrite p∈≡p
  = _ , refl , refl , mE , refl , (refl , refl , mrS) , msS , mΦ

soundness (CTR¬p {p' = p'} φ∈Φ ≡j p≢p') (state en _ (a ∷ rSI) sSI)
          (refl , refl , mE@(mβ , mErest) , refl , (refl , mrS) , msS , mΦ)
  with modφ∈Φ φ∈Φ mΦ | mβ a
... | refl | mMCa
  rewrite ≡j
  with Environment.accounts en a | mMCa
... | just (p , s , c) | refl , mpC
  with p ≟ p'
... | yes refl = ⊥-elim (p≢p' refl)
... | no x
  = (nothing ∷ [I]) , refl , refl , wkmodE mE , refl
  , (refl , wkmodS mrS) , wkmodS msS , (refl , wkmodΦ mΦ)
\end{minted}
\caption{Soundness of special prog state transitions}
\label{prog-step-SC}
\end{listing}

The special program state transitions can also occur during contract
chain executions and make up one of the special cases for execution
state transitions.  As with prog state execution within execution
state execution, these special cases are proven by reusing their
soundness proof for program state transitions.  The remaining cases
are concerned with termination of a contract execution or the handling
of pending operations and require a combination of the techniques used
for special program state transitions on a larger scale.

\end{comment}
%%% Local Variables:
%%% mode: latex
%%% TeX-master: "itp2024"
%%% End:
