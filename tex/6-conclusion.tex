\section{Related Work}
\label{sec:related-work}

Research on formal verification of blockchain-based applications
has experienced rapid growth in the last decade. Various techniques
and frameworks have been applied to enhance the safety of smart
contracts. In this section, we 
discuss some key approaches, particularly those employing symbolic
execution in the context of smart contracts. 
 
\subsection{Verification of Smart Contracts}
\label{sec:verif-smart-contr}


Symbolic execution is a powerful technique for
systematically exploring program paths and identifying potential
vulnerabilities in smart contracts. Most of the existing tools focus
on the Ethereum platform. Tsankov et al. introduced
SECURIFY \cite{securify}, a tool that utilizes symbolic execution to
perform practical security analysis on Ethereum smart contracts. It
targets common vulnerability security patterns specified in a
designated domain-specific language. SECURIFY symbolically encodes the
dependence graph of the contract in stratified Datalog to extract
semantic information from the code. After obtaining semantic facts, it
checks whether the security patterns hold or not. Similarly, Manticore
\cite{manticore} and KEVM \cite{kevm} use symbolic execution to
analyze Ethereum smart contracts. KEVM is an executable formal
specification built with the K Framework for the Ethereum virtual
machine's bytecode (EVM), a stack-based and low-level smart contract
language for the Ethereum blockchain. Since tokens can hold a
significant amount of value, they are often targeted for
attacks. Therefore, several tools \cite{kevm,park} conduct case
studies for the implementations of token standards. 

Several approaches use existing formal verification frameworks to
ensure the correctness and security of smart contracts. Amani et
al. \cite{isabelle} proposed the formal verification of Ethereum smart
contracts in Isabelle/HOL. Hirai \cite{hirai} formalizes the EVM using
Lem, a language to specify semantic definitions. The formal
verification of smart contracts is achieved using the Isabelle proof
assistant. 
Mi-cho-Coq~\cite{micho} is a framework for the proof assistant Coq to verify
functional correctness of Michelson smart contracts. They formalize
the semantics of a Michelson in Coq using a weakest precondition
calculus and verify several contracts.
It provides full coverage of the language whereas our goal is to give
a blueprint for a soundness proof of symbolic execution.

There are several tools for automated verification including
solc-verify \cite{solc}, VerX \cite{verx}, and Oyente
\cite{oyente}. solc-verify processes smart contracts written in
Solidity and discharges verification conditions using modular program
analysis and SMT solving. It operates at the level of the contract
source code, with properties specified as contract invariants and
function pre- and post-conditions provided as annotations in the code
by the developer. This approach offers a scalable, automated, and
user-friendly formal verification solution for Solidity smart
contracts. The core of solc-verify involves translating Solidity
contracts to Boogie IVL (Intermediate Verification Language), a
language designed for verification.  

Nishida et al. \cite{helmholtz} developed HELMHOLTZ, an automated
verification tool for Michelson. While both research efforts aim to
build a verification tool for smart contracts written in Michelson,
HELMHOLTZ is based on  refinement types, whereas we consider symbolic
execution.
HELMHOLTZ has better coverage of Michelson instructions than we
currently have, but it can only verify a single contract whereas our
model and soundness proof covers full inter-contract verification.
The HELMHOLTZ developers plan to extend Helmholtz with inter-contract behavior.

% iContract targets the Solidity language. It also utilizes pre and
% post conditions, similar to our tool, to specify user properties. It
% locally installs the Solidity compiler to compile a user-provided
% Solidity file into a JSON file containing the typed abstract syntax
% tree (AST). Then, iContract analyzes the AST to encode contracts
% into predicates using the Z3 library. They leverage the NatSpec
% format to define their own specifications. 


% A lot of different tools have been developed to aid programmers in verifying smart contracts.
% Most of them focus on the \emph{Ethereum} platform and many employ symbolic execution tools,
% like Manticore~\cite{manticore}, Oyente~\cite{oyente}, or Securify~\cite{securify}.

Bau et al.~\cite{abstract-interpretation} implement a static analyzer for Michelson
within the modular static analyser MOPSA that is based on abstract interpretation.
It is able to infer invariants on a contract's storage over several calls
and it can prove the absence of errors at run time.

Da Horta et al.~\cite{WHYtool} aim at automating as much of the verification process as possible
by automatically translating a Michelson contract into an equivalent program
for the deductive program verification platform WHY3.
However, they found that sometimes user intervention was required,
and their tool can only verify single contracts individually.

% The Helmholtz verifier~\cite{helmholtz} uses refinement types to prove
% user-defined specifications of Michelson programs with the SMT solver Z3.


%%% from Abstract Interpretation of Michelson Smart-Contracts %%%%%%%%%
% The Micse project [6] allows for automated
% static analysis, using the Z3 SMT solver. The Tezla [30]
% project allows translating the Michelson instructions into a
% suitable intermediate representation for dataflow analysis.

\subsection{Symbolic Execution for Bytecode Interpretation}
\label{sec:symb-exec-bytec}

As there are some parallels between Michelson and bytecode languages,
we discuss symbolic execution methods for some seleted bytecode languages.

Albert et al.~\cite{albert_et_al} transform Java bytecode into a logic
program to utilize analysis techniques from logic programming, specifically
symbolic semantics, for the formal verification of the bytecode. They
verify properties such as termination and run-time error freeness and infer
resource bounds. The dynamic aspects of
bytecode, such as control flow and data flow, are effectively handled
through the analysis performed on the logic program. Balasubramanian,
Daniel et al.~\cite{daniel_et_al} include dynamic symbolic execution tailored for
Java-based web server environments. Their tool
analyzes the bytecode interactions within the Java Virtual Machine and
focuses on bytecode instructions, method calls, object manipulations
and memory interactions to detect vulnerabilities and bugs. 

Several approaches address formal semantics and analysis for
WebAssembly (Wasm)
\cite{marques_et_al,lehmann_el_at,watt_et_al}. Watt, Conrad et
al.~\cite{watt_et_al} present Wasm Logic, a formal program logic
for WebAssembly. The authors mechanize
Wasm logic and its proof of correctness in Isabelle/HOL. To this end,
they propose an alternative semantics.
Just like our work (we propose a logic on an alternative semantics,
mechanize it, and prove its soundness in Agda), their aim is to provide a logical basis for static
analysis tools.

Marques et al.~\cite{marques_et_al} present a concolic execution engine that systematically
explores different program paths by combining concrete and symbolic
execution to enable automated testing and fault detection. It models
execution behavior and uses constraint 
solvers to generate inputs and explore paths, taking into account the
complexity of Wasm's stack-based execution and binary format. 
Unlike our work, their work is geared towards implementing a realistic tool.

\subsection{Related Uses of Dynamic Logic}
\label{sec:related-uses-dynamic}

The idea of using dynamic logic for symbolic execution can be traced
back to Heisel et al.~\cite{DBLP:conf/ki/HeiselRS87}. They formalize
Burstall's verification method~\cite{DBLP:conf/ifip/Burstall74} using
symbolic execution and induction in the framework of dynamic logic.

Maingaud et al.~\cite{DBLP:conf/foveoos/MaingaudBBHM10} define a
program logic for imperative ML programs based on dynamic logic and prove its
soundness. Their goal is to use this logic as a basis for symbolic
execution. 

Similar to our approach, the research of Ahrendt et
al.~\cite{ahrendt_et_al} emphasizes data integrity in Solidity smart
contracts. This framework verifies smart contracts and ensures strong
data integrity and functional correctness under various conditions. It
introduces a specification language for defining contract properties
and behaviors that are critical for security and reliability. Similar
approaches to ours aim to verify the correctness and security of smart
contracts, but differ in methodology and target languages. Their
approach uses dynamic logic for invariant-based specifications with
prototype-based tools, while our approach uses dynamic logic for
symbolic execution and focuses on formal proofs. 

Abstract execution~\cite{DBLP:conf/fm/SteinhofelH19} is a static
verification framework based on symbolic execution. It is geared at
schematic programs, i.e., programs with placeholders for program
fragments, so that it can be used to prove certain program
transformations correct. Its logical basis is dynamic logic 
extending earlier work for Java \cite{KeY3}.

\section{Conclusion}
\label{sec:conclusion}

We presented a dynamic logic for Michelson as well as its extension to blockchain operations
on a small but representative subset of Michelson.
The goal was to create a core model that covers instances of all kinds
of operations and that can be easily extended with further Michelson instructions.
We achieved full coverage of scalar functional instructions, the majority of Michelson
instructions.
To include any further scalar instruction,
one only has to specify its typing rule and its implementation in Agda.
The symbolic execution rule and the soundness proof for that rule is
already provided by our model.
Further instructions that retrieve information from the execution environment can be added
easily as well by extending the \ADT{Environment} record and its subcomponents
to include such information.

We cover three exemplary instructions for control flow,
because most other conditional and looping instructions
are either very similar or very simple and thus easy to include in the presented model.
One aspect of Michelson that is not covered is first-class functions.
Including them might require some reworking of the current model to store such values on the stack.

Efficient symbolic execution is \textbf{not} a goal of this work:
Agda can normalize a concrete or symbolic execution state to enable
inspection of the state after one or more execution steps, but in our
experiments normalization was sometimes infeasible after 
less than ten symbolic execution steps.
Nevertheless, we plan to use our soundness proof as the basis for an efficient
symbolic interpreter for Michelson in ongoing work.


%%% Local Variables:
%%% mode: latex
%%% TeX-master: "itp2024"
%%% End:
