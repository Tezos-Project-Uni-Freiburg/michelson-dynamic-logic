\section{Related work}
\label{sec:related-work}
A lot of different tools have been developed to aid programmers in verifying smart contracts.
Most of them focus on the \emph{Ethereum} platform and many employ symbolic execution tools,
like Manticore~\cite{manticore}, Oyente~\cite{oyente} or Securify~\cite{securify}.

Mi-cho-Coq is a framework for the proof assistant Coq that can verify
functional correctness of Michelson smart contracts,
using a Michelson interpreter in Coq and a weakest precondition calculus~\cite{mi-cho-coq}.

Bau et al. implemented a static analyzer for Michelson
within the modular static analyser MOPSA that is based on Abstract Interpretation.
It is able to infer invariants on a contracts storage over several calls
and can prove the absence of runtime errors~\cite{abstract-interpretation}.

Da Horta et al. aim at automating as much of the verification process as possible
by automatically translating a Michelson contract into an equivalent program
for the deductive program verification platform WHY3.
However they found that sometimes user intervention was required,
and their tool can only verify single contracts individually~\cite{why3}.

The Helmholtz verifier uses refinement types to prove
user-defined specifications of Michelson programs with the SMT solver Z3.
It has better coverage of Michelson than we currently have,
but it can currently only verify a single contract
and the developers plan to extend Helmholtz
to be able to verify inter-contract behavior~\cite{helmholtz}.


%%% from Abstract Interpretation of Michelson Smart-Contracts %%%%%%%%%
% The Micse project [6] allows for automated
% static analysis, using the Z3 SMT solver. The Tezla [30]
% project allows translating the Michelson instructions into a
% suitable intermediate representation for dataflow analysis.

\section{Conclusion}

We presented a dynamic logic for Michelson as well as its extension for blockchain operations
on a small but representative subset of Michelson.
The goal was to create a small model that covers most of its complexity
and can be easily extended with other Michelson instructions.
This was especially successful for onedimensional functional instructions,
which cover most of the instructions for basic data manipulation.
To include any other such instruction,
one only has to specify its typing rule and give its implementation in Agda,
and everything else, especially its symbolic execution rule and the soundness proof for that rule,
will already be correctly implemented by our model.
Other instructions that retrieve information about the execution environment can be added
easily as well by extending the \verb/Environment/ record and its subcomponents
to include such information.

We only covered three exemplary instructions for control flow,
but most other instructions of that type
are either very similar or very simple and will be easy to include into the presented model.
One aspect of Michelson that was not covered is lambda types,
which might require some reworking of the current model to store such values on the stack.

Another limitation of this work is the efficiency of symbolic execution:
Agda can normalize a symbolic execution state to present the current values after the execution 
steps, but it will internally always save the state as an application of every preceeding
symbolic execution step.
Since these involve a lot of weakenings of all subcomponents,
the computational load becomes insurmountable for normal computers even after
less than ten symbolic execution steps.
So for efficient symbolic execution it will be necessary to compile the execution model
to some other language.


%%% Local Variables:
%%% mode: latex
%%% TeX-master: "itp2024"
%%% End:
