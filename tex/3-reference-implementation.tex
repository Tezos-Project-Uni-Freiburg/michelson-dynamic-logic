\section{Michelson Reference Implementation}
\label{sec:refImpl}

Program execution is defined in a stepwise manner by giving a function that maps
the current execution state of a program to a new state resulting from executing
the first instruction:
\mint{agda}|prog-step : Prog-state → Prog-state|
Besides saving the program, stack and execution environment, this \verb/Prog-state/ must contain
some additional information to enable the execution of certain control flow instructions.
For that reason their semantics are explained first
before we present the full definition of \verb/Prog-state/.

\subsection{Execution of control flow instructions}\label{sec:control-flow}

\verb/IF-NONE thn els/ checks the top stack element of type \verb/option/
whether it contains any value.
If not, i.e. it has the value \verb/NONE/, the \verb/thn/ branch is executed on the
rest of the stack:

\[	\inferrule*	[Right=IF-NONE-true]
 	{\text{thn} : \text{\emph{StackIn}} \mapsto \text{\emph{StackOut}}}
 	{\text{IF-NONE thn els} : N\!O\!N\!E :: \text{\emph{StackIn}} 
		\mapsto \text{\emph{StackOut}}}
\]

If however the top of the stack is \verb/SOME x/,
the \verb/els/ branch is executed on the stack where \verb/SOME x/
is replaced with that value \verb/x/:

\[	\inferrule*	[Right=IF-NONE-false]
	{\text{thn} : x :: \text{\emph{StackIn}} \mapsto \text{\emph{StackOut}}}
	{\text{IF-NONE thn els} : S\!\!\;O\!M\!E\; x :: \text{\emph{StackIn}} 
		\mapsto \text{\emph{StackOut}}}
\]

To execute this in a stepwise manner, the program saved in \verb/Prog-state/ is
expanded according to these rules:
When the program is \verb/IF-NONE thn els ; rem-prg/, it is mapped to a state
whose program is either \verb/thn ;; rem-prg/ or \verb/els ;; rem-prg/,
depending on the value on top of the stack (\verb/;;/ concatenates two programs).

\verb/DIP n prg/ executes \verb/prg/ on the stack that results from removing the first n elements
of the current stack.
It produces a stack where the top n elements are left unchanged
and the rest is manipulated according to \verb/prg/:
\[	\inferrule*	[Right=DIP]
	{\text{prg} : \text{\emph{StackIn}} \mapsto \text{\emph{StackOut}}
	\\\\	length(\text{\emph{top}}) == \text{n}}
	{\text{DIP n prg} :	\text{\emph{top}} +\!\!\!+\; \text{\emph{StackIn}}
		\mapsto		\text{\emph{top}} +\!\!\!+\; \text{\emph{StackOut}} }
\]

Just like before, the program in the \verb/Prog-state/ is expanded with \verb/prg/,
but this subprogram must be executed on the current stack minus its top n elements.
Also, these elements have to be retrieved after execution of the subprogram and placed back on
the stack.
Therefore a mechanism for holding on to the top of the stack while executing the subprogram
and retrieving it afterwards is necessary.

Execution of \verb/ITER/ requires the same features in a slightly different way.
It consumes the list on top of the current stack.
When the list is empty, nothing happens:

% 
\[	\inferrule*	[Right=ITER-nil]
	{ }
	{\text{ITER prg} : N\!I\!L :: \text{\emph{StackIn}}  \mapsto \text{\emph{StackIn}} }
\]

Otherwise the subprogram is applied to each list element:

\[	\inferrule*	[Right=ITER-cons]
	{	\text{prg} :      x  :: \text{\emph{StackIn}}  \mapsto \text{\emph{StackOut}}
	\\\\	\text{ITER prg} : xs :: \text{\emph{StackOut}} \mapsto \text{\emph{StackEnd}} }
	{\text{ITER prg} : (x :: xs) :: \text{\emph{StackIn}}  \mapsto \text{\emph{StackEnd}} }
\]

Notice that the typing rules for \verb/ITER/ require that the type of the underlying stack must
be preserved, its values however may be modified.

For stepwise execution, when the list is not empty, it must be replaced with its first element
and the program is expanded with the subprogram that consumes that element.
After execution of the subprogram,
the rest of the list has to be retrieved
and when it containes more elements it must be further processed this way.
Otherwise, if it is empty, it can be discarded and execution of the next instruction
after \verb/ITER/ can be resumed.
% we need a mechanism to retrieve the rest of the list
% and either continue iterating over it or proceed with executing the rest of the program.

Since these subprograms may be arbitrarily complex, they can contain more of these kind of
instructions that must save some part of the stack for later,
so this data must be stored on a second stack which we call the shadowstack.
To retrieve and handle data from that stack, we also need new instructions
which will be added to the end of the respective subprograms for that purpose.
\verb/DIP'/ will simply take the elements from the shadowstack that were placed there by \verb/DIP/
and put them back onto the main stack, after it has been modified by the subprogram.
\verb/ITER'/ will check the remaining list on top of the shadowstack.
If it still contains some elements, the program will be expanded again and the next list element
will be placed on to the main stack. Otherwise the empty list is dropped and execution contiues
with the next instruction after \verb/ITER/.

These new instructions are called shadow instructions and will be intrinsically typed like
normal instructions, only that they are indexed by four stacks in total, the in- and output stacks
of the main stack which will be called real stack, and the in- and output stacks of the shadow
stacks.
Shadow programs extend the definition of programs to include these new instructions
and will be defined with an analogous intrinsic typing scheme for all four stacks
(see \listref{shadow}).

%% listing ruler max width ------------------------------------------------|?X
\begin{listing}[!ht]
\begin{minted}{agda}
data ShadowInst : Stack → Stack → Stack → Stack → Set where
  ITER'     : ∀ {ty rS sS}
            → Program      [ ty // rS ]                              rS
            → ShadowInst           rS   [ list ty // sS ]            rS  sS
  DIP'      : ∀ top {rS sS}
            → ShadowInst           rS        (top ++ sS )    (top ++ rS) sS

data ShadowProg : Stack → Stack → Stack → Stack → Set where
  end  : ∀ {rS sS} → ShadowProg rS sS rS sS
  _;_  : ∀ {ri rn si ro so}
       → Instruction ri     rn
       → ShadowProg  rn si  ro so
       → ShadowProg  ri si  ro so
  _∙_  : ∀ {ri si rn sn ro so}
       → ShadowInst  ri si  rn sn
       → ShadowProg  rn sn  ro so
       → ShadowProg  ri si  ro so
\end{minted}
\caption{Shadow instructions and programs}
\label{shadow}
\end{listing}

\subsection{Program execution}

In the last chapter we are only concerned with the type of a Michelson stack.
For program execution, both types and values of stack elements are relevant.
We represented a stack during program execution with the Agda datatype \verb/Int/
given in \listref{Int}.
It behaves like a list of Michelson values, indexed by their types.

\begin{listing}[!ht]
\begin{minted}{agda}
data Int : Stack → Set where
  [I]  : Int []
  _∷_ : ∀ {ty S} → ⟦ ty ⟧ → Int S → Int (ty ∷ S)
\end{minted}
\caption{Stacks of typed values}
\label{Int}
\end{listing}

A program state contains the shadow program that is currently executed,
the real and shadow stack, and an \verb/Environment/ datatype which provides the necessary
context information to execute \verb/env-func/ instructions (see \listref{Prog-state}).
It is parameterized by the resulting output stacks, which does not change during program execution.
When executing more than one contract as we demonstrate in \secref{sec:contract-execution},
this parameterization ensures that the results from completed contract executions are well typed.

%% listing ruler max width ------------------------------------------------|?X
\begin{listing}[!ht]
\begin{minted}{agda}
record Prog-state {ro so : Stack} : Set where
  constructor state
  field
    {ri si} : Stack
    en  : Environment
    prg : ShadowProg ri si  ro so
    rSI : Int ri
    sSI : Int si
\end{minted}
\caption{Program states}
\label{Prog-state}
\end{listing}

The function \verb/prog-step/ will map any program execution state to the state
resulting from executing the first instruction of the program.
We explain two exemplary cases given in \listref{prog-step}.

%% listing ruler max width ------------------------------------------------|?X
\begin{listing}[!ht]
\begin{minted}{agda}
prog-step : ∀ {ro so} → Prog-state {ro} {so} → Prog-state {ro} {so}

prog-step  ρ@(state _  (fct ft ; prg)   rSI _)
  = record ρ {  prg = prg
             ;  rSI = (appft ft (Itop rSI)  +I+  Ibot rSI) }

prog-step  ρ@(state _ (ITER     ip ; prg)    (x ∷ rSI)               sSI)
  = record ρ {  prg =  ITER'    ip ∙ prg  ; rSI = rSI ;    sSI = x ∷ sSI }
prog-step  ρ@(state _ (ITER'    ip ∙ prg)           _          ([] ∷ sSI))
  = record ρ {  prg =                prg  ;                sSI =     sSI }
prog-step  ρ@(state _ (ITER'    ip ∙ prg)         rSI ([ x // xs ] ∷ sSI))
  = record ρ {  prg = ip  ;∙  ITER' ip ∙ prg
             ;  rSI = x ∷ rSI  ;  sSI = xs ∷ sSI }
\end{minted}
\caption{Single program step execution}
\label{prog-step}
\end{listing}

There is a generic way for executing \verb/func-type/ instruction
by utilizing the following functions:
\mint{agda}|impl : ∀ {args result} → func-type args result → ⟦ args ⇒ result ⟧|
maps every function type to its implementation in Agda, and
\mint{agda}|apply : ∀ {args result} → ⟦ args ⇒ result ⟧ → Int args → Int [× result ]|
is a generic way of applying functions to their argument values where
arguments and results are given as an interpretation.
\mint{agda}|appft : ∀ {args result} → func-type args result → Int args → Int [× result ]|
% \verb| 
% \verb/appft/ 
combines these functions, and due to the intrinsic typing scheme
the real stack interpretation \verb/rSI/ is indexed by \verb/args ++ S/
(see \listref{Instruction}) and Agda can implicitly apply \verb/Itop/ and \verb/Ibot/
which split \verb/rSI/ into its top part containing only the
arguments for \verb/ft/ and the remaining stack which is left unchanged
(\verb/+I+/ is the concatenation operator for interpretations).

The semantics for executing \verb/ITER/ are already explained in \secref{sec:control-flow}.
\listref{example:ITER-ADD} showcases an example program that
calculates the sum of the list elements and the number below the list.

\begin{listing}[!ht]
\begin{minted}{agda}
example-ITER : Program [ list (base nat) / base nat ] [ base nat ]
example-ITER = ITER (ADD ; end) ; end
\end{minted}
\caption{Simple ITER program}
\label{example:ITER-ADD}
\end{listing}

\tabref{prog-step:ITER-ADD} gives the stacks and shadow program of each \verb/Prog-state/
resulting from applying \verb/prog-step/ until program termination
for the given input stack interpretations (neglecting \verb/end/ for readability).

%% listing ruler max width ------------------------------------------------|?X
\begin{table}[!ht]
\begin{verbatim}
                     rSI                   sSI                       prg
  ----------------------------------------------------------------------
  [ 18 / 24 ] ∷  0 ∷ [I]                   [I]               ITER  (ADD)
                 0 ∷ [I]     [ 18 / 24 ] ∷ [I]               ITER' (ADD)
           18 ∷  0 ∷ [I]          [ 24 ] ∷ [I]         ADD ; ITER' (ADD)
                18 ∷ [I]          [ 24 ] ∷ [I]               ITER' (ADD)
           24 ∷ 18 ∷ [I]              [] ∷ [I]         ADD ; ITER' (ADD)
                42 ∷ [I]              [] ∷ [I]               ITER' (ADD)
                42 ∷ [I]                   [I]                       end
\end{verbatim}
% \begin{tabular}{@{}rrr@{}}
% \toprule
% \midrule
% \bottomrule
% \end{tabular}
\caption{program states during execution of \listref{example:ITER-ADD}}
\label{prog-step:ITER-ADD}
\end{table}

\subsection{Contract execution and execution chains}\label{sec:contract-execution}

\verb/prog-step/ can execute any Michelson program, not only those that comply
to the typing restrictions of a contract.
But it provides neither a mechanism to update the blockchain after successful contract execution
nor one to execute other blockchain operations which might be emitted by a contract.

\begin{comment}
When a contract execution terminates, the final stack interpretation will contain a pair
of a list of blockchain operations to be emitted by the contract as well as the updated
storage value of the contract.
Also contract execution is triggered by transfering some amount of Tezos tokens to it,
so it's balance and storage has to be updated and the emitted operations
must be staged for execution.
\end{comment}

To implement these aspects of contract execution, the \verb=Prog-state= is wrapped
into two other records shown in \listref{Exec-state}:
\verb=Prg-running= additionally saves the contracts involved in the current execution.
% (and makes sure that the executed program will terminate with the expected stacks,
% see \listref{Prg-running}).
The \verb=Exec-state= holds the \verb=Blockchain= where contract execution results are saved,
and a list of pending blockchain operations to be executed.
\verb/MPstate/ encodes if a contract is currently executed or not.
If so, its value will contain \verb/Prg-running/, otherwise it is empty and execution
will proceed with the pending blockchain operations.

%% listing ruler max width ------------------------------------------------|?X
\begin{listing}[!ht]
\begin{minted}{agda}
record Prg-running : Set where
  constructor pr
  field
    {p s x y} : Type
    current : Contract {p} {s}
    sender  : Contract {x} {y}
    ρ       : Prog-state {[ pair (list ops) s ]} {[]}

record Exec-state : Set where
  constructor exc
  field
    accounts : Blockchain
    MPstate  : Maybe Prg-running
    pending  : List (List Operation × ⟦ base addr ⟧)
\end{minted}
\caption{Contract execution state}
\label{Exec-state}
\end{listing}

\verb|exec-step : Exec-state → Exec-state| maps any execution state to its successor state
just like \verb/prog-step/ did for program states.
It only implements the features mentioned above that cannot be modelled by the program state alone,
but its definition is still very large because it involves checking many cases using
\verb/with/ clauses in Agda. 
We briefly explain its implementation, giving each case in the same
order as in the implementation
% \modulel{03-concrete-execution}{195}.

\begin{itemize}
	\item	When execution of the current contract has terminated
		(i.e. \verb/MPstate/ is given and \verb=Prog-state.prg= matches \verb=end=),
		due to the intrinsic typing scheme the shadow stack interpretation must be empty
		and the real stack interpretation will contain the emitted blockchain operations
		and updated storage value.
		If the executed contract has called itself (\verb/with cadr ≟ₙ sadr/),
		only its storage is updated on the blockchain.
		Otherwise the balances of the involved contracts must be updated as well.
		In either case the emitted operations are appended to the pending operations
		(together with the address from which they were emitted for later references).
	\item	In all other cases of a running program, the \verb=Prog-state= is executed
		using \verb=prog-step=
	\item	The remaining cases are those where no contract is currently executed
		and \verb=pending= is checked for other operations to be executed.
		Our model only implements the \verb/TRANSFER-TOKENS/ operation
		that initiates a new contract execution.
		Many parameters have to be checked in this case:
		\begin{itemize}
			\item	if the operation was not emitted from a valid account
				(i.e. \verb/accounts sadr/ matches \verb/nothing/)
				all operations from this account will be ignored.
			\item	if it is valid, it must be checked wheather the transfer
				operation is for the same contract or a different one
				(\verb/sadr ≟ₙ cadr/)
			\item	in either case, the type of the parameter must be checked
				against the input type of the called contract
				(\verb/ty ≟ p/)
			\item	when the transfer targets a different account than the one
				that emitted the operation, we also check if the
				emitters balance contains enough tokens to support the transfer
				(\verb/Contract.balance sender <? tok/)
				and if the transfer target is a valid account
				(\verb/accounts cadr/)
				before checking the parameter type for this case.
		\end{itemize}
		Note that two of these cases can never come up during an actual execution of
		a Michelson smart contract execution chain:
		The \verb=TRANSFERE-TOKENS= instruction only works for values of the
		correct parameter type, and operations can only be emitted by valid accounts.
		But unlike the typing restrictions of the updated storage that can be encoded
		with the intrinsic typing scheme and the parameterization of the program state,
		it is not possible encode these properties
		in the model and these impossible cases must be checked anyway.

		It should also be mentioned that this implementation of multi contract execution
		does not reflect the exact behaviour of the Tezos blockchain:
		The list of pending operations we use is a \emph{FIFO} structure,
		but in reality the order of execution of emitted operations is more complicated
		(see \cite{devres} for more Details).
		It is possible to reflect this behaviour when implementing
		the different account types of the Tezos blockchain.
		Our model suffices as a proof of concept.
\end{itemize}


\begin{comment}

\subsubsection{Shadow Stack and extended Instructions/Programs}\label{subsec:shadow}

\verb=DIP= and \verb=ITER= are special in that they need a second stack to be executed:
During the execution of the sub-program of \verb=DIP= the top \verb=n= elements must be stored
to be later retrieved when sub-program execution has terminated.
Likewise for \verb=ITER=, which consumes the list at the top of the stack
by executing its sub-program for every list element.
Here the currently remaining list has to be stored during sub-program execution.
Since the sub-programs can also contain such instructions that need to store data away for later,
a second stack is needed, as well as new instructions to operate on it:
To execute \texttt{DIP n prg ; \emph{remainingProg}}, the top \verb=n= elements of the
main stack are transfered to this second stack (we call it the \emph{shadow stack}) and
the instruction is replaced by \verb=prg= followed by the \emph{shadow instruction} \verb=DIP'=
which retrieves them from the shadow stack and puts them back onto the mainstack
before continuing execution with \texttt{\emph{remainingProg}}.
For the instruction \verb=ITER prg= the list on top of the main stack will be placed
onto the shadow stack and the instruction will be replaced by its shadow version \verb=ITER' prg=
which does all the actual work: It checks if the list at the top of the shadow stack is empty.
If so, it will be dropped and execution continues.
If not, the first element in the list will be moved to the top of the main stack and
\verb=prg= will be executed. After that \verb=ITER'= is executed again to check the list
until all elements have been iterated over.

These new shadow instructions must therefore be parameterized by 4 stacks:
main input stack, shadow input stack, main output stack and shadow output stack.
Analogous to the abbreviations in our code we will call the main stack \emph{real stack}.

Shadow programs are programs containing ``real'' and shadow instructions (see \listref{shadow}).


Operators to concatenate programs and shadow programs are given in the canonical way:
\begin{listing}[!ht]
\begin{minted}{agda}
_;;_ : ∀ {Si So Se} → Program Si So → Program So Se → Program Si Se
_;∙_ : ∀ {ri rn si ro so}
     → Program ri rn → ShadowProg rn si ro so → ShadowProg ri si ro so
\end{minted}
\caption{program concatenations}
\label{concat}
\end{listing}

\end{comment}

%%% Local Variables:
%%% mode: latex
%%% TeX-master: "itp2024"
%%% End:
