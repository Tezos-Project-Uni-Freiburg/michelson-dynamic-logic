\section{Michelson Reference Implementation}
\label{sec:refImpl}

Program execution is defined by giving a function that maps
the current execution state of a program to a new state resulting from executing
the first instruction:
\ConcreteprogStep

Besides saving the program, stack, and execution environment, this \ADT{ProgState} must contain
some additional information to enable the execution of certain control flow instructions.
For that reason we recall their semantics from the Michelson specification
before we present the full definition of \ADT{ProgState}.

\subsection{Execution of control flow
  instructions}\label{sec:control-flow}

The specification defines the semantics of Michelson in terms of a big-step
judgment $p : \text{\emph{StackIn}} \mapsto
\text{\emph{StackOut}}$. This choice enables some liberties that are
difficult to model in a step-by-step (i.e., small-step) execution
model. We discuss these issues with some representative instructions.

The instruction \ACon{IF-NONE}~\textit{thn}~\textit{els} checks the top stack element of type \verb/option/
whether it contains any value.
If it is \verb/NONE/, the \textit{thn} branch is executed on the
rest of the stack:

\[	\inferrule*	[Right=IF-NONE-true]
 	{\textit{thn} : \text{\emph{StackIn}} \mapsto \text{\emph{StackOut}}}
 	{\text{\ACon{IF-NONE} }\textit{thn}~\textit{els} : N\!O\!N\!E :: \text{\emph{StackIn}} 
		\mapsto \text{\emph{StackOut}}}
\]

If however the top of the stack is \verb/SOME x/,
the \verb/els/ branch is executed on the stack where \verb/SOME x/
is replaced with that value \verb/x/:

\[	\inferrule*	[Right=IF-NONE-false]
	{\text{thn} : x :: \text{\emph{StackIn}} \mapsto \text{\emph{StackOut}}}
	{\text{\ACon{IF-NONE} thn els} : S\!\!\;O\!M\!E\; x :: \text{\emph{StackIn}} 
		\mapsto \text{\emph{StackOut}}}
\]

To specify the corresponding small-step rule requires a type-respecting concatenation
operator \verb/;;/ on programs. 
The program \ACon{IF-NONE}\verb/ thn els ; rem-prg/ either transitions to
\verb/thn ;; rem-prg/ or to \verb/els ;; rem-prg/,
depending on the value on top of the stack.

The instruction $\ACon{DIP}~n~p$ executes program $p$ on the stack that results from removing the first $n$ elements
of the current stack and reattaches them afterwards.
\[	\inferrule*	[Right=DIP]
	{\text{prg} : \text{\emph{StackIn}} \mapsto \text{\emph{StackOut}}
	\\\\	length(\text{\emph{front}}) == \text{n}}
	{\text{\ACon{DIP} n prg} :	\text{\emph{front}} +\!\!\!+\; \text{\emph{StackIn}}
		\mapsto		\text{\emph{front}} +\!\!\!+\; \text{\emph{StackOut}} }
\]

In the small-step version, dropping the first $n$ elements of the
stack is easy, but reattaching them requires extra machinery.
Thus, a mechanism for holding on to the top of the stack while executing the subprogram
and retrieving it afterwards is necessary.

Execution of \ACon{ITER} requires the same features in a slightly different way.
It consumes the list on top of the current stack.
If the list is empty, nothing happens:
% 
\[	\inferrule*	[Right=ITER-nil]
	{ }
	{\text{\ACon{ITER} prg} : N\!I\!L :: \text{\emph{StackIn}}  \mapsto \text{\emph{StackIn}} }
\]

Otherwise the subprogram is applied to each list element:
\[	\inferrule*	[Right=ITER-cons]
	{	\text{prg} :      x  :: \text{\emph{StackIn}}  \mapsto \text{\emph{StackOut}}
	\\\\	\text{\ACon{ITER} prg} : xs :: \text{\emph{StackOut}} \mapsto \text{\emph{StackEnd}} }
	{\text{\ACon{ITER} prg} : (x :: xs) :: \text{\emph{StackIn}}  \mapsto \text{\emph{StackEnd}} }
\]

The typing for \ACon{ITER} requires that the type of the underlying stack must
be preserved, but the subprogram prg is
entitled to access and modify the stack beyond the first element $x$.
Let's now consider stepwise execution. If the list on top has the form $x
:: xs$,  we need to stash the tail list $xs$ somewhere while the subprogram
processes the stack with $x$ on top.
After execution of the subprogram,
we have to recover $xs$ and try again with \ACon{ITER}.
% we need a mechanism to retrieve the rest of the list
% and either continue iterating over it or proceed with executing the rest of the program.

As subprograms can be arbitrarily complex, in particular, they
may contain \ACon{DIP} and \ACon{ITER}, we need a nestable solution.
To this end, we add a single new instruction \ACon{mpush} that pushes
an arbitrary list of values on the stack. This instruction is
different from the normal \ACon{PUSH} instruction, which is limited to
\ACon{Pushable} values that have a textual representation.

Our small-step version of \ACon{DIP}~$n$~$dp$ takes the top $n$ elements from the stack
and starts executing the program $dp$ followed by 
the new instruction \ACon{mpush}~\textit{front} where
\textit{front} is the list of the $n$ values that were removed from
the stack.
\ConcreteprogStepDIP

The small-step version of \ACon{ITER}~$ip$ just pops the stack if the
list is empty.
Otherwise, if the top contains $x :: xs$, it pops this value, puts $x$
on top of the stack and executes $ip$ followed by
\ACon{mpush}~\texttt{[ xs ]} and then \ACon{ITER}~$ip$ and the rest of
the program.
\ConcreteprogStepITER

\begin{figure}[tp]
  \ConcreteExampleITER
  \caption{Simple program using \ACon{ITER}}
  \label{fig:ITER-ADD}
\end{figure}
For illustration, 
\tabref{prog-step:ITER-ADD} gives the stacks and shadow program of
each intermediate state resulting from applying \AFun{prog-step} to
the program in Figure~\ref{fig:ITER-ADD} until program termination
for the given input stack interpretation (omitting \ACon{end} for
readability).
This program adds a list of numbers on top of the stack to the number below.

%% listing ruler max width ------------------------------------------------|?X
\begin{table}[tp]
\begin{verbatim}
                  rSI                                    prg
----------------------------------------------------------------------
[ 18 , 24 ] ∷  0 ∷ []                             ITER (ADD)
         18 ∷  0 ∷ []         ADD ; MPUSH [ 24 ]; ITER (ADD)
              18 ∷ []               MPUSH [ 24 ]; ITER (ADD)
     [ 24 ] ∷ 18 ∷ []                             ITER (ADD)
         24 ∷ 18 ∷ []             ADD ; MPUSH []; ITER (ADD)
              42 ∷ []                   MPUSH []; ITER (ADD)
         [] ∷ 42 ∷ []                             ITER (ADD)
              42 ∷ []                                    end
\end{verbatim}
% \begin{tabular}{@{}rrr@{}}
% \toprule
% \midrule
% \bottomrule
% \end{tabular}
\caption{Program states during execution of Figure~\ref{fig:ITER-ADD}}
\label{prog-step:ITER-ADD}
\end{table}

% \ACon{ITER'} will check the remaining list on top of the shadowstack.
% If it still contains some elements, the program will be expanded again and the next list element
% will be placed on to the main stack. Otherwise the empty list is dropped and execution contiues
% with the next instruction after \verb/ITER/.

The new instruction is indexed by two stack types like
any other instruction.
We introduce \emph{shadow programs} to extend the definition of
programs to include the new instruction. Shadow programs only appear
at the top-level, never as subprograms nested in instructions. We elide the definition of
\ADT{ShadowProg} which is analogous to \ADT{Program}.
\ConcreteShadowInst

\subsection{Program execution}
\label{sec:program-execution}

So far we only concerned ourselves with the type of a Michelson stack.
For program execution, both the types and values of stack elements are relevant.
To this end, we have to lift the interpretation of a single type,
i.e., a function from {\AType} to {\ASet}, to the interpretation of a
list of types. The library predicate \ADT{All} does exactly that: it
``maps'' a {\ASet}-typed function over a list, which yields a
heterogeneously typed list.

For example, the value interpretation of a type stack is a value stack where
corresponding elements $t$ and $v$ are related by the type
interpretation, that is, $v : \ASem{t}$. 
\FunctionsInt

The definition of a program state abstracts over a $Mode$ which
contains a suitable type interpretation.
A program state contains the shadow program that is currently executed,
the stack, and an environment which provides the
context information to execute blockchain instructions like
\ACon{AMOUNT} and \ACon{BALANCE}.
It is parameterized by the output stack type, which does not change during execution.
When executing more than one contract as we demonstrate in \secref{sec:contract-execution},
this parameterization ensures that the results from completed contract executions are well typed.
\ConcreteProgState

\begin{figure}[tp]
  \ConcreteprogStepfct
  \caption{Single program step execution (excerpt)}
  \label{fig:prog-step-example}
\end{figure}
The function \AFun{prog-step} executes the first instruction of a
program on the current state.
We explain two exemplary cases shown in
Figure~\ref{fig:prog-step-example}.  To explain the first stanza of
the code we have to make a
confession. As several instructions have very similar semantics, our
internal representation of instructions is a refinement of the
datatype shown in Figure~\ref{fig:core-michelson-instructions}. For
example, all instructions that just apply a function to the top of the
stack are grouped under a constructor \ACon{fct} and \ADT{func-type}
is the type defining these instructions.
\FunctionsInstructionfct

The function \AFun{app-fct} applies such a function to a concrete
stack. Roughly speaking, if the underlying function has type $a_1 \to \dots \to a_n \to
(r_1 \times \dots \times r_m)$ it gets transformed into a function
between heterogeneously typed lists
$[a_1, \dots, a_n] \to [r_1, \dots, r_m]$. We elide the definition and
just remark that the function $[{\times}\_]$ implements the
transformation between $(r_1 \times \dots \times r_m)$ and $[r_1,
\dots, r_m]$. The functions \AFun{H.front} and \AFun{H.rest} (in Fig.~\ref{fig:prog-step-example}) split the
input stack according to the types expected by the function \textit{ft}. The
function \AFun{H.++} is concatenation of heterogeneous lists.

The \ACon{DROP} instruction drops the top of the stack.



\subsection{Contract execution and execution chains}\label{sec:contract-execution}

The \AFun{prog-step} function can execute any Michelson program, not only those that comply
to the typing restrictions of a contract.
But it does not provide a mechanism to update the blockchain after successful contract execution
nor one to execute other blockchain operations which might be emitted by a contract.

\begin{comment}
When a contract execution terminates, the final stack interpretation will contain a pair
of a list of blockchain operations to be emitted by the contract as well as the updated
storage value of the contract.
Also contract execution is triggered by transfering some amount of Tezos tokens to it,
so it's balance and storage has to be updated and the emitted operations
must be staged for execution.
\end{comment}


To implement these aspects of contract execution, the \ADT{ProgState}
is augmented with further information as shown in Figure~\ref{fig:contract-execution-state}.
\begin{figure}[tp]
  \ConcretePrgRunning
  \ConcreteTransaction
  \ConcreteExecState
  \caption{Contract execution state}
  \label{fig:contract-execution-state}
\end{figure}
\ADT{PrgRunning} holds the contracts involved in the current
execution: \ACon{self} is the current contract and \ACon{sender} is
the sender.
% (and makes sure that the executed program will terminate with the expected stacks,
% see \listref{PrgRunning}).
The \ADT{ExecState} holds the \ADT{Blockchain}, where contract execution results are saved,
and a list of pending blockchain transactions to be executed. These
transactions of type \ADT{Transaction} comprise a list of operations and the address of the
sender of these operations.
The field \ADT{MPstate} encodes if a contract is currently executing.
If so, its value contains a running program of type \ADT{PrgRunning}
where we can take a step. If it is empty, execution
proceeds with the pending blockchain operations.

\begin{figure}[tp]
  \ConcreteExecStepProgram
  \caption{Program execution}
  \label{fig:exec-step-1}
\end{figure}
The function {\ConcreteExecStep} maps an execution state to its successor state
just like \AFun{prog-step} did for program states.
It only implements the features mentioned above that cannot be modeled
by the program state alone.
Its definition is too big to include it in full; instead
we briefly explain its implementation, giving each case in the same
order as in the implementation.
% \modulel{03-concrete-execution}{195}.

Figure~\ref{fig:exec-step-1} contains the cases when a contract is executing.
\begin{enumerate}
\item When execution of the current contract has terminated
  (i.e., \ADT{MPstate} is $\ACon{inj₁}~pr$ and \ADT{ProgState.prg} matches \ADT{end}),
  then intrinsic typing ensures that the shadow stack interpretation is empty
  and the real stack interpretation contains the emitted blockchain operations
  and the new storage value.
  The emitted operations are added to the \ACon{pending} field and
  it remains to update terminated contract's storage on the blockchain. 
  % Otherwise the balances of the involved contracts must be updated as well.
  % In either case the emitted operations are appended to the pending operations
  % (together with the address from which they were emitted for later references).
\item In all other cases of a running program, the \ADT{ProgState} of
  the running program is executed using \ADT{prog-step}.
\end{enumerate}
In the remaining cases \ADT{MPstate} is $\ACon{inj₂}~tt$ which
  means that no contract is currently executed. In this case
  \ACon{pending} is checked for other operations to be executed. 
  Our model only implements the \ACon{TRANSFER-TOKENS} operation
  that initiates a new contract execution.
  Many parameters have to be checked in this case:
  \begin{itemize}
  \item if the operation was not emitted from a valid account
    (i.e. \verb/accounts sender-addr/ is \verb/nothing/)
    all operations from this account will be ignored.
    % any call on pending has a valid account address
  \item if the sender account is valid, it must be checked whether the
    transfer operation is for the same contract or a different one
    % why
  \item in either case, the type of the parameter must be checked
    against the input type of the called contract
    % this is guaranteed for calls on pending
  \item when the transfer targets a different account than the one
    that emitted the operation, we also check if the emitter's balance
    contains enough tokens to support the transfer 
    and if the transfer target is a valid account
    before checking the parameter type for this case.
    % first check is needed, but the second is not.
  \end{itemize}
  Two of these cases can never come up during an actual execution of
  a Michelson smart contract execution chain:
  The \ADT{TRANSFER-TOKENS} instruction only works for values of type $\Acontract~t$ the correct parameter type, and operations can only be emitted by valid accounts.
		But unlike the typing restrictions of the updated storage that can be encoded
		with the intrinsic typing scheme and the parameterization of the program state,
		it is not possible encode these properties
		in the model and these impossible cases must be checked anyway.

		It should also be mentioned that this implementation of multi contract execution
		does not reflect the exact behaviour of the Tezos blockchain:
		The list of pending operations we use is a \emph{FIFO} structure,
		but in reality the order of execution of emitted operations is more complicated
		(see \cite{devres} for more Details).
		It is possible to reflect this behaviour when implementing
		the different account types of the Tezos blockchain.
		Our model suffices as a proof of concept.



\begin{comment}

\subsubsection{Shadow Stack and extended Instructions/Programs}\label{subsec:shadow}

\ADT{DIP} and \ADT{ITER} are special in that they need a second stack to be executed:
During the execution of the sub-program of \verb=DIP= the top \verb=n= elements must be stored
to be later retrieved when sub-program execution has terminated.
Likewise for \verb=ITER=, which consumes the list at the top of the stack
by executing its sub-program for every list element.
Here the currently remaining list has to be stored during sub-program execution.
Since the sub-programs can also contain such instructions that need to store data away for later,
a second stack is needed, as well as new instructions to operate on it:
To execute \texttt{DIP n prg ; \emph{remainingProg}}, the top \verb=n= elements of the
main stack are transfered to this second stack (we call it the \emph{shadow stack}) and
the instruction is replaced by \verb=prg= followed by the \emph{shadow instruction} \verb=DIP'=
which retrieves them from the shadow stack and puts them back onto the mainstack
before continuing execution with \texttt{\emph{remainingProg}}.
For the instruction \verb=ITER prg= the list on top of the main stack will be placed
onto the shadow stack and the instruction will be replaced by its shadow version \verb=ITER' prg=
which does all the actual work: It checks if the list at the top of the shadow stack is empty.
If so, it will be dropped and execution continues.
If not, the first element in the list will be moved to the top of the main stack and
\verb=prg= will be executed. After that \verb=ITER'= is executed again to check the list
until all elements have been iterated over.

These new shadow instructions must therefore be parameterized by 4 stacks:
main input stack, shadow input stack, main output stack and shadow output stack.
Analogous to the abbreviations in our code we will call the main stack \emph{real stack}.

Shadow programs are programs containing ``real'' and shadow instructions (see \listref{shadow}).


Operators to concatenate programs and shadow programs are given in the canonical way:
\begin{listing}[!ht]
\begin{minted}{agda}
_;;_ : ∀ {Si So Se} → Program Si So → Program So Se → Program Si Se
_;∙_ : ∀ {ri rn si ro so}
     → Program ri rn → ShadowProg rn si ro so → ShadowProg ri si ro so
\end{minted}
\caption{program concatenations}
\label{concat}
\end{listing}

\end{comment}

%%% Local Variables:
%%% mode: latex
%%% TeX-master: "itp2024"
%%% End:
