\section{Introduction}
\label{sec:introduction}

Blockchain technology and smart contracts provide decentralized and
immutable systems for secure transactions and automated agreements.
Smart contracts have been targets of spectacular and costly attacks as
contracts are immutable and their source code is publicly available on
the blockchain. 
Hence, it is vital as well as challenging to ensure the correctness of smart contracts
before their deployment. Formal methods and various verification
techniques have been proposed to address this challenge.

The Tezos blockchain \cite{tezos-whitepaper} and its smart contract
language Michelson have been designed from ground up with verification
in mind. Several frameworks have been developed based on, e.g.,
interactive theorem proving \cite{micho}, refinement typing
\cite{helmholtz}, and automated theorem proving \cite{WHYtool}.  We
are interested in automated verification of Michelson programs, which
rules out interactive approaches.  Symbolic
execution~\cite{DBLP:journals/cacm/King76,DBLP:conf/ifip/Burstall74}
is one of the standard approaches to automatically obtain verification
conditions like weakest preconditions for failures as well as normal
termination from a program. Next, an SMT-solver discharges these
verification conditions. There is a wide range of approaches that
apply symbolic execution combined with SMT-solving to smart contracts,
mostly for the Ethereum blockchain (see~\autoref{sec:related-work}).

While there are many approaches to symbolic execution
\cite{DBLP:conf/osdi/CadarDE08,DBLP:conf/icse/CsallnerTS08,Pasareanu2020},
we choose one based on dynamic logic.
Dynamic logic (DL) \cite{DL} is a modal logic for reasoning about
programs. Its signature features are modalities for program
execution. These modalities enable the expression of assertions about
program behavior as logical formulas. For instance, the formula $[p]
\Psi$ states partial correctness: if program $p$ terminates, then $\Psi$ is true. That
is, a Hoare triple $\{\Phi\}~p~\{ \Psi\}$ can be
encoded by $\Phi \to [p]\Psi$. DL also provides a modality
$\langle p\rangle$ for total correctness, but we do not consider it in this work.

Dynamic logic comes with proof rules for the modality derived from the
structure of $p$. For example, if $p;q$ stands for sequential 
execution of $p$ and $q$, then the proof rule $[p;q]\Psi
\leftrightarrow [p][q]\Psi$ states that execution of $p$ enables
execution of $q$ such that $\Psi$ holds in the end. Similarly, the
rule $[\varepsilon]\Psi \leftrightarrow \Psi$ states that the empty
program $\varepsilon$ does not modify the validity of $\Psi$.

\begin{comment}
  Such formulas are verified by successively reducing the program it
  contains until one is left with a purely first order formula that
  does not contain programs anymore, which can then be verified using
  the calculus of the first order logic that the DL is based on:
  \begin{align*}
    \langle i ; i' ; &\cdots ; i^n \rangle \phi
    \\ \leftrightarrow	\qquad		\langle     i' ; &\cdots ; i^n \rangle \phi'
    \\							 &\quad\vdots
    \\ \leftrightarrow	\qquad	\langle \rangle \phi^n	 &\leftrightarrow \phi^n
    \\							 &\quad\vdots
  \end{align*}

  % \section*{Scope and structure}
  This thesis focuses on these symbolic execution rules of the
  calculus and their soundness proof.  Firstly, we choose a
  representative subset of Michelson and give a reference
  implementation of it.  Then we define the symbolic execution rules
  for that subset and prove their soundness with respect to the
  reference implementation.
\end{comment}



In the past, dynamic
logic has been used successfully for as a basis for symbolic execution
in the context of the verification of Java programs \cite{KeY3}, as it
is particularly well suited to keep track of a changing environment (i.e., mutable
objects on Java's heap). 
We design a DL to model Michelson execution because we want to reason about
transactions that span several contract runs. In Michelson
terminology, these transactions are called \emph{chained contract
  executions}, where an externally started contract run
initiates further internal contract runs.
Our DL design models the relevant parts of the
blockchain run-time system on top of the purely functional execution of
Michelson programs. On the level of the run-time system contracts are
very similar to objects: they are identified by an address and they
come with mutable attributes (state and balance). 


The DL treatment of the functional part of Michelson is quite
intuitive: programs are sequences of Michelson instructions, we model
the execution state of a Michelson program by a formula of the form $\Phi \to [p]\Psi$, and the proof
rules for $[i;p]\Psi$ (where $i$ is a single instruction) define the
semantics of symbolic execution.

Gas is an important aspect of computation on the blockchain. The
initial caller of a contract has to pay for executing the transaction
(consisting of one or more chained contract runs) in terms of gas.
A transaction that
runs out of gas is rolled back by the run-time system of the
blockchain as if it never happened. As Michelson does not
suffer from reentrancy problems (cf.\ Section~\ref{sec:michelson}),
gas does not affect reasoning about the functional correctness of
(chained) contract execution. For that reason,  our DL design does not
account for gas. 

It is the sole goal of this paper to provide a \textbf{machine
  verified specification} of symbolic execution for Michelson, rather than an
efficient or otherwise realistic implementation. For that reason, the
paper does not cover all instructions, but rather a carefully chosen
representative subset. This is in contrast to related work
\cite{micho,helmholtz,WHYtool} that describes \textbf{actual
verification tools}. To be useful for a wide range of programs, such a
tool must support as many Michelson instructions as
possible\footnote{Keeping up with the rapid evolution of the language is
  challenging for those tools. As of this writing, most of them support the instruction
  set available in late 2022.}, it must
be reasonably efficient, and it must deal with loops and
nontermination in an appropriate way. None of these issues are
concerns for our specification.

\subsection*{Contributions}
\label{sec:contributions}

\begin{enumerate}
\item We select a representative subset of Michelson
  instructions so as to provide proof templates for all 
  current and future instructions that work similarly.
% \item We define a core calculus for Michelson that covers
%   a representative subset of instructions.
\item We provide a parameterized semantics definition with instances
  for the concrete semantics as well as for an abstract semantics,
  which implements the dynamic logic for Michelson. 
\item We prove the soundness of this logic first for single programs,
  and then for several programs chained in a transaction.
\end{enumerate}

The Agda implementation of the contributions is
available.\footnote{\url{https://freidok.uni-freiburg.de/data/255176},
development version  \url{\repo}.}


\subsection*{Overview}
\label{sec:overview}

Section~\ref{sec:michelson} gives an overview of Michelson, introduces
its type system and our intrinsically typed representation of the language.
Section~\ref{sec:refImpl} defines the execution model of Michelson,
first for single contracts, and then for the chained execution of
several contracts that call each other.
Section~\ref{sec:DL} introduces dynamic logic and its
symbolic execution rules, again first for single execution, and then
for chained execution.
Section~\ref{sec:semantics-soundness} explains the major components of
the soundness proof of the dynamic logic.
Section~\ref{sec:related-work} discusses related work and conclusions.

The paper contains many excerpts from the live, type checked
definitions and proofs in Agda. In particular, all major definitions
and statements of theorems are shown in Agda notation to ensure
consistency of the paper with the machine-checked proofs.



%%% Local Variables:
%%% mode: latex
%%% TeX-master: "itp2024"
%%% End:
