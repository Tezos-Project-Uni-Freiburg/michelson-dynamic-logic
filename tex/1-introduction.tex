\section{Introduction}
\label{sec:introduction}

Blockchain technology and smart contracts provide decentralized and
immutable systems for secure transactions and automated agreements.
Smart contracts have been targets of spectacular and costly attacks as
contracts are immutable and their source code is publicly available on
the blockchain. 
Hence, it is vital as well as challenging to ensure the correctness of smart contracts
before their deployment. Formal methods and various verification
techniques have been proposed to address this challenge.

The Tezos blockchain \cite{tezos-whitepaper} and its smart contract
language Michelson have been designed from
ground up with verification in mind. Several frameworks have been
developed based, e.g., on interactive theorem proving \cite{micho},
refinement typing \cite{helmholtz}, and automatic theorem
proving \cite{WHYtool}. 
Symbolic execution~\cite{DBLP:journals/cacm/King76} is another important method of obtaining verification
conditions from a program. There is a wide range of approaches that
apply symbolic execution to smart contracts, but most of them are
directed towards the Ethereum blockchain.

While there are many approaches to symbolic execution
\cite{DBLP:conf/osdi/CadarDE08,DBLP:conf/icse/CsallnerTS08,Pasareanu2020},
we choose one based on dynamic logic.
Dynamic logic (DL) \cite{DL} is a modal logic for reasoning about
programs. Its key innovation lies in modalities for program
execution. These modalities enable the expression of assertions about
program behavior as logical formulas. For instance, the formula $[p]
\Psi$ states that if program $p$ terminates, then $\Psi$ is true. That
is, a Hoare triple $\{\Phi\}~p~\{ \Psi\}$ can be
encoded by $\Phi \to [p]\Psi$. DL also provides a possibility modality
$\langle p\rangle$, but we do not consider it in this work.

Dynamic logic comes with proof rules for the modality that correspond
to the structure of $p$. For example, if $p;q$ stands for sequential
execution of $p$ and $q$, then the proof rule $[p;q]\Psi
\leftrightarrow [p][q]\Psi$ states that execution of $p$ enables
execution of $q$ such that $\Psi$ holds in the end. Similarly, the
rule for the empty program is $[\varepsilon]\Psi \leftrightarrow \Psi$
as the empty program cannot affect the validity of $\Psi$.

\begin{comment}
  Such formulas are verified by successively reducing the program it
  contains until one is left with a purely first order formula that
  does not contain programs anymore, which can then be verified using
  the calculus of the first order logic that the DL is based on:
  \begin{align*}
    \langle i ; i' ; &\cdots ; i^n \rangle \phi
    \\ \leftrightarrow	\qquad		\langle     i' ; &\cdots ; i^n \rangle \phi'
    \\							 &\quad\vdots
    \\ \leftrightarrow	\qquad	\langle \rangle \phi^n	 &\leftrightarrow \phi^n
    \\							 &\quad\vdots
  \end{align*}

  % \section*{Scope and structure}
  This thesis focuses on these symbolic execution rules of the
  calculus and their soundness proof.  Firstly, we choose a
  representative subset of Michelson and give a reference
  implementation of it.  Then we define the symbolic execution rules
  for that subset and prove their soundness with respect to the
  reference implementation.
\end{comment}



In the past, dynamic
logic has been used successfully for symbolic execution in the context
of verification of Java programs \cite{KeY3}, as it is particularly
well suited to keep track of a changing environment (i.e., mutable
objects on Java's heap). 
In this work, we show that the approach based on dynamic logic is also
well suited to model the symbolic execution of smart contracts implemented
in Michelson.
Here, programs are sequences of Michelson instructions, we model the
execution state of a Michelson program by a formula of the form $\Phi \to [p]\Psi$, and the proof
rules for $[i;p]\Psi$ (where $i$ is a single instruction) define the
semantics of symbolic execution.



\subsection*{Contributions}
\label{sec:contributions}

\begin{enumerate}
\item We define a core calculus for Michelson that covers
  a representative subset of instructions.
\item We provide a parameterized semantics definition with instances
  for the concrete semantics as well as for an abstract semantics,
  which comprises the dynamic logic for Michelson. 
\item We prove the soundness of this logic first for single programs,
  and then for several programs chained in a transaction.
\end{enumerate}

The Agda implementation of the contributions is available in the
supplement and will be submitted for artefact evaluation.
% following repository: \url{\repo}. 

Our model includes token transfer between contracts, but we do not account for gas.
A transaction (consisting of one or more contract invocations) that
runs out of gas is rolled back by the run-time system of the
blockchain as if it never happened. Therefore, gas does not affect
reasoning about the functional correctness of contract execution.

It is the sole goal of this paper to provide a \textbf{machine
  verified specification} of symbolic execution for Michelson, rather than an
efficient or otherwise realistic implementation. 

\subsection*{Overview}
\label{sec:overview}

Section~\ref{sec:michelson} gives an overview of Michelson, introduces
its type system and our intrinsically typed representation of the language.
Section~\ref{sec:refImpl} defines the execution model of Michelson,
first for single contracts, and then for the chained execution of
several contracts that call each other.
Section~\ref{sec:DL} introduces dynamic logic and its
symbolic execution rules, again first for single execution, and then
for chained execution.
Section~\ref{sec:semantics-soundness} explains the major components of
the soundness proof of the dynamic logic.
Section~\ref{sec:related-work} discusses related work and conclusions.

The paper contains many excerpts from the live Agda implementation. In
particular, all major definitions and statements of theorems are shown
in Agda notation.


\begin{comment}
These kind of deductions are derived by giving a set of symbolic execution rules
for each Michelson instruction:
\[			\langle \phi , \text{instr; } p \rangle
	\Longrightarrow \langle \psi , p \rangle
\]

\section{Michelson}\label{sec:Michelson}

\[						15 :: 27 :: \text{\emph{remainingStack}}
	\overset{\text{ADD}}{\Longrightarrow}	      42 :: \text{\emph{remainingStack}}
\]
\[	\inferrule* 	[Right=ADD]
			{15 :: 27 :: \text{\emph{remainingStack}}}
			      {42 :: \text{\emph{remainingStack}}}
\]
Instructions for control flow take sub-programs as arguments
and are defined in big step semantics~\cite{devres}:
\[	\inferrule*	[Right=IF_{true}]
	{\text{then} : \text{\emph{StackIn}} \mapsto \text{\emph{StackOut}}}
	{\text{IF then else} : True :: \text{\emph{StackIn}} \mapsto \text{\emph{StackOut}}}
\]
			{\inferrule* 	[Right=\emph{then}]
					{\text{\emph{inStack}}}
					{\text{\emph{outStack}}}
			\\\\	{True :: \text{\emph{inStack}}}}
			{\text{\emph{outStack}}}

So a Michelson program is a function that converts a given input stack to an output stack
by successively applying each instruction.

\draft{A program can be type checked ``directly'', but for execution, the sub-programs of
control flow instructions must be expanded \ldots termination is guaranteed by gas consumption}

\subsection{Contract}

Smart contracts are characerized by their \verb=input= and \verb=storage= types,
and the Michelson program of each contract must be \emph{well typed} with regard to these types:
it has to accept a stack of a single element of type \verb=Pair input storage=
and produce a stack with an element of type \verb=Pair (list operations) storage=.

= = = = = = = = = = = = = = = = = = = = = = = = = = = = = = = = = = = = = = = = = = = = = = = = 

Every account on the tezos blockchain has a specific address and holds some amount of tezos tokens.
Smart contracts are accounts that additionally store a data value of a given type
and a Michelson program.
Also every contract expects an input value of a given type,
and the program has to be \emph{well typed} regarding the input and storage types of the contract:
It must take a stack of a single value,
which is a pair of the input value that is transmitted when calling the contract
and the data that the contract holds in storage.
It also must produce an output stack with a single value,
which has to be a pair of a list of emitted operations and an updated storage value.
So well typed means that the program must convert a stack of type \verb=Pair input storage=
into a stack of a value of type \verb=Pair (list operations) storage=.

Lets look at an example:

% \begin{lstlisting}[caption=Michelson: Add parameter to storage]
\begin{verbatim}
        UNPAIR;
        ADD;
        NIL operation;
        PAIR
\end{verbatim}
\todo{explain this simple example (and maybe typing with it \ldots)}
\begin{listing}[!ht]
\begin{verbatim}
        UNPAIR;
        ADD;
        NIL operation;
        PAIR
\end{verbatim}
\caption{maybe this is nicer as a listing \ldots}
\end{listing}

%CHAT
Michelson is designed to be a low-level, simple, and formally verifiable language to ensure the security and integrity of smart contracts on the Tezos network. Developers typically use high-level languages like LLL, SmartPy, or Ligo to compile down to Michelson for deploying smart contracts on Tezos

\section{scope and layout of this thesis}

This thesis focuses on the symbolic execution rules of the Dynamic Logic for Michelson
and will only give and prove those rules.
The DL as well as its reference implementation for the soundness proof are given as Agda modules.
Specifications for Michelson are taken from~\cite{Mref, devres}.

\draft{An earlier version of the developed code did include some rules for deductions on \ldots}

The DL works on two levels: It can derive statements/conclusions for single program fragments
\draft{that can be defined for any input output stack}.
This framework is later expanded to also work for contract execution chains
where more than one single contract is executed and each contract may also trigger
subsequent executions of other contracts.
On this level however, program fragments by themself cannot be considered,
only entire programs for well-typed contracts.

We will first explain some of the ratiolane behind our design choices in Agda
and introduce our typing system for Michelson.
Then we will explain the reference implementation before diving into the different
types of symbolic execution rules.
Lastly their soundness will be shown, followed by a concluding discussion and related work.

\section{Michelson coverage}

Also, since Michelson is still a bit too big to cover every instruction,
only a portion of it has been implemented,
focusing on those instructions that cover all the complexity of Michelson
while keeping the efford managable, but also keeping the possibility in mind
to expand the resulting work to cover further instructions and data types.

We included \verb=pair=, \verb=list= and \verb=option= from the complex types in Michelson,
as well as \verb=operation= (abbreviated to \verb=ops= in the Agda code), \verb=contract=
and the primitive data types \verb=unit= \verb=nat= \verb=mutez= and \verb=address=
to be able to create somewhat interesting example programs.

The instructions \verb=TRANSFER TOKENS= and \verb=CONTRACT= has been implemented
to demonstrate the execution of contracts triggered by other contracts
and \verb=ITER= was implemented to showcase the ability to consume complex data types like lists.
Branching is showcased with the \verb=IF_NONE= instruction which consumes the \verb=option=
data type that comes up a lot when dealing with other contracts.

There are several features regarding the Tezos blockchain that would require
an extension of the given model, like Sapling or Tickets, but the
already accomplished extension to enable execution of more than one contract
should serve as a flexible foundation for further extensions regarding such features.

Besides blockchain related features, the subset of Michelson that has been implemented
is believed to cover the entire comlexity of the language
(which is simple by designe and only comes from its complex data types),
with only one notable restriction:
The \verb=lambda= data type and its corresponding instructions are not implemented in our model.
We believe it should be possible to integrating it,
however it will certainly necessitate some extensions that may require significant reworkings.

\end{comment}

%%% Local Variables:
%%% mode: latex
%%% TeX-master: "itp2024"
%%% End:
